\documentclass[12pt]{article}
\usepackage{pmmeta}
\pmcanonicalname{StandardEnumeration}
\pmcreated{2013-03-22 13:33:12}
\pmmodified{2013-03-22 13:33:12}
\pmowner{mathcam}{2727}
\pmmodifier{mathcam}{2727}
\pmtitle{standard enumeration}
\pmrecord{21}{34158}
\pmprivacy{1}
\pmauthor{mathcam}{2727}
\pmtype{Definition}
\pmcomment{trigger rebuild}
\pmclassification{msc}{68Q45}
\pmclassification{msc}{03B65}
\pmsynonym{lexicographic enumeration}{StandardEnumeration}
%\pmkeywords{standard enumeration}
%\pmkeywords{language}
%\pmkeywords{characteristic function}
%\pmkeywords{characteristic sequence}
\pmdefines{characteristic function}
\pmdefines{characteristic sequence}

\usepackage{amssymb}
\usepackage{amsmath}
\usepackage{amsfonts}
\begin{document}
\PMlinkescapeword{natural}
\PMlinkescapeword{order}

The \emph{standard enumeration} of $\lbrace 0,1\rbrace ^{*}$ is the sequence of strings $s_0 =\lambda$, $s_1 = 0$, $s_2 = 1$, $s_3 = 00$, $s_4 = 01$, $\cdots$ in lexicographic order.

The \emph{characteristic function} of a language $A$ is $\chi_{A}:\mathbb{N}\rightarrow \lbrace 0, 1\rbrace$ such that 
\[\chi_{A}(n)=\begin{cases}
1,\text{ if }s_n \in A\\
0,\text{ if }s_n \notin A.
\end{cases}\]
The \emph{characteristic sequence} of a language $A$ (also denoted as $\chi_A$) is the concatenation of the values of the characteristic function in the natural order.
%%%%%
%%%%%
\end{document}
