\documentclass[12pt]{article}
\usepackage{pmmeta}
\pmcanonicalname{InvarianceTheorem}
\pmcreated{2013-03-22 13:43:54}
\pmmodified{2013-03-22 13:43:54}
\pmowner{rspuzio}{6075}
\pmmodifier{rspuzio}{6075}
\pmtitle{invariance theorem}
\pmrecord{7}{34420}
\pmprivacy{1}
\pmauthor{rspuzio}{6075}
\pmtype{Theorem}
\pmcomment{trigger rebuild}
\pmclassification{msc}{68Q30}

% this is the default PlanetMath preamble.  as your knowledge
% of TeX increases, you will probably want to edit this, but
% it should be fine as is for beginners.

% almost certainly you want these
\usepackage{amssymb}
\usepackage{amsmath}
\usepackage{amsfonts}

% used for TeXing text within eps files
%\usepackage{psfrag}
% need this for including graphics (\includegraphics)
%\usepackage{graphicx}
% for neatly defining theorems and propositions
%\usepackage{amsthm}
% making logically defined graphics
%%%\usepackage{xypic}

% there are many more packages, add them here as you need them

% define commands here
\begin{document}
The invariance theorem states that a universal Turing machine provides an 
optimal means of description, up to a constant. Formally, for every 
Turing machine $M$ there exists a constant $c$ such that for
all binary strings $x$ we have \[C_U(x) \leq C_M(x) + c.\]
Here, $C_U$ means the complexity with respect to the universal
Turing machine and $C_M$ means the complexity with respect to $M$.

This follows trivially from the definition of a universal Turing machine,
taking $c = l(<M>)$ as the length of the encoding of $M$.

The invariance theorem holds likewise for prefix and conditional complexities.
%%%%%
%%%%%
\end{document}
