\documentclass[12pt]{article}
\usepackage{pmmeta}
\pmcanonicalname{Iterator}
\pmcreated{2013-03-22 16:25:16}
\pmmodified{2013-03-22 16:25:16}
\pmowner{PrimeFan}{13766}
\pmmodifier{PrimeFan}{13766}
\pmtitle{iterator}
\pmrecord{6}{38571}
\pmprivacy{1}
\pmauthor{PrimeFan}{13766}
\pmtype{Definition}
\pmcomment{trigger rebuild}
\pmclassification{msc}{68N15}

% this is the default PlanetMath preamble.  as your knowledge
% of TeX increases, you will probably want to edit this, but
% it should be fine as is for beginners.

% almost certainly you want these
\usepackage{amssymb}
\usepackage{amsmath}
\usepackage{amsfonts}

% used for TeXing text within eps files
%\usepackage{psfrag}
% need this for including graphics (\includegraphics)
%\usepackage{graphicx}
% for neatly defining theorems and propositions
%\usepackage{amsthm}
% making logically defined graphics
%%%\usepackage{xypic}

% there are many more packages, add them here as you need them

% define commands here

\begin{document}
An \emph{iterator} is a variable that one by one takes on all the values (usually integers) in a given range. The letter $i$ is often used for iterators (though that letter is overloaded as the imaginary unit), $j$ is often used as a second iterator.

In mathematics iterators are most often used in sum and product notation. For example, in $$\prod_{i = 3}^7 42i^2$$ the iterator $i$ takes on values from 3 to 7.

In computer programming languages iterators are often associated with the "for" reserved word, e.g., ``\verb!FOR I = 1 TO 47!'' in BASIC, ``\verb!for (i = 1; i < 48; i++)!'' in C++. In Mathematica, iterators are often given in curly braces together with an (omittable) initial value and a required finish value separated by commas, e.g., ``\verb=Table[2^n - 1, {n, 2, 48}]=''.
%%%%%
%%%%%
\end{document}
