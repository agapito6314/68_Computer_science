\documentclass[12pt]{article}
\usepackage{pmmeta}
\pmcanonicalname{PolynomialHierarchyIsAHierarchy}
\pmcreated{2013-03-22 13:02:14}
\pmmodified{2013-03-22 13:02:14}
\pmowner{uzeromay}{4983}
\pmmodifier{uzeromay}{4983}
\pmtitle{polynomial hierarchy is a hierarchy}
\pmrecord{5}{33438}
\pmprivacy{1}
\pmauthor{uzeromay}{4983}
\pmtype{Result}
\pmcomment{trigger rebuild}
\pmclassification{msc}{68Q15}

\endmetadata

% this is the default PlanetMath preamble.  as your knowledge
% of TeX increases, you will probably want to edit this, but
% it should be fine as is for beginners.

% almost certainly you want these
\usepackage{amssymb}
\usepackage{amsmath}
\usepackage{amsfonts}

% used for TeXing text within eps files
%\usepackage{psfrag}
% need this for including graphics (\includegraphics)
%\usepackage{graphicx}
% for neatly defining theorems and propositions
%\usepackage{amsthm}
% making logically defined graphics
%%%\usepackage{xypic}

% there are many more packages, add them here as you need them

% define commands here
%\PMlinkescapeword{theory}
\begin{document}
The polynomial hierarchy is a hierarchy.  Specifically: \[\Sigma^p_i\cup\Pi^p_i\subseteq\Delta^p_{i+1}\subseteq\Sigma^p_{i+1}\cap\Pi^p_{i+1}.\]

\section*{Proof}

To see that $\Sigma^p_i\cup\Pi^p_i\subseteq\Delta^p_{i+1}=\mathcal{P}^{\Sigma^p_i}$, 
observe that the machine which checks its input against its oracle and accepts or rejects when the oracle accepts or rejects (respectively) is easily in $\mathcal{P}$, as is the machine which rejects or accepts when the oracle accepts or rejects (respectively).  These easily emulate $\Sigma^p_i$ and $\Pi^p_i$ respectively.

Since $\mathcal{P}\subseteq\mathcal{NP}$, it is clear that 
$\Delta^p_i\subseteq\Sigma^p_i$.  Since $\mathcal{P}^{\mathcal{C}}$ is 
closed 
under complementation for any 
complexity class 
$\mathcal{C}$ (the associated 
machines are deterministic and always halt, so the complementary machine just 
reverses which states are accepting), if 
$L\in\mathcal{P}^{\Sigma^p_i}\subseteq\Sigma^p_i$ then so is $\overline{L}$, 
and therefore $L\in\Pi^p_i$.



Unlike the arithmetical hierarchy, the polynomial hierarchy is not known to be proper.  Indeed, if $\mathcal{P}=\mathcal{NP}$ then $\mathcal{P}=\mathcal{PH}$, so a proof that the hierarchy is proper would be quite significant.
%%%%%
%%%%%
\end{document}
