\documentclass[12pt]{article}
\usepackage{pmmeta}
\pmcanonicalname{SearchTree}
\pmcreated{2013-03-22 17:22:02}
\pmmodified{2013-03-22 17:22:02}
\pmowner{Mathprof}{13753}
\pmmodifier{Mathprof}{13753}
\pmtitle{search tree}
\pmrecord{5}{39728}
\pmprivacy{1}
\pmauthor{Mathprof}{13753}
\pmtype{Definition}
\pmcomment{trigger rebuild}
\pmclassification{msc}{68P10}
\pmclassification{msc}{68P05}

\endmetadata

% this is the default PlanetMath preamble.  as your knowledge
% of TeX increases, you will probably want to edit this, but
% it should be fine as is for beginners.

% almost certainly you want these
\usepackage{amssymb}
\usepackage{amsmath}
\usepackage{amsfonts}

% used for TeXing text within eps files
%\usepackage{psfrag}
% need this for including graphics (\includegraphics)
%\usepackage{graphicx}
% for neatly defining theorems and propositions
%\usepackage{amsthm}
% making logically defined graphics
%%%\usepackage{xypic}

% there are many more packages, add them here as you need them

% define commands here

\begin{document}
A \emph{search tree} is a tree where every subtree of a node has keys 
less than any other subtree of the node to its right. The keys in a node are 
conceptually between subtrees and are greater than any keys in subtrees to its 
left and less than any keys in subtrees to its right.

A search might  visit nodes in the tree to locate some information. At the outset of
the search one has a given key and seeks to find the node in the tree having that key.
As the nodes are visited a comparison of the node's key to the given key is made
and a decision to go to either the left subtree or the right subtree is made.
Knowing that all keys to the left (right)  are smaller (larger) makes
the search easy to carry out.

%%%%%
%%%%%
\end{document}
