\documentclass[12pt]{article}
\usepackage{pmmeta}
\pmcanonicalname{RelationConstruction}
\pmcreated{2013-03-22 17:49:31}
\pmmodified{2013-03-22 17:49:31}
\pmowner{Jon Awbrey}{15246}
\pmmodifier{Jon Awbrey}{15246}
\pmtitle{relation construction}
\pmrecord{6}{40289}
\pmprivacy{1}
\pmauthor{Jon Awbrey}{15246}
\pmtype{Topic}
\pmcomment{trigger rebuild}
\pmclassification{msc}{68P15}
\pmclassification{msc}{08A02}
\pmclassification{msc}{05C65}
\pmclassification{msc}{05B30}
\pmclassification{msc}{05B20}
\pmclassification{msc}{03E20}
\pmclassification{msc}{03B10}
\pmclassification{msc}{68R01}
\pmrelated{RelationTheory}
\pmrelated{RelationComposition2}
\pmrelated{RelationReduction}
\pmrelated{TacitExtension}

% this is the default PlanetMath preamble.  as your knowledge
% of TeX increases, you will probably want to edit this, but
% it should be fine as is for beginners.

% almost certainly you want these
\usepackage{amssymb}
\usepackage{amsmath}
\usepackage{amsfonts}

% used for TeXing text within eps files
%\usepackage{psfrag}
% need this for including graphics (\includegraphics)
%\usepackage{graphicx}
% for neatly defining theorems and propositions
%\usepackage{amsthm}
% making logically defined graphics
%%%\usepackage{xypic}

% there are many more packages, add them here as you need them

% define commands here

\begin{document}
\PMlinkescapephrase{base}
\PMlinkescapephrase{Base}
\PMlinkescapephrase{components}
\PMlinkescapephrase{Components}
\PMlinkescapephrase{plus}
\PMlinkescapephrase{Plus}

\textbf{Relation construction} and \textbf{relational constructibility} have to do with the ways that one relation is determined by an indexed family or a sequence of other relations, called the \textit{relation dataset}.  The relation dataset typically consists of a specified set of other relations, called the \textit{relational base} or the \textit{relational components}, plus a specified set of operations on relations, called the \textit{method of construction}. 

Relation composition and relation reduction are special cases of relation constructions.

%%%%%
%%%%%
\end{document}
