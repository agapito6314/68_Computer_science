\documentclass[12pt]{article}
\usepackage{pmmeta}
\pmcanonicalname{NondeterministicTuringMachine}
\pmcreated{2013-03-22 13:01:25}
\pmmodified{2013-03-22 13:01:25}
\pmowner{Henry}{455}
\pmmodifier{Henry}{455}
\pmtitle{non-deterministic Turing machine}
\pmrecord{6}{33416}
\pmprivacy{1}
\pmauthor{Henry}{455}
\pmtype{Definition}
\pmcomment{trigger rebuild}
\pmclassification{msc}{68Q05}
\pmrelated{TuringMachine2}
\pmdefines{certificate}

% this is the default PlanetMath preamble.  as your knowledge
% of TeX increases, you will probably want to edit this, but
% it should be fine as is for beginners.

% almost certainly you want these
\usepackage{amssymb}
\usepackage{amsmath}
\usepackage{amsfonts}

% used for TeXing text within eps files
%\usepackage{psfrag}
% need this for including graphics (\includegraphics)
%\usepackage{graphicx}
% for neatly defining theorems and propositions
%\usepackage{amsthm}
% making logically defined graphics
%%%\usepackage{xypic}

% there are many more packages, add them here as you need them

% define commands here
%\PMlinkescapeword{theory}
\begin{document}
The definition of a non-deterministic Turing machine is the same as the definition of a deterministic Turing machine except that $\delta$ is a relation,  not a function.  Hence, for any particular state and symbol, there may be multiple possible legal moves.

If $S\in\Gamma^+$ we say $T$ accepts $S$ if, when $S$ is the input, there is some finite sequence of legal moves such that $\delta$ is undefined on the state and symbol pair which results from the last move in the sequence and such that the final state is an element of $F$.  If $T$ does not accept $S$ then it rejects $S$.

An alternative definition of a non-deterministic Turing machine is as a deterministic Turing machine with an extra one-way, read-only tape, the guess tape.  Then we say $T$ accepts $S$ if there is any string $c(S)$ such that, when $c(S)$ is placed on the guess tape, $T$ accepts $S$.  We call $c(S)$ a \emph{certificate} for $S$, and otherwise that it rejects $S$.  In some cases the guess tape is allowed to be two-way; this generates different time and space complexity classes than the one-way case (the one-way case is equivalent to the original definition).
%%%%%
%%%%%
\end{document}
