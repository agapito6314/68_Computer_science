\documentclass[12pt]{article}
\usepackage{pmmeta}
\pmcanonicalname{CountingProblem}
\pmcreated{2013-03-22 13:02:17}
\pmmodified{2013-03-22 13:02:17}
\pmowner{Henry}{455}
\pmmodifier{Henry}{455}
\pmtitle{counting problem}
\pmrecord{6}{33439}
\pmprivacy{1}
\pmauthor{Henry}{455}
\pmtype{Definition}
\pmcomment{trigger rebuild}
\pmclassification{msc}{68Q25}

\endmetadata

% this is the default PlanetMath preamble.  as your knowledge
% of TeX increases, you will probably want to edit this, but
% it should be fine as is for beginners.

% almost certainly you want these
\usepackage{amssymb}
\usepackage{amsmath}
\usepackage{amsfonts}

% used for TeXing text within eps files
%\usepackage{psfrag}
% need this for including graphics (\includegraphics)
%\usepackage{graphicx}
% for neatly defining theorems and propositions
%\usepackage{amsthm}
% making logically defined graphics
%%%\usepackage{xypic}

% there are many more packages, add them here as you need them

% define commands here
%\PMlinkescapeword{theory}
\begin{document}
If $R$ is a search problem then $c_R(x)=|\{y\mid R(x)\}|$ is the corresponding counting function and $\#R=\{(x,y)\mid y\leq c_R(x)\}$ denotes the corresponding \emph{counting problem}.  Note that $c_R$ is a search problem while $\#R$ is a decision problem, however $c_R$ can be $\mathcal{C}$ Cook reduced to $\#R$ (for appropriate $\mathcal{C}$) using a binary search (the reason $\#R$ is defined the way it is, rather than being the graph of $c_R$, is to make this binary search possible).
%%%%%
%%%%%
\end{document}
