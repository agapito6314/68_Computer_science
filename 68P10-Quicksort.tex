\documentclass[12pt]{article}
\usepackage{pmmeta}
\pmcanonicalname{Quicksort}
\pmcreated{2013-03-22 13:59:19}
\pmmodified{2013-03-22 13:59:19}
\pmowner{thouis}{3290}
\pmmodifier{thouis}{3290}
\pmtitle{quicksort}
\pmrecord{13}{34763}
\pmprivacy{1}
\pmauthor{thouis}{3290}
\pmtype{Definition}
\pmcomment{trigger rebuild}
\pmclassification{msc}{68P10}
%\pmkeywords{comparison sort}
%\pmkeywords{quick-sort}
\pmrelated{SortingProblem}

\endmetadata

% this is the default PlanetMath preamble.  as your knowledge
% of TeX increases, you will probably want to edit this, but
% it should be fine as is for beginners.

% almost certainly you want these
\usepackage{amssymb}
\usepackage{amsmath}
\usepackage{amsfonts}

% used for TeXing text within eps files
%\usepackage{psfrag}
% need this for including graphics (\includegraphics)
%\usepackage{graphicx}
% for neatly defining theorems and propositions
%\usepackage{amsthm}
% making logically defined graphics
%%%\usepackage{xypic}

% there are many more packages, add them here as you need 

% define commands here
\begin{document}
\newcommand{\Lindent}{0.4in}
\newenvironment{Lalgorithm}[4]{
\textbf{Algorithm} \textsc{#1}\texttt{(#2)}\newline
\textit{Input}: #3\newline
\textit{Output}: #4\newline
}{}
\newenvironment{Lfloatalgorithm}[6][h]{
\begin{figure}[#1]
\caption{#2}
\begin{Lalgorithm}{#3}{#4}{#5}{#6}
}{
\end{Lalgorithm}
\end{figure}
}
\newcommand{\Lgets}{\ensuremath{\gets}}
\newcommand{\Lgroup}[1]{\textbf{begin}\\\hspace*{\Lindent}\parbox{\textwidth}{#1}\\\textbf{end}}
\newcommand{\Lif}[2]{\textbf{if} #1 \textbf{then}\\\hspace*{\Lindent}\parbox{\textwidth}{#2}}
\newcommand{\Lelse}[1]{\textbf{else}\\\hspace*{\Lindent}\parbox{\textwidth}{#1}}
\newcommand{\Lelseif}[2]{\textbf{else if} #1 \textbf{then}\\\hspace*{\Lindent}\parbox{\textwidth}{#2}}
\newcommand{\Lfor}[2]{\textbf{for} #1 \textbf{do}\\\hspace*{\Lindent}\parbox{\textwidth}{#2}}


Quicksort is a divide-and-conquer algorithm for sorting in the
comparison model.    Its expected running time is $O(n\lg n)$ for
sorting $n$ values. 


\subsection*{Algorithm}

Quicksort can be implemented recursively, as follows:



\begin{Lalgorithm}{Quicksort}{$L$}{A list $L$ of $n$ elements}{The list $L$ in sorted order}
\Lif{$n > 1$}{
   $p \gets \mbox{ random element of } L$\\
   $A \gets \{x | x \in L, x < p\}$\\
   $B \gets \{z | z \in L, z = p\}$\\
   $C \gets \{y | y \in L, y > p\}$\\
   $S_A \gets Quicksort(A)$\\
   $S_C \gets Quicksort(C)$\\
   {\bf return} $Concatenate(S_A,B,S_C)$
}\\
\Lelse{{\bf return} $L$}
\end{Lalgorithm}

\subsection*{Analysis}
The behavior of quicksort can be analyzed by considering the
computation as a binary tree.  Each node of the tree corresponds to
one recursive call to the quicksort procedure.   

Consider the initial input to the algorithm, some list $L$.  Call the
Sorted list $S,$ with $i$th and $j$th elements $S_i$ and $S_j.$  These
two elements will be compared with some probability $p_{ij}$.  This
probability can be determined by considering two preconditions on
$S_i$ and $S_j$ being compared: 
\begin{itemize}
\item $S_i$ or $S_j$ must be chosen as a pivot $p$, since comparisons
only occur against the pivot.
\item No element between $S_i$ and $S_j$ can have already been chosen
as a pivot before $S_i$ or $S_j$ is chosen. Otherwise, would be
separated into different sublists in the recursion.
\end{itemize}

The probability of any particular element being chosen as the pivot is
uniform.  Therefore, the chance that $S_i$ or $S_j$ is chosen as the
pivot before any element between them is $2/(j-i+1)$.  This is
precisely $p_{ij}.$

The expected number of comparisons is just the summation over all
possible comparisons of the probability of that particular comparison
occurring.  By linearity of expectation, no independence assumptions
are necessary.  The expected number of comparisons is therefore

\begin{eqnarray}
\sum_{i=1}^{n}\sum_{j>i}^{n}p_{ij} & = & \sum_{i=1}^{n}\sum_{j>i}^{n}\frac{2}{j-i+1} \\
 & = & \sum_{i=1}^{n}\sum_{k=2}^{n-i+1}\frac{2}{k}\\
 & \le & 2\sum_{i=1}^{n}\sum_{k=1}^{n}\frac{1}{k}\\
 & = & 2nH_n = O(n\lg n),
\end{eqnarray}

where $H_n$ is the $n$th Harmonic number.

The worst case behavior is $\Theta(n^2)$, but this almost never occurs (with high probability it does not occur) with random pivots.
%%%%%
%%%%%
\end{document}
