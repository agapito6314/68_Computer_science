\documentclass[12pt]{article}
\usepackage{pmmeta}
\pmcanonicalname{MyhillNerodeTheorem}
\pmcreated{2013-03-22 18:52:13}
\pmmodified{2013-03-22 18:52:13}
\pmowner{Ziosilvio}{18733}
\pmmodifier{Ziosilvio}{18733}
\pmtitle{Myhill-Nerode theorem}
\pmrecord{14}{41712}
\pmprivacy{1}
\pmauthor{Ziosilvio}{18733}
\pmtype{Theorem}
\pmcomment{trigger rebuild}
\pmclassification{msc}{68Q70}
\pmclassification{msc}{20M35}
\pmsynonym{Myhill-Nerode theorem for languages}{MyhillNerodeTheorem}

% this is the default PlanetMath preamble.  as your knowledge
% of TeX increases, you will probably want to edit this, but
% it should be fine as is for beginners.

% almost certainly you want these
\usepackage{amssymb}
\usepackage{amsmath}
\usepackage{amsfonts}

% used for TeXing text within eps files
%\usepackage{psfrag}
% need this for including graphics (\includegraphics)
%\usepackage{graphicx}
% for neatly defining theorems and propositions
%\usepackage{amsthm}
% making logically defined graphics
%%%\usepackage{xypic}

% there are many more packages, add them here as you need them

% define commands here

\begin{document}
Let $L$ be a language on the finite alphabet $A$
and let $\mathcal{N}_L$ be its Nerode equivalence.
The following are equivalent.
\begin{enumerate}
\item
$L$ is recognized by a deterministic finite automaton.
\item
$A^\ast/\mathcal{N}_L$ is finite.
\end{enumerate}
Moreover, the number of classes of $\mathcal{N}_L$
is the smallest number of states of a DFA recognizing $L$.

\textit{Proof.}
First, suppose
\begin{math}
A^\ast/\mathcal{N}_L=\{q_0=[\lambda]_{\mathcal{N}_L},\ldots,q_{k-1}\}=Q,
\end{math}
where $\lambda$ is the empty word on $A$.
Construct a DFA $\mathcal{A}=\left<Q,A,q_0,\delta,F\right>$
(called the \emph{Nerode automaton} for $L$)
with $\delta:Q\times A\to Q$ defined by
\begin{equation} \label{eq:map}
\delta(q,a)=[wa]_{\mathcal{N}_L}\;,\;\;w\in q\;,
\end{equation}
and
\begin{equation} \label{eq:final}
F=\{q\in Q\mid\exists w\in L\mid w\in q\}\;.
\end{equation}
Then $\delta$ is well defined
because $w_1\mathcal{N}_Lw_2$ implies $w_1u\mathcal{N}_Lw_2u$.
It is also straightforward that $\mathcal{A}$ recognizes $L$.

On the other hand,
let $\mathcal{A}=\left<Q,A,q_0,\delta,F\right>$
be a DFA that recognizes $L$.
Extend $\delta$ to $Q\times A^\ast$ by putting
$\delta(q,\lambda)=q$ and
$\delta(q,aw)=\delta(\delta(q,a),w)$
for every $q\in Q$, $a\in A$, $w\in A^\ast$.
Define $f:Q\to A^\ast/\mathcal{N}_L\cup\{\emptyset\}$ as
\begin{equation} \label{eq:f}
f(q) = \left\{\begin{array}{ll}
[w]_{\mathcal{N}_L}
& \mathrm{if}\;\delta(q_0,w)=q \\
\emptyset
& \mathrm{if}\;\delta(q_0,w)\neq q\forall w\in A^\ast
\end{array}
\right.
\end{equation}
Then $f$ is well defined.
In fact, suppose $q_1=q_2=q$:
then either $f(q_1)=f(q_2)=\emptyset$,
or there are $w_1,w_2\in A^\ast$ such that $\delta(q_0,w_1)=\delta(q_0,w_2)=q$.
But in the latter case, $\delta(q_0,w_1u)=\delta(q_0,w_2u)=\delta(q,u)$
for any $u\in A^\ast$,
hence $w_1\mathcal{N}_Lw_2$ since $\mathcal{A}$ recognizes $L$.
Finally, for any $w\in A^\ast$ we have
\begin{math}
[w]_{\mathcal{N}_L}=f\left(\delta(q_0,w)\right),
\end{math}
so every class of $\mathcal{N}_L$ has a preimage according to $f$;
consequently, $|Q|\geq|A^\ast/\mathcal{N}_L|$.
\hfill$\Box$

%%%%%
%%%%%
\end{document}
