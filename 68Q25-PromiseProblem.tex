\documentclass[12pt]{article}
\usepackage{pmmeta}
\pmcanonicalname{PromiseProblem}
\pmcreated{2013-03-22 13:02:27}
\pmmodified{2013-03-22 13:02:27}
\pmowner{Henry}{455}
\pmmodifier{Henry}{455}
\pmtitle{promise problem}
\pmrecord{4}{33443}
\pmprivacy{1}
\pmauthor{Henry}{455}
\pmtype{Definition}
\pmcomment{trigger rebuild}
\pmclassification{msc}{68Q25}

% this is the default PlanetMath preamble.  as your knowledge
% of TeX increases, you will probably want to edit this, but
% it should be fine as is for beginners.

% almost certainly you want these
\usepackage{amssymb}
\usepackage{amsmath}
\usepackage{amsfonts}

% used for TeXing text within eps files
%\usepackage{psfrag}
% need this for including graphics (\includegraphics)
%\usepackage{graphicx}
% for neatly defining theorems and propositions
%\usepackage{amsthm}
% making logically defined graphics
%%%\usepackage{xypic}

% there are many more packages, add them here as you need them

% define commands here
%\PMlinkescapeword{theory}
\begin{document}
A \emph{promise problem} is a generalization of a decision problem.  It is defined by two decision problems $L_1$ and $L_2$ with $L_1\cap L_2=\emptyset$.  A Turing machine decides a promise problem if, for any $x\in L_1\cup L_2$, it accepts when $x\in L_1$ and rejects if $x\in L_2$.  Behavior is undefined when $x\notin L_1\cup L_2$ (this is the promise: that $x$ is in one of the two sets).

If $L_2=\Gamma^+\setminus L_1$ then this is just the decision problem for $L_1$.
%%%%%
%%%%%
\end{document}
