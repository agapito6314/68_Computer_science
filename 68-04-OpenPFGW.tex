\documentclass[12pt]{article}
\usepackage{pmmeta}
\pmcanonicalname{OpenPFGW}
\pmcreated{2013-03-22 16:36:59}
\pmmodified{2013-03-22 16:36:59}
\pmowner{PrimeFan}{13766}
\pmmodifier{PrimeFan}{13766}
\pmtitle{OpenPFGW}
\pmrecord{7}{38814}
\pmprivacy{1}
\pmauthor{PrimeFan}{13766}
\pmtype{Topic}
\pmcomment{trigger rebuild}
\pmclassification{msc}{68-04}
\pmsynonym{PrimeForm}{OpenPFGW}

\endmetadata

% this is the default PlanetMath preamble.  as your knowledge
% of TeX increases, you will probably want to edit this, but
% it should be fine as is for beginners.

% almost certainly you want these
\usepackage{amssymb}
\usepackage{amsmath}
\usepackage{amsfonts}

% used for TeXing text within eps files
%\usepackage{psfrag}
% need this for including graphics (\includegraphics)
%\usepackage{graphicx}
% for neatly defining theorems and propositions
%\usepackage{amsthm}
% making logically defined graphics
%%%\usepackage{xypic}

% there are many more packages, add them here as you need them

% define commands here

\begin{document}
{\em OpenPFGW}, formerly called {\em PrimeForm}, is an open source primality testing program on the Intel platform that has found many of the large primes listed in the Prime Pages. It uses sieve methods, pseudorandom permutation (PRP) as well as methods for specific kinds of primes, as programmed by George Woltman for GIMPS. As of 2006, the largest prime it has found is $19 \times 2^{2206266} + 1$.

According to the Prime Pages, the official website is \PMlinkexternal{a Yahoo! Group}{http://tech.groups.yahoo.com/group/primeform/}.

%%%%%
%%%%%
\end{document}
