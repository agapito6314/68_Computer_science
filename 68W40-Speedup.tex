\documentclass[12pt]{article}
\usepackage{pmmeta}
\pmcanonicalname{Speedup}
\pmcreated{2013-03-22 12:04:12}
\pmmodified{2013-03-22 12:04:12}
\pmowner{akrowne}{2}
\pmmodifier{akrowne}{2}
\pmtitle{speedup}
\pmrecord{6}{31132}
\pmprivacy{1}
\pmauthor{akrowne}{2}
\pmtype{Definition}
\pmcomment{trigger rebuild}
\pmclassification{msc}{68W40}
\pmclassification{msc}{68W10}
%\pmkeywords{parallel computing}

\endmetadata

\usepackage{amssymb}
\usepackage{amsmath}
\usepackage{amsfonts}
\usepackage{graphicx}
%%%\usepackage{xypic}
\begin{document}
\emph{Speedup} is a way to quantify the advantage of using a parallel algorithm over a sequential algorithm.  The speedup $S$ is defined as

$$ S = \frac{R}{P} $$
Where $R$ is the running time of the best available sequential algorithm and $P$ is the running time of the parallel algorithm.  

Ideally, on a system with $N$ processors, the speedup for any algorithm would be $N$.  Amdahl's law deals with the speedup in more realistic situations.
%%%%%
%%%%%
%%%%%
\end{document}
