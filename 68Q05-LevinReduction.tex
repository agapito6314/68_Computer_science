\documentclass[12pt]{article}
\usepackage{pmmeta}
\pmcanonicalname{LevinReduction}
\pmcreated{2013-03-22 13:01:44}
\pmmodified{2013-03-22 13:01:44}
\pmowner{Henry}{455}
\pmmodifier{Henry}{455}
\pmtitle{Levin reduction}
\pmrecord{5}{33427}
\pmprivacy{1}
\pmauthor{Henry}{455}
\pmtype{Definition}
\pmcomment{trigger rebuild}
\pmclassification{msc}{68Q05}
\pmclassification{msc}{68Q10}

\endmetadata

% this is the default PlanetMath preamble.  as your knowledge
% of TeX increases, you will probably want to edit this, but
% it should be fine as is for beginners.

% almost certainly you want these
\usepackage{amssymb}
\usepackage{amsmath}
\usepackage{amsfonts}

% used for TeXing text within eps files
%\usepackage{psfrag}
% need this for including graphics (\includegraphics)
%\usepackage{graphicx}
% for neatly defining theorems and propositions
%\usepackage{amsthm}
% making logically defined graphics
%%%\usepackage{xypic}

% there are many more packages, add them here as you need them

% define commands here
%\PMlinkescapeword{theory}
\begin{document}
If $R_1$ and $R_2$ are search problems and $\mathcal{C}$ is a complexity class then a $\mathcal{C}$ \emph{Levin reduction} of $R_1$ to $R_2$ consists of three functions $g_1, g_2, g_3\in\mathcal{C}$ which satisfy:

\begin{itemize}
\item $g_1$ is a $\mathcal{C}$ Karp reduction of $L(R_1)$ to $L(R_2)$

\item If $R_1(x,y)$ then $R_2(f(x),g(x,y))$

\item If $R_2(f(x),z)$ then $R_1(x,h(x,z))$
\end{itemize}

Note that a $\mathcal{C}$ Cook reduction can be constructed by calculating $f(x)$, using the oracle to find $z$, and then calculating $h(x,z)$.

$\mathcal{P}$ Levin reductions are just called Levin reductions.
%%%%%
%%%%%
\end{document}
