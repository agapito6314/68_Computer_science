\documentclass[12pt]{article}
\usepackage{pmmeta}
\pmcanonicalname{Hypertext}
\pmcreated{2013-03-22 15:31:32}
\pmmodified{2013-03-22 15:31:32}
\pmowner{kahuja}{10742}
\pmmodifier{kahuja}{10742}
\pmtitle{Hypertext}
\pmrecord{8}{37403}
\pmprivacy{1}
\pmauthor{kahuja}{10742}
\pmtype{Definition}
\pmcomment{trigger rebuild}
\pmclassification{msc}{68P20}

% this is the default PlanetMath preamble.  as your knowledge
% of TeX increases, you will probably want to edit this, but
% it should be fine as is for beginners.

% almost certainly you want these
\usepackage{amssymb}
\usepackage{amsmath}
\usepackage{amsfonts}

% used for TeXing text within eps files
%\usepackage{psfrag}
% need this for including graphics (\includegraphics)
%\usepackage{graphicx}
% for neatly defining theorems and propositions
%\usepackage{amsthm}
% making logically defined graphics
%%%\usepackage{xypic}

% there are many more packages, add them here as you need them

% define commands here
\begin{document}
A hypertext is a kind of database system which allows creative linking of objects. It is essentially made by structural hyperlinks that follow the structural metadata and external referential links. Hypertext system is essentially used to better organize large information sources so that browsing through them is easier. An example of hypertext is the World Wide Web.

If H be a structure and C be a collection then hypertext HT = (H, Contents, P) is a triple such that:

(1) Contents $\subseteq$ C $\cup$ AllSubStreams $\cup$ AllSubStructuredStreams

(2) P is a function which associates a node of the hypertext with the node content.
%%%%%
%%%%%
\end{document}
