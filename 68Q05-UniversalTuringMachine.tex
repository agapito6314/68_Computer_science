\documentclass[12pt]{article}
\usepackage{pmmeta}
\pmcanonicalname{UniversalTuringMachine}
\pmcreated{2013-03-22 13:43:46}
\pmmodified{2013-03-22 13:43:46}
\pmowner{tromp}{1913}
\pmmodifier{tromp}{1913}
\pmtitle{universal Turing machine}
\pmrecord{7}{34416}
\pmprivacy{1}
\pmauthor{tromp}{1913}
\pmtype{Definition}
\pmcomment{trigger rebuild}
\pmclassification{msc}{68Q05}
\pmrelated{ArtificialInteglligence}
\pmrelated{StrongAIThesis}

\endmetadata

% this is the default PlanetMath preamble.  as your knowledge
% of TeX increases, you will probably want to edit this, but
% it should be fine as is for beginners.

% almost certainly you want these
\usepackage{amssymb}
\usepackage{amsmath}
\usepackage{amsfonts}

% used for TeXing text within eps files
%\usepackage{psfrag}
% need this for including graphics (\includegraphics)
%\usepackage{graphicx}
% for neatly defining theorems and propositions
%\usepackage{amsthm}
% making logically defined graphics
%%%\usepackage{xypic}

% there are many more packages, add them here as you need them

% define commands here
\begin{document}
A universal Turing machine $U$ is a Turing machine with a single binary
one-way read-only input tape, on which it expects to find the encoding of an arbitrary Turing machine $M$. The set of all Turing machine encodings must
be prefix-free, so that no special end-marker or `blank' is needed to recognize
a code's end. Having transferred the description of $M$ onto its worktape,
$U$ then proceeds to simulate the behaviour of $M$ on the remaining
contents of the input tape. If $M$ halts, then $U$ cleans up its worktape,
leaving it with just the output of $M$, and halts too.

If we denote by $M()$ the partial function computed by machine $M$,
and by $<M>$ the encoding of machine $M$ as a binary string,
then we have $U(<M>x)=M(x)$.

There are two kinds of universal Turing machine, depending on whether the
input tape alphabet of the simulated machine is $\{0,1,\#\}$ or just $\{0,1\}$.
The first kind is a {\em plain} Universal Turing machine;
while the second is a {\em prefix} Universal Turing machine,
which has the nice property
that the set of inputs on which it halts is prefix free.

The letter $U$ is commonly used to denote a fixed universal machine, whose
type is either mentioned explicitly or assumed clear from context.
%%%%%
%%%%%
\end{document}
