\documentclass[12pt]{article}
\usepackage{pmmeta}
\pmcanonicalname{KolmogorovComplexityFunction}
\pmcreated{2013-03-22 13:43:49}
\pmmodified{2013-03-22 13:43:49}
\pmowner{tromp}{1913}
\pmmodifier{tromp}{1913}
\pmtitle{Kolmogorov complexity function}
\pmrecord{9}{34418}
\pmprivacy{1}
\pmauthor{tromp}{1913}
\pmtype{Definition}
\pmcomment{trigger rebuild}
\pmclassification{msc}{68Q30}
\pmsynonym{algorithmic entropy}{KolmogorovComplexityFunction}
\pmsynonym{information content}{KolmogorovComplexityFunction}

\endmetadata

% this is the default PlanetMath preamble.  as your knowledge
% of TeX increases, you will probably want to edit this, but
% it should be fine as is for beginners.

% almost certainly you want these
\usepackage{amssymb}
\usepackage{amsmath}
\usepackage{amsfonts}

% used for TeXing text within eps files
%\usepackage{psfrag}
% need this for including graphics (\includegraphics)
%\usepackage{graphicx}
% for neatly defining theorems and propositions
%\usepackage{amsthm}
% making logically defined graphics
%%%\usepackage{xypic}

% there are many more packages, add them here as you need them

% define commands here
\begin{document}
The (plain) complexity $C(x)$ of a binary string $x$ is the length of a shortest
program $p$ such that $U(p)=x$, i.e. the (plain) universal Turing machine on
input $p$, outputs $x$ and halts. The lexicographically least such $p$ is
denoted $x^{\ast}$.
The prefix complexity $K(x)$ is defined similarly in terms of the prefix universal
machine. When clear from context, $x^{\ast}$ is also used to denote the
lexicographically least prefix program for $x$.

Plain and prefix conditional complexities $C(x|y), K(x|y)$ are defined similarly
but with $U(x|y)=x$, i.e. the universal machine starts out with $y$ written
on its worktape.

Subscripting these functions with a Turing machine $M$, as in
$K_M(x|y)$, denotes the corresponding complexity in which we use machine $M$
in place of the universal machine $U$.
%%%%%
%%%%%
\end{document}
