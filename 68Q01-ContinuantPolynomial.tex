\documentclass[12pt]{article}
\usepackage{pmmeta}
\pmcanonicalname{ContinuantPolynomial}
\pmcreated{2013-03-22 18:03:36}
\pmmodified{2013-03-22 18:03:36}
\pmowner{veselin}{20506}
\pmmodifier{veselin}{20506}
\pmtitle{Continuant polynomial}
\pmrecord{7}{40590}
\pmprivacy{1}
\pmauthor{veselin}{20506}
\pmtype{Definition}
\pmcomment{trigger rebuild}
\pmclassification{msc}{68Q01}
%\pmkeywords{Continued Fraction}
%\pmkeywords{Euclidean Algorithm}

\endmetadata

% this is the default PlanetMath preamble.  as your knowledge
% of TeX increases, you will probably want to edit this, but
% it should be fine as is for beginners.

% almost certainly you want these
\usepackage{amssymb}
\usepackage{amsmath}
\usepackage{amsfonts}

% used for TeXing text within eps files
%\usepackage{psfrag}
% need this for including graphics (\includegraphics)
%\usepackage{graphicx}
% for neatly defining theorems and propositions
%\usepackage{amsthm}
% making logically defined graphics
%%%\usepackage{xypic}

% there are many more packages, add them here as you need them

% define commands here

\begin{document}
It is defined as follow
$$
K_n(x_1,x_2,\dots,x_n):=
\left\{
\begin{array}{ll}
 1, & {\rm if~} n=0;\\
 x_1, & {\rm if~} n=1;\\
 x_1 K_{n-1}(x_2,\dots,x_n) + K_{n-2}(x_3,\dots,x_n), & {\rm if~} n>1.\\
\end{array}
\right.
$$

It is easy to show, that
$$
K_n(x_1,x_2,\dots,x_n)=
\det
\left(
 \begin{array}{rrrrrr}
   x_n &  +1  &  0  & \dots &  0  &  0  \\
    -1  &x_{n-1}&  +1  & \dots &  0  &  0  \\
    0  &  -1  &  x_{n-2}  & \ddots &  0  &  0  \\

   \vdots & \vdots & \ddots & \ddots & \vdots & \vdots \\

    0  &  0  &  0  & \dots &  x_2  &  +1  \\
    0  &  0  &  0  & \dots &  -1  &  x_1  \\
 \end{array}
\right).
$$

References: Knuth D.E. {\bf "The Art of Computer Programming, Volume 2: Seminumerical Algorithms}.
%%%%%
%%%%%
\end{document}
