\documentclass[12pt]{article}
\usepackage{pmmeta}
\pmcanonicalname{ProofOfAmdahlsLaw}
\pmcreated{2013-03-22 13:10:36}
\pmmodified{2013-03-22 13:10:36}
\pmowner{lieven}{1075}
\pmmodifier{lieven}{1075}
\pmtitle{proof of Amdahl's Law}
\pmrecord{5}{33622}
\pmprivacy{1}
\pmauthor{lieven}{1075}
\pmtype{Proof}
\pmcomment{trigger rebuild}
\pmclassification{msc}{68M20}

\endmetadata

% this is the default PlanetMath preamble.  as your knowledge
% of TeX increases, you will probably want to edit this, but
% it should be fine as is for beginners.

% almost certainly you want these
\usepackage{amssymb}
\usepackage{amsmath}
\usepackage{amsfonts}

% used for TeXing text within eps files
%\usepackage{psfrag}
% need this for including graphics (\includegraphics)
%\usepackage{graphicx}
% for neatly defining theorems and propositions
%\usepackage{amsthm}
% making logically defined graphics
%%%\usepackage{xypic}

% there are many more packages, add them here as you need them

% define commands here
\begin{document}
Suppose an algorithm needs $n$ operations to compute the result. With 1 processor, the algorithm will take $n$ time units. With $N$ processors, the $(1-f)n$ parallelizable operations will take $\frac{(1-f)n}{N}$ time units and the remaining $fn$ non parallelizable operations will take $fn$ time units for a total running time of $fn+\frac{(1-f)n}{N}$ time units. So the speedup $S$ is $\frac{n}{fn+\frac{(1-f)n}{N}}=\frac{1}{f+\frac{1-f}{N}}$.
%%%%%
%%%%%
\end{document}
