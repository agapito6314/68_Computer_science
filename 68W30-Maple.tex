\documentclass[12pt]{article}
\usepackage{pmmeta}
\pmcanonicalname{Maple}
\pmcreated{2013-03-22 16:21:49}
\pmmodified{2013-03-22 16:21:49}
\pmowner{Mravinci}{12996}
\pmmodifier{Mravinci}{12996}
\pmtitle{Maple}
\pmrecord{6}{38500}
\pmprivacy{1}
\pmauthor{Mravinci}{12996}
\pmtype{Topic}
\pmcomment{trigger rebuild}
\pmclassification{msc}{68W30}

\endmetadata

% this is the default PlanetMath preamble.  as your knowledge
% of TeX increases, you will probably want to edit this, but
% it should be fine as is for beginners.

% almost certainly you want these
\usepackage{amssymb}
\usepackage{amsmath}
\usepackage{amsfonts}

% used for TeXing text within eps files
%\usepackage{psfrag}
% need this for including graphics (\includegraphics)
%\usepackage{graphicx}
% for neatly defining theorems and propositions
%\usepackage{amsthm}
% making logically defined graphics
%%%\usepackage{xypic}

% there are many more packages, add them here as you need them

% define commands here

\begin{document}
{\em Maple} is a computer algebra system produced by Maplesoft. The first version, developed by the University of Waterloo in Ontario, Canada, came out in 1981. The most recent version is 11.0. The program is named for a popular Canadian tree.

Like Mathematica, Maple has its own field in the On-Line Encyclopedia of Integer Sequences. 

\PMlinkexternal{Official website}{http://www.maplesoft.com/}
%%%%%
%%%%%
\end{document}
