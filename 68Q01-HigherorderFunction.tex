\documentclass[12pt]{article}
\usepackage{pmmeta}
\pmcanonicalname{HigherorderFunction}
\pmcreated{2013-03-22 12:33:38}
\pmmodified{2013-03-22 12:33:38}
\pmowner{Logan}{6}
\pmmodifier{Logan}{6}
\pmtitle{higher-order function}
\pmrecord{4}{32807}
\pmprivacy{1}
\pmauthor{Logan}{6}
\pmtype{Definition}
\pmcomment{trigger rebuild}
\pmclassification{msc}{68Q01}
\pmsynonym{functional}{HigherorderFunction}
\pmrelated{Currying}

\endmetadata

\usepackage{amssymb}
\usepackage{amsmath}
\usepackage{amsfonts}
\begin{document}
Any function that maps a function to anything or maps anything to a function is a \emph{higher-order function}.  In programming language terms, a higher-order function is any function that takes one or more functions as arguments and/or returns a function.

For example, a predicate which makes some statement about a function would be a higher-order function.
%%%%%
%%%%%
\end{document}
