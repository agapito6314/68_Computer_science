\documentclass[12pt]{article}
\usepackage{pmmeta}
\pmcanonicalname{ListOfOverloadedSymbols}
\pmcreated{2013-03-22 16:41:55}
\pmmodified{2013-03-22 16:41:55}
\pmowner{PrimeFan}{13766}
\pmmodifier{PrimeFan}{13766}
\pmtitle{list of overloaded symbols}
\pmrecord{11}{38912}
\pmprivacy{1}
\pmauthor{PrimeFan}{13766}
\pmtype{Example}
\pmcomment{trigger rebuild}
\pmclassification{msc}{68N15}
\pmclassification{msc}{00A99}

\endmetadata

% this is the default PlanetMath preamble.  as your knowledge
% of TeX increases, you will probably want to edit this, but
% it should be fine as is for beginners.

% almost certainly you want these
\usepackage{amssymb}
\usepackage{amsmath}
\usepackage{amsfonts}

% used for TeXing text within eps files
%\usepackage{psfrag}
% need this for including graphics (\includegraphics)
%\usepackage{graphicx}
% for neatly defining theorems and propositions
%\usepackage{amsthm}
% making logically defined graphics
%%%\usepackage{xypic}

% there are many more packages, add them here as you need them

% define commands here

\begin{document}
By tradition, certain symbols in mathematics are overloaded with multiple meanings, often resolvable by context.

\subsection{The Latin-1 alphabet}

$A$ Hexadecimal numeral 10. As $A(x)$ members of a sequence that are less than $x$ function

$a$ First variable. Real part of complex number. As $a(n)$ sequence function (this is a frequent use in Sloane's OEIS). As $a_n$, a recurrence relation, with $a_1$ and $a_2$ often assigned literal values.

$B$ Bernoulli number. Hexadecimal numeral 11.

$b$ Second variable. Multiplied by an imaginary unit (usually $i$), the imaginary part of a complex number.

$C$ Roman numeral 100. Catalan number. Hexadecimal numeral 12. A fallback in the absence of $\mathbb{C}$, the set of complex numbers.

$c$ Third variable. A constant, "not always the same!" (Finch, Guy). Complex number $c = a + bi$. In physics, the constant of the speed of light.

$D$ Roman numeral 500. Hexadecimal numeral 13.

$d$ Divisor. Digit. Difference.

$E$ Euler numbers. The base of natural logarithms in Mathematica's InputForm. Hexadecimal numeral 14.

$e$ The base of natural logarithms. Sylvester's sequence. In physics, a variable of energy.

$F$ Fermat number. Fibonacci number. Hexadecimal numeral 15.

% $f$ Function.

% $g$ Second function.

% $H$ Hypothesis.

$I$ Roman numeral 1. Imaginary unit in Mathematica's InputForm.

$i$ Iterator. Imaginary unit.

% $j$ Second iterator.

% $k$ Sequence length.

$L$ Roman numeral 50. Liouville sum function.

$M$ Mersenne number. Mertens function. Roman numeral 1000.

$m$ Arbitrary number related to $n$. In physics, a variable of matter.

% $N$ A fallback in the absence of $\mathbb{N}$, the set of natural numbers.

% $n$ Arbitrary number, usually an integer.

$P$ Largest prime factor function. Number in the Padovan sequence. Pell number. Perrin number.

% $p$ Prime number.

% $Q$

% $q$ Second prime.

% $R$ Repunit.

$r$ Third prime. Remainder.

% $S$ Set.

% $s$ Sum of divisors of $n$, not including $n$ itself.

% $T$ Second set.

% $u$ First variable.

% $V$ Roman numeral 5.

% $v$ Second variable.

% $W$ van der Waerden number

% $X$ Roman numeral 10.

$x$ First variable. Horizontal axis label. Arbitrary real number.

% $Y$

$y$ Second variable. Vertical axis label.

% $Z$ A fallback in the absence of $\mathbb{Z}$, the set of integers.

% $z$ Depth axis label.

\subsection{ASCII symbols}

$\ast$ Convolution operator. Multiplication operator in most computer programming languages.

$\%$ Percentage symbol. Reference to previous output in Mathematica (probably a hidden reference to the musical bis repeat sign). Modulo operator in Basic, C, Java, Javascript, Perl, etc.

\subsection{The Greek alphabet}

If it seems that the entire Greek alphabet has been overloaded, this is actually not the case. Several uses have been proposed for the more popular letters, but there are other letters with few or no assigned uses. This is also the case of Greek letter that look almost the same as Latin-1 letters, so understandably these are not popular.

% $\Alpha$

% $\alpha$

% $\Beta$

% $\beta$

% $\Gamma$ Finite tape alphabet for a Turing machine.

% $\gamma$ Euler-Mascheroni constant

% $\Delta$

$\delta$ As $\delta_S$, the silver ratio. As $\delta_{ij}$, or $\delta^{ij}$ or $\delta^i_j$, the Kronecker delta. As $\delta(x)$, the Dirac delta function. Transition function for a Turing machine.

% $\Epsilon$

$\epsilon$ In analysis, an arbitrarily small value. The trivial group representation, or sometimes the identity element in a group.

% $\Zeta$

% $\zeta$

% $\Eta$

% $\eta$

% $\Theta$

% $\theta$

% $\Iota$

% $\iota$

% $\Kappa$

% $\kappa$

% $\Lambda$

$\lambda$ Carmichael function (sometimes $\psi$ is used instead), Liouville function.

% $\Mu$

% $\mu$ M\"obius function.

% $\Nu$

% $\nu$

% $\Xi$

% $\xi$

% $\Omicron$

% $\omicron$

$\Pi$ Iterated product. A set of symbols usable by a Turing machine.

$\pi$ Ratio of a circle's radius to its circumference. Prime counting function. Totient function in Euler's notebooks.

% $\Rho$

% $\rho$

$\Sigma$ Iterated sum. A set of symbols usable by a Turing machine.

% $\sigma$ Sum of divisors function.

% $\Tau$

% $\tau$

% $\Upsilon$

% $\upsilon$

$\phi$ Totient function. Golden ratio.

% $\Phi$

% $\Chi$

% $\chi$

% $\Psi$

% $\psi$ Carmichael function (usually $\lambda$ is used instead)

% $\Omega$ Number of nondistinct prime factors function.

% $\omega$ Number of distinct prime factors function.
%%%%%
%%%%%
\end{document}
