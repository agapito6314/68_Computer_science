\documentclass[12pt]{article}
\usepackage{pmmeta}
\pmcanonicalname{Currying}
\pmcreated{2013-03-22 12:33:35}
\pmmodified{2013-03-22 12:33:35}
\pmowner{mps}{409}
\pmmodifier{mps}{409}
\pmtitle{currying}
\pmrecord{8}{32806}
\pmprivacy{1}
\pmauthor{mps}{409}
\pmtype{Definition}
\pmcomment{trigger rebuild}
\pmclassification{msc}{68Q01}
\pmsynonym{sch\"onfinkeling}{Currying}
\pmsynonym{sch\"onfinkelization}{Currying}
\pmrelated{HigherOrderFunction}
\pmdefines{curried function}
\pmdefines{uncurried function}

\endmetadata

% this is the default PlanetMath preamble.  as your knowledge
% of TeX increases, you will probably want to edit this, but
% it should be fine as is for beginners.

% almost certainly you want these
\usepackage{amssymb}
\usepackage{amsmath}
\usepackage{amsfonts}

% used for TeXing text within eps files
%\usepackage{psfrag}
% need this for including graphics (\includegraphics)
%\usepackage{graphicx}
% for neatly defining theorems and propositions
%\usepackage{amsthm}
% making logically defined graphics
%%%\usepackage{xypic}

% there are many more packages, add them here as you need them

% define commands here
\DeclareMathOperator{\Hom}{Hom}
\newcommand{\Set}{\mathbf{Set}}
\begin{document}
\emph{Currying} is the technique of emulating multiple-parametered
functions with higher-order functions.  The notion is that a function
of $n$ arguments can be thought of as a function of 1 argument that
maps to a function of $n-1$ arguments.  A \emph{curried function} is a
function represented by currying, e.g.

\[
f \colon A \to (B\to C)
\]

For conciseness, the mapping operator $\rightarrow$ is usually
considered right-associative, so one could drop the parentheses in the
expression above and write $f \colon A\to B\to C$ instead.

In contrast, an \emph{uncurried function} is usually specified as a
mapping from a Cartesian product, such as

\[
f \colon (A\times B)\to C.
\]

The term \emph{currying} is derived from the name of Haskell Curry, a
20th-century logician.  However, Curry was not the first person to
discover this notion, as it was first introduced by Gottlob Frege in
1893 and expanded by Moses Sch\"onfinkel in the 1920s.  Hence the
notion is sometimes referred to as \emph{sch\"onfinkeling}.

% Note: the term fregenation never caught on.

From the perspective of category theory, currying can be thought of as
exploiting the fact that $-\times B$ and $\Hom(B,-)$ are adjoint
functors on $\Set$.  That is, for each set $B$, there is a natural
equivalence
\[
   \nu\colon \Hom_{\Set}(-\times B,-)\overset{\cdot}{\longrightarrow}\Hom_{\Set}(-,\Hom(B,-))
\]
defined by sending a map $f\colon (A\times B)\to C$ to the map
$\nu_f\colon A\to \Hom(B, C)$.  For each $a\in A$, $\nu_f(a)\colon
B\to C$ is the map defined by $\nu_f(a)(b) = f(a,b)$.

\PMlinkescapeword{term}
%%%%%
%%%%%
\end{document}
