\documentclass[12pt]{article}
\usepackage{pmmeta}
\pmcanonicalname{NotANumber}
\pmcreated{2013-03-22 18:14:09}
\pmmodified{2013-03-22 18:14:09}
\pmowner{PrimeFan}{13766}
\pmmodifier{PrimeFan}{13766}
\pmtitle{not a number}
\pmrecord{4}{40824}
\pmprivacy{1}
\pmauthor{PrimeFan}{13766}
\pmtype{Definition}
\pmcomment{trigger rebuild}
\pmclassification{msc}{68P01}

% this is the default PlanetMath preamble.  as your knowledge
% of TeX increases, you will probably want to edit this, but
% it should be fine as is for beginners.

% almost certainly you want these
\usepackage{amssymb}
\usepackage{amsmath}
\usepackage{amsfonts}

% used for TeXing text within eps files
%\usepackage{psfrag}
% need this for including graphics (\includegraphics)
%\usepackage{graphicx}
% for neatly defining theorems and propositions
%\usepackage{amsthm}
% making logically defined graphics
%%%\usepackage{xypic}

% there are many more packages, add them here as you need them

% define commands here

\begin{document}
In a computer, {\em not a number} (usually abbreviated {\em NaN}) indicates that an arithmetical operation has resulted in a value that is not a number understood (or representable) by the computer. Division by zero is one frequent cause of a result being NaN. Asking for the square root of a negative number is another operation that can result in NaN (unless the computer has been programmed to recognize imaginary numbers).

NaN is not a specific value that can be tested for with the usual comparison operators. The programmer is advised to test for NaN with a special function provided by the programming language (such as \verb=isNaN(x)= in JavaScript.

Generally, computer algebra systems do not report NaN results even if they do display an error message for a given operation.

\begin{thebibliography}{1}
\bibitem{df} David Flanagan, {\it JavaScript the Definitive Guide: The Definitive Guide}. New York: O'Reilly (2006): 24
\end{thebibliography}
%%%%%
%%%%%
\end{document}
