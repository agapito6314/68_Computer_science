\documentclass[12pt]{article}
\usepackage{pmmeta}
\pmcanonicalname{Overload}
\pmcreated{2013-03-22 16:03:04}
\pmmodified{2013-03-22 16:03:04}
\pmowner{Mravinci}{12996}
\pmmodifier{Mravinci}{12996}
\pmtitle{overload}
\pmrecord{6}{38102}
\pmprivacy{1}
\pmauthor{Mravinci}{12996}
\pmtype{Definition}
\pmcomment{trigger rebuild}
\pmclassification{msc}{68N15}
\pmclassification{msc}{00A99}
\pmsynonym{overloaded}{Overload}

% this is the default PlanetMath preamble.  as your knowledge
% of TeX increases, you will probably want to edit this, but
% it should be fine as is for beginners.

% almost certainly you want these
\usepackage{amssymb}
\usepackage{amsmath}
\usepackage{amsfonts}

% used for TeXing text within eps files
%\usepackage{psfrag}
% need this for including graphics (\includegraphics)
%\usepackage{graphicx}
% for neatly defining theorems and propositions
%\usepackage{amsthm}
% making logically defined graphics
%%%\usepackage{xypic}

% there are many more packages, add them here as you need them

% define commands here

\begin{document}
To assign another meaning to a symbol that already has a meaning, or a new operation to an operator that is already assigned to another operation. Perhaps the quintessential example of overloading in mathematics is the case of the Greek letter $\pi$. In geometry $\pi$ refers to the ratio between the perimeter and the diameter on a circle, while in number theory $\pi(x)$ refers to the prime counting function, and not the multiplication of $x$ by the circle perimeter/diameter ratio.

In some cases it is possible to resolve meaning purely from context. For example, if $i$ occurs under a $\Sigma$ or a $\Pi$ it is most likely just a generic iterator. Absent those Greek letters, it could be the imaginary unit, $\sqrt{-1}$.

The + operator is often overloaded to be both the addition of numbers operator and the string concatenation operator.

In computer programming languages, overloading is one of the kinds of ad-hoc polymorphism, and it is argued that this helps reduce the kind of notational clutter that might ensue if every operation had to have an operator specifically designed for its type.
%%%%%
%%%%%
\end{document}
