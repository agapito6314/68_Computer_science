\documentclass[12pt]{article}
\usepackage{pmmeta}
\pmcanonicalname{PrimitiveRecursiveEncoding}
\pmcreated{2013-03-22 19:06:20}
\pmmodified{2013-03-22 19:06:20}
\pmowner{CWoo}{3771}
\pmmodifier{CWoo}{3771}
\pmtitle{primitive recursive encoding}
\pmrecord{7}{41998}
\pmprivacy{1}
\pmauthor{CWoo}{3771}
\pmtype{Definition}
\pmcomment{trigger rebuild}
\pmclassification{msc}{68Q05}
\pmclassification{msc}{68Q45}
\pmclassification{msc}{03D20}
\pmclassification{msc}{94A60}
\pmdefines{sequence number}

\endmetadata

\usepackage{amssymb,amscd}
\usepackage{amsmath}
\usepackage{amsfonts}
\usepackage{mathrsfs}

% used for TeXing text within eps files
%\usepackage{psfrag}
% need this for including graphics (\includegraphics)
%\usepackage{graphicx}
% for neatly defining theorems and propositions
\usepackage{amsthm}
% making logically defined graphics
%%\usepackage{xypic}
\usepackage{pst-plot}

% define commands here
\newcommand*{\abs}[1]{\left\lvert #1\right\rvert}
\newtheorem{prop}{Proposition}
\newtheorem{thm}{Theorem}
\newtheorem{ex}{Example}
\newcommand{\real}{\mathbb{R}}
\newcommand{\pdiff}[2]{\frac{\partial #1}{\partial #2}}
\newcommand{\mpdiff}[3]{\frac{\partial^#1 #2}{\partial #3^#1}}
\begin{document}
Recall that an encoding of a set $L$ of words over some alphabet $\Sigma$ is defined as an injective function $E: L\to \mathbb{N}$, the set of natural numbers (including $0$ here).

Finite sequences over $\mathbb{N}$ can be thought of as words over the ``infinite'' alphabet $\mathbb{N}$.  So the notion of word encoding directly carries over to encoding of finite sequences of natural numbers.  Let $\mathbb{N}^*$ be the set of all finite sequences over $\mathbb{N}$.

\textbf{Definition}.  Let $E$ be an encoding for $\mathbb{N}^*$.  A number is called a \emph{sequence number} if it is in the range of $E$.  Since $E$ is injective, we say that $E(a)$ is \emph{the} sequence number of the sequence $a$.  

Once $E$ is fixed, we also use the notation $\langle a_1, \ldots, a_k \rangle$ to mean the sequence number of the sequence $a_1,\ldots, a_k$.

Define the following operations on $\mathbb{N}$ associated with a given $E$:
\begin{enumerate}
\item $E_n := E | \mathbb{N}^*_n$, the restriction of $E$ to the set $\mathbb{N}^*_n$ of all sequences of length $n$, and $E_0$ is defined as the number $\langle \rangle$.
\item the \emph{length function}: $\operatorname{lh}: \mathbb{N} \to \mathbb{N}$:
\begin{displaymath}
\operatorname{lh}(x) := \left\{
\begin{array}{ll}
z , & \textrm{if }E^{-1}(x)\textrm{ exists, and has length }z, \\
0 , & \textrm{otherwise}.
\end{array}
\right.
\end{displaymath}
\item $(\cdot)_{\cdot}: \mathbb{N}^2\to \mathbb{N}$, such that 
\begin{displaymath}
(x)_y := \left\{
\begin{array}{ll}
z , & \textrm{if }E^{-1}(x)\textrm{ exists, has length }\ge y\textrm{, with }z\textrm{ its }y\textrm{-th number}, \\
0 , & \textrm{otherwise}.
\end{array}
\right.
\end{displaymath}
\item $*: \mathbb{N}^2 \to \mathbb{N}$, such that 
\begin{displaymath}
x * y := \left\{
\begin{array}{ll}
E(ab) , & \textrm{where }E(a)=x \textrm{ and }E(b)=y, \\
0 , & \textrm{otherwise}.
\end{array}
\right.
\end{displaymath}
The notation $ab$ stands for the concatenation of the sequences $a$ and $b$.
\item $\operatorname{ext} : \mathbb{N}^2 \to \mathbb{N}$, such that 
\begin{displaymath}
\operatorname{ext}(x,y) := \left\{
\begin{array}{ll}
E(ay) , & \textrm{where }E(a)=x , \\
0 , & \textrm{otherwise}.
\end{array}
\right.
\end{displaymath}
The notation $ay$ stands for the sequence $a$, extended by appending $y$ to the right of $a$.
\item $\operatorname{red} : \mathbb{N} \to \mathbb{N}$, such that 
\begin{displaymath}
\operatorname{red}(x) := \left\{
\begin{array}{ll}
z , & \textrm{where }E(a)=x \textrm{ and }E(b)=z \textrm{ such that }a=bc\textrm{ and }c\in \mathbb{N}, \\
0 , & \textrm{otherwise}.
\end{array}
\right.
\end{displaymath}
In other words, $z$ is the sequence number of the sequence obtained by removing the last (rightmost) number (if any) in the sequence whose sequence number is $x$, provided that $x$ is a sequence number.
\end{enumerate}

\textbf{Definition}.  An encoding $E$ for $\mathbb{N}^*$ is said to be \emph{primitive recursive} if 
\begin{itemize}
\item $E(\mathbb{N}^*)$ is a primitive recursive set, and
\item the first three types of functions defined above are primitive recursive.
\end{itemize}

\begin{prop} If $E$ is primitive recursive, so are the functions $*,\operatorname{ext}$, and $\operatorname{red}$. \end{prop}
\begin{proof}  Let $\chi(x)$ be the characteristic function of the predicate ``$x$ is a sequence number'' (via $E$).  It is primitive recursive by assumption.
\begin{enumerate}
\item Let $n= \operatorname{lh}(x)+\operatorname{lh}(y)$.  Then $x* y = E_n((x)_1, \ldots, (x)_{\operatorname{lh}(x)}, (y)_1, \ldots, (y)_{\operatorname{lh}(y)})\cdot \chi(x)\chi(y)$, which is primitive recursive.
\item Let $n= \operatorname{lh}(x)+1$.  Then $\operatorname{ext}(x,y) = E_n((x)_1, \ldots, (x)_{\operatorname{lh}(x)}, y)\chi(x)$, which is primitive recursive.
\item Let $n= \operatorname{lh}(x)\dot{-}1$.  Then 
\begin{displaymath}
\operatorname{red}(x) = \left\{
\begin{array}{ll}
E_n((x)_1, \ldots, (x)_{\operatorname{lh}(x)\dot{-} 1}) , & \textrm{if }\operatorname{lh}(x)>1\textrm{, and }\chi(x)=1 \\
\langle \rangle, & \textrm{if }\operatorname{lh}(x)=1\textrm{, and }\chi(x)=1 \\
0 , & \textrm{otherwise},
\end{array}
\right.
\end{displaymath}
which is primitive recursive.
\end{enumerate}
\end{proof}

Examples of primitive recursive encoding can be found in the parent entry (methods 2, 3, and 7).
%%%%%
%%%%%
\end{document}
