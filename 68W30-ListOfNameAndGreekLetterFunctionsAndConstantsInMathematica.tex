\documentclass[12pt]{article}
\usepackage{pmmeta}
\pmcanonicalname{ListOfNameAndGreekLetterFunctionsAndConstantsInMathematica}
\pmcreated{2013-03-22 17:13:55}
\pmmodified{2013-03-22 17:13:55}
\pmowner{PrimeFan}{13766}
\pmmodifier{PrimeFan}{13766}
\pmtitle{list of name and Greek letter functions and constants in Mathematica}
\pmrecord{4}{39560}
\pmprivacy{1}
\pmauthor{PrimeFan}{13766}
\pmtype{Data Structure}
\pmcomment{trigger rebuild}
\pmclassification{msc}{68W30}

% this is the default PlanetMath preamble.  as your knowledge
% of TeX increases, you will probably want to edit this, but
% it should be fine as is for beginners.

% almost certainly you want these
\usepackage{amssymb}
\usepackage{amsmath}
\usepackage{amsfonts}

% used for TeXing text within eps files
%\usepackage{psfrag}
% need this for including graphics (\includegraphics)
%\usepackage{graphicx}
% for neatly defining theorems and propositions
%\usepackage{amsthm}
% making logically defined graphics
%%%\usepackage{xypic}

% there are many more packages, add them here as you need them

% define commands here

\begin{document}
In Mathematica, certain important mathematical functions or constants are built-in and given the name of the mathematician who came up with the function followed by the romanization of the Greek letter most commonly used for the function.

\begin{tabular}{|l|l|}
\verb=EulerGamma= & The Euler-Mascheroni constant $\gamma$ \\
\verb=StieltjesGamma[n]= & Stieltjes constant $\gamma n$ \\
\verb=CarmichaelLambda[n]= & Carmichael function $\lambda(n)$ \\
\verb=MoebiusMu[n]= & M\"obius function $\mu(n)$ \\
\verb=LerchPhi[z, s, a]= & Lerch function $\Phi(z, s, a)$ \\
\verb=EulerPhi[n]= & Euler totient function $\phi(n)$ \\
\verb=RamanujanTau[n]= & Ramanujan function $\tau(n)$ \\
\verb=RamanujanTauL[x]= & Ramanujan $\tau$ Dirichlet L function $L(x)$ \\
\end{tabular}

Note that the Riemann zeta function $\zeta(x)$ is invoked by \verb=Zeta[x]= and not \verb=RiemannZeta[x]=.

%%%%%
%%%%%
\end{document}
