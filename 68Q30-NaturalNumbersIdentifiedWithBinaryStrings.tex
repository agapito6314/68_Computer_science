\documentclass[12pt]{article}
\usepackage{pmmeta}
\pmcanonicalname{NaturalNumbersIdentifiedWithBinaryStrings}
\pmcreated{2013-03-22 13:43:44}
\pmmodified{2013-03-22 13:43:44}
\pmowner{tromp}{1913}
\pmmodifier{tromp}{1913}
\pmtitle{natural numbers identified with binary strings}
\pmrecord{7}{34415}
\pmprivacy{1}
\pmauthor{tromp}{1913}
\pmtype{Definition}
\pmcomment{trigger rebuild}
\pmclassification{msc}{68Q30}

\endmetadata

% this is the default PlanetMath preamble.  as your knowledge
% of TeX increases, you will probably want to edit this, but
% it should be fine as is for beginners.

% almost certainly you want these
\usepackage{amssymb}
\usepackage{amsmath}
\usepackage{amsfonts}

% used for TeXing text within eps files
%\usepackage{psfrag}
% need this for including graphics (\includegraphics)
%\usepackage{graphicx}
% for neatly defining theorems and propositions
%\usepackage{amsthm}
% making logically defined graphics
%%%\usepackage{xypic}

% there are many more packages, add them here as you need them

% define commands here
\begin{document}
It is convenient to identify a natural number $n$ with the $n$th binary
string in lexicographic order:
$$
\begin{array}{ll}
0 & \epsilon \\
1 & 0 \\
2 & 1 \\
3 & 00 \\
4 & 01 \\
5 & 10 \\
6 & 11 \\
7 & 000 \\
\ldots & \ldots
\end{array}
$$
The more common {\em binary notation} for numbers fails to be a bijection
because of leading zeroes. Yet, there is a close relation: the $n$th
binary string is the result of stripping the leading 1 from the binary
notation of $n+1$.


With this correspondence in place, we can talk about such things as
the length $l(n)$ of a number $n$, which can be seen to equal
$\lfloor \log (n+1) \rfloor$.
%%%%%
%%%%%
\end{document}
