\documentclass[12pt]{article}
\usepackage{pmmeta}
\pmcanonicalname{Substitution}
\pmcreated{2013-03-22 18:55:12}
\pmmodified{2013-03-22 18:55:12}
\pmowner{CWoo}{3771}
\pmmodifier{CWoo}{3771}
\pmtitle{substitution}
\pmrecord{20}{41772}
\pmprivacy{1}
\pmauthor{CWoo}{3771}
\pmtype{Definition}
\pmcomment{trigger rebuild}
\pmclassification{msc}{68Q45}
\pmsynonym{string substitution}{Substitution}
\pmrelated{HomomorphismOfLanguages}
\pmrelated{SubstitutionsInLogic}
\pmdefines{$\lambda$-free substitution}
\pmdefines{regular substitution}

\endmetadata

\usepackage{amssymb,amscd}
\usepackage{amsmath}
\usepackage{amsfonts}
\usepackage{mathrsfs}

% used for TeXing text within eps files
%\usepackage{psfrag}
% need this for including graphics (\includegraphics)
%\usepackage{graphicx}
% for neatly defining theorems and propositions
\usepackage{amsthm}
% making logically defined graphics
%%\usepackage{xypic}
\usepackage{pst-plot}

% define commands here
\newcommand*{\abs}[1]{\left\lvert #1\right\rvert}
\newtheorem{prop}{Proposition}
\newtheorem{thm}{Theorem}
\newtheorem{ex}{Example}
\newcommand{\real}{\mathbb{R}}
\newcommand{\pdiff}[2]{\frac{\partial #1}{\partial #2}}
\newcommand{\mpdiff}[3]{\frac{\partial^#1 #2}{\partial #3^#1}}

\begin{document}
\subsubsection*{Definition}

Let $\Sigma_1, \Sigma_2$ be alphabets.  A \emph{substitution}, or \emph{string substitution}, is a function $s:\Sigma_1^* \to P(\Sigma_2^*)$ such that
\begin{itemize}
\item $s$ preserves the empty word: $s(\lambda)=\lbrace \lambda \rbrace$, and 
\item $s$ preserves concatenation: $s(\alpha\beta)=s(\alpha)s(\beta)$.
\end{itemize}
In other words, for every word $\alpha$ over $\Sigma_1$, $s(\alpha)$ is a language over $\Sigma_2$.  In the second condition above, $s(\alpha)s(\beta)$ is the concatenation of languages: $\lbrace uv\mid u\in s(\alpha), v\in s(\beta)\rbrace$.

For example, suppose $\Sigma=\lbrace a,b \rbrace$.  The map $s$ taking $u$ to $\lbrace u' \rbrace$, where $u'$ is obtained from $u$ by replacing every occurrence of $a$ by $b$ is a substitution.

One easy way to obtain more examples of substitutions is to start with some function $$f:\Sigma_1 \to P(\Sigma_2^*),$$ and extend it to all of $\Sigma_1^*$ by language concatenation: if $u=a_1\cdots a_n$, with $a_i\in \Sigma_1$, defining $$s(u):= f(a_1)\cdots f(a_n)$$
gives us a substitution $s$.  It is easy to see that the extension is unique (if $s_1$ and $s_2$ both extend $f$, then $s_1=s_2$).

In fact, every substitution is obtained this way: every substitution $s:\Sigma_1^* \to P(\Sigma_2^*)$ is the extension of its restriction to $\Sigma_1$.  This can be verified directly, but is the result of a more general fact: any function $f:A\to B$, where $B$ is a semigroup, extends uniquely to a semigroup homomorphism $f^*:A^* \to B$ where $A^*$ is the semigroup freely generated by $A$.

In the previous example, $s$ is the extension of the function that takes $a$ to $\lbrace b\rbrace$ and $b$ to $\lbrace b\rbrace$.

\subsubsection*{Closure under Substitution}

For any language $L\subseteq \Sigma_1^*$ and a substitution $s:\Sigma_1^* \to P(\Sigma_2^*)$, define $$s(L):=\bigcup \lbrace s(u)\mid u\in L\rbrace.$$

A family $\mathscr{F}$ of languages is said to be \emph{closed under substitutions} if, given any substitution $s$, with $L \in \mathscr{F}$ and $s(w) \in \mathscr{F}$ for each $w\in L$, we have $s(L)\in \mathscr{F}$.  The following families  are closed under substitutions:
\begin{itemize}
\item regular languages, 
\item context-free languages, and 
\item type-0 langauges.
\end{itemize}
As a corollary, the families of regular, context-free, and type-0 languages are closed under homomorphisms, since every homomorphism of languages is really just a special case of substitution, such that every symbol of the domain alphabet is mapped to a singleton consisting of a word over the range alphabet.

The family of context-sensitive languages is not closed under general substitutions.  Instead, it is closed under  $\lambda$-free substitutions (see remark below).

\textbf{Remarks}.  
\begin{itemize}
\item
The notion of string substitution generally corresponds to our intuitive notion of how a substitution should behave: 
\begin{quote}
given words $u,v,w$, then $\operatorname{Substitute}(u,v,w)$ is a word that is obtained from $u$ by replacing every occurrence of $v$ in $u$ by $w$.  
\end{quote}
However, this is not always the case.  For example, let $\Sigma=\lbrace a,b\rbrace$, and $s$ be the map that takes $u$ to $\lbrace u'\rbrace$, where $u'$ is obtained from $u$ by replacing all occurrences of $aa$, if any, by $b$.  Then it is easy to see that $s$ is not a substitution, for
$$s(a)s(a)=\lbrace a\rbrace \lbrace a\rbrace = \lbrace aa\rbrace$$
while
$$s(aa)=\lbrace b\rbrace \ne s(a)s(a).$$
Nevertheless, $s$ is ``intuitively'' a ``substitution''.
\item
A substitution $s$ is said to have property $\mathcal{P}$ if for each $a\in \Sigma$, the set $s(a)$ has property $\mathcal{P}$.  Thus, for example, a substitution $s$ is finite if $s(a)$ is a finite set, regular if $s(a)$ is a regular language, and $\lambda$-free if each $s(a)$ is $\lambda$-free, etc...
\end{itemize}

\begin{thebibliography}{9}
\bibitem{sg} S. Ginsburg, {\em The Mathematical Theory of Context-Free Languages}, McGraw-Hill, New York (1966).
\bibitem{dk} D. C. Kozen, {\em Automata and Computability}, Springer, New York (1997).
\end{thebibliography}

%%%%%
%%%%%
\end{document}
