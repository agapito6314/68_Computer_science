\documentclass[12pt]{article}
\usepackage{pmmeta}
\pmcanonicalname{DerivationTree}
\pmcreated{2013-03-22 19:00:17}
\pmmodified{2013-03-22 19:00:17}
\pmowner{CWoo}{3771}
\pmmodifier{CWoo}{3771}
\pmtitle{derivation tree}
\pmrecord{10}{41873}
\pmprivacy{1}
\pmauthor{CWoo}{3771}
\pmtype{Definition}
\pmcomment{trigger rebuild}
\pmclassification{msc}{68Q42}
\pmclassification{msc}{68Q45}
\pmsynonym{generation tree}{DerivationTree}
\pmdefines{parse tree}
\pmdefines{result of a derivation tree}

\endmetadata

\usepackage{amssymb,amscd}
\usepackage{amsmath}
\usepackage{amsfonts}
\usepackage{mathrsfs}

% used for TeXing text within eps files
%\usepackage{psfrag}
% need this for including graphics (\includegraphics)
\usepackage{graphicx}
% for neatly defining theorems and propositions
\usepackage{amsthm}
% making logically defined graphics
%%\usepackage{xypic}
\usepackage{pst-plot}

% define commands here
\newcommand*{\abs}[1]{\left\lvert #1\right\rvert}
\newtheorem{prop}{Proposition}
\newtheorem{thm}{Theorem}
\newtheorem{ex}{Example}
\newcommand{\real}{\mathbb{R}}
\newcommand{\pdiff}[2]{\frac{\partial #1}{\partial #2}}
\newcommand{\mpdiff}[3]{\frac{\partial^#1 #2}{\partial #3^#1}}
\begin{document}
Given a formal grammar $G=(\Sigma,N,P,\sigma)$, recall that a derivation from words $u$ to $v$ over $\Sigma$ can be visualized as a finite sequence of words over $\Sigma$, connected by the binary relation $\Rightarrow$:
\begin{equation}
u_0 \Rightarrow u_1 \Rightarrow \cdots \Rightarrow u_n
\end{equation}
where $u=u_0$ and $v=u_n$.  Each $u_i \Rightarrow u_{i+1}$ is a derivation step, which means that there is a production in $P$ which, when applied to $u_i$, yields $u_{i+1}$.  In other words, there is $A\to B$ in $P$ such that $u_i=xAy$ and $u_{i+1}=xBy$, where $x,y$ are words over $\Sigma$.

When the formal grammar $G$ is context-free, a derivation can be represented by an ordered tree, revealing the structure behind the derivation that is usually not apparent in the linear representation $(1)$ above.  This ordered tree is variously known as a \emph{derivation tree} or a \emph{parse tree}, depending how it is being used.

In the foregoing discussion, $G$ is context-free, and any derivation of $G$ begins with $\sigma$, the starting non-terminal.

\textbf{Definition}.  A \emph{parse tree} of $G$ is an ordered tree $T$ such that
\begin{enumerate}
\item the nodes of $T$ are labeled by elements of $\Sigma$, or the empty word $\lambda$,
\item if a node with label $A$ has children $N_1, \ldots, N_k$ such that $N_1 < \cdots < N_k$, and that each $N_i$ has label $A_i$, then $A \to A_1 \cdots A_k$ is a production of $P$.
\end{enumerate}

A parse tree such that the root has label $\sigma$ is called a \emph{derivation tree}, or a \emph{generation tree}.  Every subtree of a derivation tree is a parse tree.

\textbf{Remark}.  Since $G$ is context-free, in a parse tree, any node that is not a leaf is labeled by a non-terminal symbol.

For example, if $\Sigma = \lbrace \sigma, a, b, X, Y\rbrace$, $N=\lbrace \sigma, X, Y\rbrace$, and the productions of $P$ are $$\sigma \to aXY, \quad X\to bYb, \quad Y\to Xa, \quad Y\to a,$$ then 

\begin{figure}[!htp]
\centering
\includegraphics[scale=1]{dtree.eps}
\end{figure}

represents a derivation tree of $G$.  The tree represents the following derivations 
\begin{itemize}
\item $\sigma \Rightarrow aXY \Rightarrow abYbY \Rightarrow abXabY \Rightarrow abXabb$
\item $\sigma \Rightarrow aXY \Rightarrow abYbY \Rightarrow abYbb \Rightarrow abXabb$
\item $\sigma \Rightarrow aXY \Rightarrow aXb \Rightarrow abYbb \Rightarrow abXabb$
\end{itemize}

\textbf{Definition}.  If $\ell_1, \ldots, \ell_m$ are the leaves of a parse tree $T$, with $\ell_1 < \cdots < \ell_m$, then the \emph{result} of $T$ is the word $B_1\cdots B_m$, where $B_i\in \Sigma$ is the label of $\ell_i$.  A word over $\Sigma$ is said to \emph{correspond} to a parse tree if it is the result of the tree.

In the example above, the result of the tree is $abXabb$.

\textbf{Remark}.  A derivable word may correspond to several derivation trees.  See the entry ambiguous grammar for more detail.

\begin{thebibliography}{9}
\bibitem{hlcp} H.R. Lewis, C.H. Papadimitriou, {\em Elements of the Theory of Computation}. Prentice-Hall, Englewood Cliffs, New Jersey (1981).
\bibitem{AS} A. Salomaa, {\em Formal Languages}, Academic Press, New York (1973).
\bibitem{hu} J.E. Hopcroft, J.D. Ullman, {\em Formal Languages and Their Relation to Automata}, Addison-Wesley, (1969).
\end{thebibliography}
%%%%%
%%%%%
\end{document}
