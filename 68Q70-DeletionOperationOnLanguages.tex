\documentclass[12pt]{article}
\usepackage{pmmeta}
\pmcanonicalname{DeletionOperationOnLanguages}
\pmcreated{2013-03-22 18:56:23}
\pmmodified{2013-03-22 18:56:23}
\pmowner{CWoo}{3771}
\pmmodifier{CWoo}{3771}
\pmtitle{deletion operation on languages}
\pmrecord{8}{41794}
\pmprivacy{1}
\pmauthor{CWoo}{3771}
\pmtype{Definition}
\pmcomment{trigger rebuild}
\pmclassification{msc}{68Q70}
\pmclassification{msc}{68Q45}
\pmsynonym{deletion-closed}{DeletionOperationOnLanguages}
\pmsynonym{deletion-closure}{DeletionOperationOnLanguages}
\pmrelated{ShuffleOfLanguages}
\pmrelated{InsertionOperationOnLanguages}
\pmdefines{deletion}
\pmdefines{deletion closed}
\pmdefines{deletion closure}

\endmetadata

\usepackage{amssymb,amscd}
\usepackage{amsmath}
\usepackage{amsfonts}
\usepackage{mathrsfs}

% used for TeXing text within eps files
%\usepackage{psfrag}
% need this for including graphics (\includegraphics)
%\usepackage{graphicx}
% for neatly defining theorems and propositions
\usepackage{amsthm}
% making logically defined graphics
%%\usepackage{xypic}
\usepackage{pst-plot}

% define commands here
\newcommand*{\abs}[1]{\left\lvert #1\right\rvert}
\newtheorem{prop}{Proposition}
\newtheorem{thm}{Theorem}
\newtheorem{ex}{Example}
\newcommand{\real}{\mathbb{R}}
\newcommand{\pdiff}[2]{\frac{\partial #1}{\partial #2}}
\newcommand{\mpdiff}[3]{\frac{\partial^#1 #2}{\partial #3^#1}}
\begin{document}
Let $\Sigma$ be an alphabet and $u,v$ be words over $\Sigma$.  A \emph{deletion} of $v$ from $u$ is a word of the form $u_1u_2$, where $u=u_1vu_2$.  If $w$ is a deletion of $v$ from $u$, then $u$ is an insertion of $v$ into $w$.

\emph{The deletion} of $v$ from $u$ is the set of all deletions of $v$ from $u$, and is denoted by $u \longrightarrow v$.

For example, if $u=(ab)^6$ and $v=aba$, then $$u\longrightarrow v = \lbrace (ba)^4b, ab(ba)^3b, (ab)^2(ba)^2b, (ab)^3bab, (ab)^4 b\rbrace.$$

More generally, given two languages $L_1,L_2$ over $\Sigma$, the \emph{deletion} of $L_2$ from $L_1$ is the set $$L_1 \longrightarrow L_2: = \bigcup \lbrace u\longrightarrow v\mid u \in L_1, v\in L_2 \rbrace.$$

For example, if $L_1=\lbrace (ab)^n \mid n \ge 0\rbrace$ and $L_2 = \lbrace a^n b^n \mid n \ge 0\rbrace$, then $L_1 \longrightarrow L_2 = L_1$, and $L_2 \longrightarrow L_1 = L_2$.  If $L_3 = \lbrace a^n \mid n\ge 0\rbrace$, then $L_2 \longrightarrow L_3 = a^m b^n \mid n\ge m \ge 0 \rbrace$, and $L_3 \longrightarrow L_2 = L_3$.

A language $L$ is said to be \emph{deletion-closed} if $L \longrightarrow L \subseteq L$.  $L_1,L_2$, and $L_3$ from above are all deletion closed, as well as the most obvious example: $\Sigma^*$.  $L_2\longrightarrow L_3$ is not deletion closed, for $a^3b^4 \longrightarrow ab^3 = \lbrace a^2b \rbrace \nsubseteq L_2\longrightarrow L_3$.

It is easy to see that arbitrary intersections of deletion-closed languages is deletion-closed.

Given a language $L$, the intersection of all deletion-closed languages containing $L$, denoted by $\operatorname{Del}(L)$, is called the \emph{deletion-closure} of $L$.  In other words, $\operatorname{Del}(L)$ is the smallest deletion-closed language containing $L$.

The deletion-closure of a language $L$ can be constructed from $L$, as follows:
\begin{eqnarray*}
L_0 &:=& L \\
L_{n+1} &:=& L_n \longrightarrow (L_n \cup \lbrace \lambda \rbrace) \\
L' &:=& \bigcup_{i=0}^{\infty} L_i
\end{eqnarray*}
Then $\operatorname{Del}(L) = L'$.

For example, if $L=\lbrace a^p \mid p \mbox{ is a prime number }\rbrace$, then $\operatorname{Del}(L)=\lbrace a^n \mid n\ge 0 \rbrace$, and if $L=\lbrace a^m b^n \mid m\ge n\ge 0\rbrace$, then $\operatorname{Del}(L)=\lbrace a^m b^n \mid m,n\ge 0 \rbrace$.

\textbf{Remarks}.
\begin{itemize}
\item If $L_1,L_2$ are regular languages, so is $L_1 \longrightarrow L_2$.  In other words, the family $\mathscr{R}$ of regular languages is closed under the deletion binary operation.
\item In addition, $\mathscr{R}$ is closed under deletion closure: if $L$ is regular, so is $\operatorname{Del}(L)$.
\item However, $\mathscr{F}$, the family of context-free languages, is neither closed under deletion, nor under deletion closure.
\end{itemize}

\begin{thebibliography}{9}
\bibitem{mi} M. Ito, {\em Algebraic Theory of Automata and Languages}, World Scientific, Singapore (2004).
\end{thebibliography}
%%%%%
%%%%%
\end{document}
