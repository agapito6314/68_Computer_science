\documentclass[12pt]{article}
\usepackage{pmmeta}
\pmcanonicalname{InsertionOperationOnLanguages}
\pmcreated{2013-03-22 18:56:53}
\pmmodified{2013-03-22 18:56:53}
\pmowner{CWoo}{3771}
\pmmodifier{CWoo}{3771}
\pmtitle{insertion operation on languages}
\pmrecord{6}{41805}
\pmprivacy{1}
\pmauthor{CWoo}{3771}
\pmtype{Definition}
\pmcomment{trigger rebuild}
\pmclassification{msc}{68Q70}
\pmclassification{msc}{68Q45}
\pmsynonym{insertion-closed}{InsertionOperationOnLanguages}
\pmsynonym{insertion-closure}{InsertionOperationOnLanguages}
\pmrelated{DeletionOperationOnLanguages}
\pmrelated{ShuffleOfLanguages}
\pmdefines{insertion}
\pmdefines{insertion closed}
\pmdefines{insertion closure}
\pmdefines{$k$-insertion}

\usepackage{amssymb,amscd}
\usepackage{amsmath}
\usepackage{amsfonts}
\usepackage{mathrsfs}

% used for TeXing text within eps files
%\usepackage{psfrag}
% need this for including graphics (\includegraphics)
%\usepackage{graphicx}
% for neatly defining theorems and propositions
\usepackage{amsthm}
% making logically defined graphics
%%\usepackage{xypic}
\usepackage{pst-plot}

% define commands here
\newcommand*{\abs}[1]{\left\lvert #1\right\rvert}
\newtheorem{prop}{Proposition}
\newtheorem{thm}{Theorem}
\newtheorem{ex}{Example}
\newcommand{\real}{\mathbb{R}}
\newcommand{\pdiff}[2]{\frac{\partial #1}{\partial #2}}
\newcommand{\mpdiff}[3]{\frac{\partial^#1 #2}{\partial #3^#1}}
\begin{document}
Let $\Sigma$ be an alphabet, and $u,v$ words over $\Sigma$.  An \emph{insertion} of $u$ into $v$ is a word of the form $v_1uv_2$, where $v=v_1v_2$.  The concatenation of two words is just a special case of insertion.  Also, if $w$ is an insertion of $u$ into $v$, then $v$ is a deletion of $u$ from $w$.

\emph{The insertion} of $u$ into $v$ is the set of all insertions of $v$ into $u$, and is denoted by $v\rhd u$.

The notion of insertion can be extended to languages.  Let $L_1,L_2$ be two languages over $\Sigma$.  The insertion of $L_1$ into $L_2$, denoted by $L_1 \rhd L_2$, is the set of all insertions of words in $L_1$ into words in $L_2$.  In other words, $$L_1 \rhd L_2 = \bigcup \lbrace u\rhd v\mid u\in L_1, v\in L_2 \rbrace.$$  So $u \rhd v = \lbrace u\rbrace \rhd \lbrace v\rbrace$.

A language $L$ is said to be \emph{insertion closed} if $L\rhd L \subseteq L$.  Clearly $\Sigma^*$ is insertion closed, and arbitrary intersection of insertion closed languages is insertion closed.  Given a language $L$, the \emph{insertion closure} of $L$, denoted by $\operatorname{Ins}(L)$, is the smallest insertion closed language containing $L$.  It is clear that $\operatorname{Ins}(L)$ exists and is unique.

More to come...

The concept of insertion can be generalized.  Instead of the insertion of $u$ into $v$ taking place in one location in $v$, the insertion can take place in several locations, provided that $u$ must also be broken up into pieces so that each individual piece goes into each inserting location.  More precisely, given a positive integer $k$, a \emph{$k$-insertion of $u$ into $v$} is a word of the form $$v_1 u_1 \cdots v_k u_k v_{k+1}$$ where $u=u_1 \cdots u_k$ and $v=v_1 \cdots v_{k+1}$.  So an insertion is just a $1$-insertion.  \emph{The $k$-insertion} of $u$ into $v$ is the set of all $k$-insertions of $u$ into $v$, and is denoted by $u \rhd^{[k]} v$.  Clearly $\rhd^{[1]} = \rhd$.

\textbf{Example}.  Let $\Sigma=\lbrace a,b\rbrace$, and $u=aba$, $v=bab$, and $w=bababa$.  Then 
\begin{itemize}
\item $w$ is an insertion of $u$ into $v$, as well as an insertion of $v$ into $u$, for $w=vu\lambda = \lambda v u$.
\item $w$ is also a $2$-insertion of $u$ into $v$: 
\begin{itemize}
\item
the decompositions $u=(ab)(a)$ and $v=(b)(ab)\lambda$
\item
or the decompositions  $u=\lambda u$ and $v=\lambda v\lambda$
\end{itemize}
produce $(b)(ab)(ab)(a)\lambda = \lambda \lambda v u \lambda = w$.
\item Finally, $w$ is also a $2$-insertion of $v$ into $u$, as the decompositions $u = \lambda u \lambda$ and $v= v \lambda$ produce $\lambda v u\lambda  \lambda = w$.
\item $u\rhd v=\lbrace ababab, babaab, baabab, bababa\rbrace$.
\end{itemize}

From this example, we observe that a $k$-insertion is a $(k+1)$-insertion, and every $k$-insertion of $u$ into $v$ is a $(k+1)$-insertion of $v$ into $u$.  As a result, $$u \rhd^{[k]} v \subseteq (u \rhd^{[k+1]} v) \cap (v \rhd^{[k+1]} u).$$

As before, given languages $L_1$ and $L_2$, \emph{the $k$-insertion} of $L_1$ into $L_2$, denoted by $L_1 \rhd^{[k]} L_2$, is the union of all $u \rhd^{[k]} v$, where $u\in L_1$ and $v\in L_2$.

\textbf{Remark}.  Some closure properties regarding insertions: let $\mathscr{R}$ be the family of regular languages, and $\mathscr{F}$ the family of context-free languages.  Then $\mathscr{R}$ is closed under $\rhd^{[k]}$, for each positive integer $k$.  $\mathscr{F}$ is closed $\rhd^{[k]}$ only when $k=1$.  If $L_1\in \mathscr{R}$ and $L_2\in \mathscr{F}$, then $L_1 \rhd^{[k]} L_2$ and $L_2 \rhd^{[k]} L_1$ are both in $\mathscr{F}$.  The proofs of these closure properties can be found in the reference.

\begin{thebibliography}{9}
\bibitem{mi} M. Ito, {\em Algebraic Theory of Automata and Languages}, World Scientific, Singapore (2004).
\end{thebibliography}
%%%%%
%%%%%
\end{document}
