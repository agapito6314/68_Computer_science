\documentclass[12pt]{article}
\usepackage{pmmeta}
\pmcanonicalname{Btree}
\pmcreated{2013-03-22 17:22:08}
\pmmodified{2013-03-22 17:22:08}
\pmowner{Mathprof}{13753}
\pmmodifier{Mathprof}{13753}
\pmtitle{B-tree}
\pmrecord{9}{39730}
\pmprivacy{1}
\pmauthor{Mathprof}{13753}
\pmtype{Definition}
\pmcomment{trigger rebuild}
\pmclassification{msc}{68P10}
\pmclassification{msc}{68P05}
\pmdefines{order}

% this is the default PlanetMath preamble.  as your knowledge
% of TeX increases, you will probably want to edit this, but
% it should be fine as is for beginners.

% almost certainly you want these
\usepackage{amssymb}
\usepackage{amsmath}
\usepackage{amsfonts}

% used for TeXing text within eps files
%\usepackage{psfrag}
% need this for including graphics (\includegraphics)
%\usepackage{graphicx}
% for neatly defining theorems and propositions
%\usepackage{amsthm}
% making logically defined graphics
%%%\usepackage{xypic}

% there are many more packages, add them here as you need them

% define commands here

\begin{document}
A \emph{B-tree of \PMlinkescapetext{order} $m$} is a \PMlinkname{balanced}{BalancedTree} search tree in 
which 
\begin{itemize}
\item
every node has at most $m$ children,
\item 
every node, other than a root or leaf, has at least $m/2$ children,
\item
the root has at least 2 children if it is not a leaf,
\item
all leaf nodes appear on the same level
\item
a nonleaf node with $k$ children has $k-1$ keys.
\end{itemize}


If the tree has $n$ nodes its height is $O(log_2(n))$.

%%%%%
%%%%%
\end{document}
