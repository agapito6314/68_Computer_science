\documentclass[12pt]{article}
\usepackage{pmmeta}
\pmcanonicalname{ComputationallyIndistinguishable}
\pmcreated{2013-03-22 13:03:11}
\pmmodified{2013-03-22 13:03:11}
\pmowner{Henry}{455}
\pmmodifier{Henry}{455}
\pmtitle{computationally indistinguishable}
\pmrecord{5}{33457}
\pmprivacy{1}
\pmauthor{Henry}{455}
\pmtype{Definition}
\pmcomment{trigger rebuild}
\pmclassification{msc}{68Q30}

% this is the default PlanetMath preamble.  as your knowledge
% of TeX increases, you will probably want to edit this, but
% it should be fine as is for beginners.

% almost certainly you want these
\usepackage{amssymb}
\usepackage{amsmath}
\usepackage{amsfonts}

% used for TeXing text within eps files
%\usepackage{psfrag}
% need this for including graphics (\includegraphics)
%\usepackage{graphicx}
% for neatly defining theorems and propositions
%\usepackage{amsthm}
% making logically defined graphics
%%%\usepackage{xypic}

% there are many more packages, add them here as you need them

% define commands here
%\PMlinkescapeword{theory}
\begin{document}
If $\{D_n\}_{n\in\mathbb{N}}$ and $\{E_n\}_{n\in\mathbb{N}}$ are distribution ensembles (on $\Omega$) then we say they are \emph{computationally indistinguishable} if for any probabilistic, polynomial time algorithm $A$ and any polynomal function $f$ there is some $m$ such that for all $n>m$:

$$|\operatorname{Prob}_A(D_n)=\operatorname{Prob}_A(E_n)|<\frac{1}{p(n)}$$

where $\operatorname{Prob}_A(D_n)$ is the probability that $A$ accepts $x$ where $x$ is chosen according to the distribution $D_n$.
%%%%%
%%%%%
\end{document}
