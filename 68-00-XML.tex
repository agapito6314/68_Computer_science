\documentclass[12pt]{article}
\usepackage{pmmeta}
\pmcanonicalname{XML}
\pmcreated{2013-03-22 15:31:43}
\pmmodified{2013-03-22 15:31:43}
\pmowner{yota}{10184}
\pmmodifier{yota}{10184}
\pmtitle{XML}
\pmrecord{5}{37408}
\pmprivacy{1}
\pmauthor{yota}{10184}
\pmtype{Definition}
\pmcomment{trigger rebuild}
\pmclassification{msc}{68-00}

% this is the default PlanetMath preamble.  as your knowledge
% of TeX increases, you will probably want to edit this, but
% it should be fine as is for beginners.

% almost certainly you want these
\usepackage{amssymb}
\usepackage{amsmath}
\usepackage{amsfonts}

% used for TeXing text within eps files
%\usepackage{psfrag}
% need this for including graphics (\includegraphics)
%\usepackage{graphicx}
% for neatly defining theorems and propositions
%\usepackage{amsthm}
% making logically defined graphics
%%%\usepackage{xypic}

% there are many more packages, add them here as you need them

% define commands here
\begin{document}
          Extensible Markup Language (XML) is a flexible, text based, format derived from SGML. XML was originally designed for satisfying electronic publishing work. XML is a standard for data exchange on the Web and Internet. 
XML and SGML can be used for describing the structure of metadata. 
          XML documents are made up of storage units called entities, which contain either parsed or unparsed data. Parsed data consists of character data or markup. Markup encodes a description of the document's storage layout and logical structure. XML provides a mechanism to impose constraints on the storage layout and logical structure.

The \PMlinkescapetext{specifications} are published and maintained by the \PMlinkexternal{W3C}{http://www.w3.org/XML/}, they are \PMlinkescapetext{open} and freely available on their website.
%%%%%
%%%%%
\end{document}
