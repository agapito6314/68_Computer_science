\documentclass[12pt]{article}
\usepackage{pmmeta}
\pmcanonicalname{MatrixCharacterizationsOfAutomata}
\pmcreated{2013-03-22 19:02:13}
\pmmodified{2013-03-22 19:02:13}
\pmowner{CWoo}{3771}
\pmmodifier{CWoo}{3771}
\pmtitle{matrix characterizations of automata}
\pmrecord{7}{41913}
\pmprivacy{1}
\pmauthor{CWoo}{3771}
\pmtype{Definition}
\pmcomment{trigger rebuild}
\pmclassification{msc}{68Q05}
\pmclassification{msc}{68Q42}
\pmclassification{msc}{03D10}

\endmetadata

\usepackage{amssymb,amscd}
\usepackage{amsmath}
\usepackage{amsfonts}
\usepackage{mathrsfs}

% used for TeXing text within eps files
%\usepackage{psfrag}
% need this for including graphics (\includegraphics)
\usepackage{graphicx}
% for neatly defining theorems and propositions
\usepackage{amsthm}
% making logically defined graphics
%%\usepackage{xypic}
\usepackage{pst-plot}

% define commands here
\newcommand*{\abs}[1]{\left\lvert #1\right\rvert}
\newtheorem{prop}{Proposition}
\newtheorem{thm}{Theorem}
\newtheorem{lem}{Lemma}
\newtheorem{ex}{Example}
\newcommand{\real}{\mathbb{R}}
\newcommand{\pdiff}[2]{\frac{\partial #1}{\partial #2}}
\newcommand{\mpdiff}[3]{\frac{\partial^#1 #2}{\partial #3^#1}}
\begin{document}
Let $A=(S,\Sigma,\delta,I,F)$ be a finite automaton.  Recall that $A$ can be visualized by a directed graph $G_A$ called the state diagram of $A$.  The nodes of $G_A$ are the states of $A$ (elements of $S$), and the edges of $G_A$ are labeled by the input symbols of $A$ (elements of $\Sigma$).

Suppose $S=\lbrace s_1,\ldots, s_n\rbrace$.  For each symbol $a\in \Sigma$, we define an $n\times n$ matrix $M(a)$ over the non-negative integers as follows: the cell 
\begin{displaymath}
M_{ij}:= \left\{
\begin{array}{ll}
1 & \textrm{if }s_j\in \delta(s_i,a), \\
0 & \textrm{otherwise.}
\end{array}
\right.
\end{displaymath}
In other words, $M(a)$ is a matrix composed of $0$'s and $1$'s, such that cell $(i,j)$ is $1$ provided that there is an edge from node $s_i$ to $s_j$ with label $a$.  $M$ may be viewed as a function from $\Sigma$ to the set $\mathfrak{M}(n)$ of $n\times n$ matrices whose entries are $1$'s and $0$'s, mapping $a$ to $M(a)$ just described.  

To completely describe $A$, we use two $n$-dimensional vectors $v_I$ and $v_F$ to specify $I$ and $F$ respectively.  The $i$-th component of $v_I$ is $1$ if and only if $s_i$ is a start state.  Otherwise, it is a $0$.  $v_F$ is defined similarly.  Thus, the triple $(M,v_I,v_F)$ describes $A$.

The following is the state diagram of an automaton with two states $s_1,s_2$ over $\lbrace a,b\rbrace$, and its description by matrices:

\begin{figure}[!h]
\centering
\scalebox{1.0}{\includegraphics{automaton_mat.eps}}
\end{figure}

Conversely, given a triple $(M,v,w)$, where $M:\Sigma \to \mathfrak{M}(n)$ is a function, and $v,w$ are two $n$-dimensional vectors $\lbrace 0,1\rbrace$, we can construct an automaton $A_M$ as follows: $A_M=(S,\Sigma,\delta,I,F)$ where 
\begin{enumerate}
\item $S$ has $n$ elements $s_1,\ldots, s_n$;
\item $I$ is a subset of $S$ such that $s_i\in I$ iff the $i$-th component of $v$ is $1$;
\item $F$ is a subset of $S$ such that $s_j\in F$ iff the $j$-th component of $w$ is $1$;
\item for each pair $(s_i,a)\in S\times \Sigma$, $\delta(s_i,a)$ is the subset of $S$ such that $s_j\in \delta(s_i,a)$ iff cell $(i,j)$ of $m(a)$ is $1$.
\end{enumerate}

Let us look more closely now at the function $M$.  Given a function $M:\Sigma \to \mathfrak{M}(n)$, we may extend it in a unique way to a homomorphism from $\Sigma^*$ to $\mathfrak{M}(n)^*$, the monoid of $n\times n$ matrices generated by $\mathfrak{M}(n)$, where the multiplication is defined by the ordinary matrix multiplication.  In other words, if $u_1,u_2$ are words over $\Sigma$, $$m(u_1u_2)=m(u_1)m(u_2),$$ the product of matrices $m(u_1)$ and $m(u_2)$.  We use the same notation $M$ for this extension.  In the example above, we see that
$$M(a^2) = \left(\begin{array}{ccc} 0 & 1 \\ 0 & 1 \end{array} \right), \quad 
M(ab) = \left(\begin{array}{ccc} 1 & 0 \\ 1 & 0 \end{array} \right), \quad \mbox{and} \quad
M(b^2) = \left(\begin{array}{ccc} 0 & 0 \\ 0 & 0 \end{array} \right).$$

The following two lemmas are some consequences:

\begin{lem} For any word $u$ over $\Sigma$, cell $(i,j)$ of the matrix $M(u)$ is the number of paths from $s_i$ to $s_j$ with label $u$.  \end{lem}

Treating $v_I$ as a row vector and $v_F$ as a column vector, we get

\begin{lem} For any word $u$ over $\Sigma$, 
\begin{itemize}
\item the $i$-th component of the row vector $v_I M(u)$ is the number of paths from a start state to $s_i$ with label $u$.  
\item the $j$-th component of the column vector $M(u) v_F$ is the number of paths from $s_j$ to a final state with label $u$.
\item $v_I M(u) v_F$ is the number of paths from a start state to a final state with label $u$.
\end{itemize}
\end{lem}

Combining the two lemmas, it is easy to see that

\begin{prop}  A language $R$ over $\Sigma$ is regular iff it can be expressed by a triple $(M,v,w)$ described above in the following sense: $$R=\lbrace u\in \Sigma^* \mid vM(u)w > 0 \rbrace$$ where $v$ is written as a row vector, and $w$ is written as a column vector.
\end{prop}
%%%%%
%%%%%
\end{document}
