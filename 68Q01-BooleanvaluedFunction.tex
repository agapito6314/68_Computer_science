\documentclass[12pt]{article}
\usepackage{pmmeta}
\pmcanonicalname{BooleanvaluedFunction}
\pmcreated{2013-03-22 17:58:12}
\pmmodified{2013-03-22 17:58:12}
\pmowner{Jon Awbrey}{15246}
\pmmodifier{Jon Awbrey}{15246}
\pmtitle{Boolean-valued function}
\pmrecord{10}{40476}
\pmprivacy{1}
\pmauthor{Jon Awbrey}{15246}
\pmtype{Definition}
\pmcomment{trigger rebuild}
\pmclassification{msc}{68Q01}
\pmclassification{msc}{68P01}
\pmclassification{msc}{06E30}
\pmclassification{msc}{03G05}
\pmclassification{msc}{03C55}
\pmclassification{msc}{03B05}
\pmclassification{msc}{68R01}
\pmsynonym{characteristic function}{BooleanvaluedFunction}
\pmsynonym{indicator function}{BooleanvaluedFunction}
\pmsynonym{predicate}{BooleanvaluedFunction}
\pmsynonym{proposition}{BooleanvaluedFunction}
\pmrelated{BooleanDomain}
\pmdefines{Boolean function}

% this is the default PlanetMath preamble.  as your knowledge
% of TeX increases, you will probably want to edit this, but
% it should be fine as is for beginners.

% almost certainly you want these
\usepackage{amssymb}
\usepackage{amsmath}
\usepackage{amsfonts}

% used for TeXing text within eps files
%\usepackage{psfrag}
% need this for including graphics (\includegraphics)
%\usepackage{graphicx}
% for neatly defining theorems and propositions
%\usepackage{amsthm}
% making logically defined graphics
%%%\usepackage{xypic}

% there are many more packages, add them here as you need them

% define commands here

\begin{document}
\PMlinkescapeword{necessary}
\PMlinkescapeword{Necessary}
\PMlinkescapeword{object}
\PMlinkescapeword{Object}
\PMlinkescapeword{objects}
\PMlinkescapeword{Objects}
\PMlinkescapeword{term}
\PMlinkescapeword{Term}
\PMlinkescapeword{terms}
\PMlinkescapeword{Terms}

A \textbf{Boolean-valued function} is a function of type $f : X \to \mathbb{B}$, where $X$ is an arbitrary set and where $\mathbb{B}$ is a Boolean domain.

A \textbf{Boolean function} (of finite arity $k$) is a Boolean-valued function of type $f : \mathbb{B}^k \to \mathbb{B}$.

In some contexts a Boolean-valued function may be referred to as a \textit{predicate} or a \textit{proposition}.  In other contexts these terms may be used to describe the syntactic entities that denote or express Boolean-valued functions, or that have Boolean-valued functions among their canonical or intended \PMlinkname{models}{Model}.  When necessary to avoid confusion, terms like \textit{propositional expression} or \textit{predicate formula} may be used to distinguish \PMlinkname{signs}{SignRelation} from their objects.

%%%%%
%%%%%
\end{document}
