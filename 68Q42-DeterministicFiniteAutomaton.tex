\documentclass[12pt]{article}
\usepackage{pmmeta}
\pmcanonicalname{DeterministicFiniteAutomaton}
\pmcreated{2013-03-22 12:26:37}
\pmmodified{2013-03-22 12:26:37}
\pmowner{CWoo}{3771}
\pmmodifier{CWoo}{3771}
\pmtitle{deterministic finite automaton}
\pmrecord{16}{32553}
\pmprivacy{1}
\pmauthor{CWoo}{3771}
\pmtype{Definition}
\pmcomment{trigger rebuild}
\pmclassification{msc}{68Q42}
\pmclassification{msc}{68Q05}
\pmclassification{msc}{03D10}
\pmsynonym{dfa}{DeterministicFiniteAutomaton}
\pmsynonym{finite state machine}{DeterministicFiniteAutomaton}
\pmsynonym{fsm}{DeterministicFiniteAutomaton}
\pmrelated{Automaton}
\pmrelated{ContextFreeLanguage}
\pmrelated{RegularLanguage}
\pmrelated{Language}
\pmrelated{NonDeterministicFiniteAutomaton}
\pmrelated{SubsetConstruction}

\endmetadata

\usepackage{amssymb}
\usepackage{amsmath}
\usepackage{amsfonts}
\usepackage[all]{xy}
\begin{document}
A \emph{deterministic finite automaton} (or DFA) is a deterministic automaton with a finite input alphabet and a finite number of states.  It can be formally defined as a 5-tuple
$(S, \Sigma, \delta, q_0, F)$, where 
\begin{itemize}
\item $S$ is a non-empty finite set of \emph{states}, 
\item $\Sigma$ is the alphabet (defining what set of input strings the automaton operates on), 
\item $\delta : S\times\Sigma\rightarrow S$ is the \emph{transition function},
\item $q_0 \in S$ is the starting state, and 
\item $F \subseteq S$ is a set of final (or accepting states).  
\end{itemize}
A DFA works exactly like a general automaton: operation begins at $q_0$, and movement from state to state is governed by the transition function $\delta$.  A word is accepted exactly when a final state is reached upon reading the last (rightmost) symbol of the word.

DFAs represent regular languages, and can be used to test whether any string in $\Sigma^*$ is in the language it represents.  Consider the following regular language over the alphabet $\Sigma := \left\{ \verb.a., \verb.b. \right\}$ (represented by the regular expression \verb=aa*b=):

\begin{eqnarray*}
\verb.<.S\verb.>. & \verb.::=. & \verb=a=\,A \\
\verb.<.A\verb.>. & \verb.::=. & b\,\verb.|.\,\verb=a=\,A \\
\end{eqnarray*}

This language can be represented by the DFA with the following state diagram:

$$
\UseComputerModernTips
\xymatrix {
    \ar[r] &
        *+[o][F-]{0} \ar[r]_a \ar[rd]_b &
        *+[o][F-]{1} \ar@(r,u)[]_a \ar[r]_b &
        *++[o][F=]{2} \ar[dl]^{a,b}
    \\
        &&
        *+[o][F-]{3} \ar@(r,d)[]^{a,b}
}
$$

The vertex 0 is the initial state $q_0$, and the vertex 2 is the only state in $F$.  Note that for every vertex there is an edge leading away from it with a label for each symbol in $\Sigma$.  This is a requirement of DFAs, which guarantees that operation is well-defined for any finite string.

If given the string \verb=aaab= as input, operation of the DFA above is as follows.  The first \verb=a= is removed from the input string, so the edge from 0 to 1 is followed.  The resulting input string is \verb=aab=.  For each of the next two \verb=a=s, the edge is followed from 1 to itself.  Finally, \verb=b= is read from the input string and the edge from 1 to 2 is followed.  Since the input string is now $\lambda$, the operation of the DFA halts.  Since it has halted in the accepting state 2, the string \verb=aaab= is accepted as a sentence in the regular language implemented by this DFA.

Now let us trace operation on the string \verb=aaaba=.  Execution is as above, until state 2 is reached with \verb=a= remaining in the input string.  The edge from 2 to 3 is then followed and the operation of the DFA halts.  Since 3 is not an accepting state for this DFA, \verb=aaaba= is \emph{not} accepted.

\textbf{Remarks}.
\begin{itemize}
\item 
A DFA can be modified to include \emph{$\epsilon$-transitions}{EpsilonTransitions}.  But the resulting DFA can be simulated by another DFA (without any epsilon transitions).
\item
Although the operation of a DFA is much easier to compute than that of a non-deterministic automaton, it is non-trivial to directly generate a DFA from a regular grammar.  It is much easier to generate a non-deterministic finite automaton from the regular grammar, and then \emph{transform} the non-deterministic finite automaton into a DFA.
\end{itemize}
%%%%%
%%%%%
\end{document}
