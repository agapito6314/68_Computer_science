\documentclass[12pt]{article}
\usepackage{pmmeta}
\pmcanonicalname{SortingProblem}
\pmcreated{2013-03-22 11:43:37}
\pmmodified{2013-03-22 11:43:37}
\pmowner{Logan}{6}
\pmmodifier{Logan}{6}
\pmtitle{sorting problem}
\pmrecord{10}{30127}
\pmprivacy{1}
\pmauthor{Logan}{6}
\pmtype{Algorithm}
\pmcomment{trigger rebuild}
\pmclassification{msc}{68P10}
\pmclassification{msc}{82C35}
%\pmkeywords{sort}
%\pmkeywords{algorithm}
%\pmkeywords{comparison}
\pmrelated{TotalOrder}
\pmrelated{PartialOrder}
\pmrelated{Relation}
\pmrelated{Heapsort}
\pmrelated{Bubblesort}
\pmrelated{BinarySearch}
\pmrelated{InPlaceSortingAlgorithm}
\pmrelated{InsertionSort}
\pmrelated{LandauNotation}
\pmrelated{Quicksort}
\pmrelated{SelectionSort}

\usepackage{amssymb}
\usepackage{amsmath}
\usepackage{amsfonts}
\usepackage{graphicx}
%%%%%%%\usepackage{xypic}
\begin{document}
Let $\leq$ be a \emph{total ordering} on the set $S$.
Given a sequence of $n$ elements, $x_1, x_2, \dots, x_n \in S$, find a sequence
of distinct indices $1 \leq i_1, i_2, \dots, i_n \leq n$ such that $x_{i_1} \leq x_{i_2} \leq \dots \leq x_{i_n}$.
\\
\\
The sorting problem is a heavily studied area of computer science, and many \emph{sorting algorithms} exist to
solve it.  The most general algorithms depend only upon the relation $\leq$, and are called \emph{comparison-based sorts}.
It can be proved that the lower bound for sorting by any comparison-based sorting algorithm is $\Omega(n\log{n})$.
\\
\\
A few other specialized sort algorithms rely on particular properties of the values of elements in $S$ (such as their structure) in order
to achieve lower time complexity for sorting certain sets of elements.  Examples of such a sorting algorithm are
the \emph{bucket sort} and the \emph{radix sort}.
%%%%%
%%%%%
%%%%%
%%%%%
%%%%%
%%%%%
%%%%%
\end{document}
