\documentclass[12pt]{article}
\usepackage{pmmeta}
\pmcanonicalname{BinarySearch}
\pmcreated{2013-03-22 11:44:34}
\pmmodified{2013-03-22 11:44:34}
\pmowner{mathcam}{2727}
\pmmodifier{mathcam}{2727}
\pmtitle{binary search}
\pmrecord{18}{30165}
\pmprivacy{1}
\pmauthor{mathcam}{2727}
\pmtype{Algorithm}
\pmcomment{trigger rebuild}
\pmclassification{msc}{68P10}
\pmclassification{msc}{16S40}
\pmclassification{msc}{20G42}
\pmclassification{msc}{16W35}
\pmclassification{msc}{81R50}
\pmclassification{msc}{16W30}
\pmclassification{msc}{57T05}
\pmclassification{msc}{54-01}
%\pmkeywords{lookup}
\pmrelated{TotalOrder}
\pmrelated{PartialOrder}
\pmrelated{SortingProblem}
\pmrelated{InsertionSort}

\usepackage{amssymb}
\usepackage{amsmath}
\usepackage{amsfonts}
%\usepackage{graphicx}
%%%%%%%\usepackage{xypic}

\newcommand{\Lindent}{0.4in}
\newenvironment{Lalgorithm}[4]{
    \textbf{Algorithm} \textsc{#1}\texttt{(#2)}\newline
    \textit{Input}: #3\newline
    \textit{Output}: #4

}{}
\newenvironment{Lfloatalgorithm}[6][h]{
    \begin{figure}[#1]
    \caption{#2}
    \begin{Lalgorithm}{#3}{#4}{#5}{#6}
}{
    \end{Lalgorithm}
    \end{figure}
}
\newcommand{\Lgets}{\ensuremath{\gets}}
\newcommand{\Lgroup}[1]{\textbf{begin}\\\hspace*{\Lindent}\parbox{\textwidth}{#1}\\\textbf{end}}
\newcommand{\Lif}[2]{\textbf{if} #1 \textbf{then}\\\hspace*{\Lindent}\parbox{\textwidth}{#2}}
\newcommand{\Lelse}[1]{\textbf{else}\\\hspace*{\Lindent}\parbox{\textwidth}{#1}}
\newcommand{\Lelseif}[2]{\textbf{else if} #1 \textbf{then}\\\hspace*{\Lindent}\parbox{\textwidth}{#2}}
\begin{document}
\PMlinkescapeword{entire}
\PMlinkescapeword{relation}

\subsection*{The Problem}

Let $\leq$ be a total ordering on the set $S$.  Given a sequence of $n$ elements,
$L = \left\{ x_1 \leq x_2 \leq \dots \leq x_n \right\}$, and a value $y \in S$,
locate the position of any elements in $L$ that are equal to $y$, or determine that
none exist.

\subsection*{The Algorithm}

The \emph{binary search} technique is a fundamental method for locating an element of
a particular value within a sequence of sorted elements (see Sorting Problem).
The idea is to eliminate half of the search space with each comparison.

First, the
middle element of the sequence is compared to the value we are searching for.  If this
element matches the value we are searching for, we are done.  If, however, the middle
element is ``less than'' the value we are chosen for (as specified by the \PMlinkname{relation}{Relation} used
to specify a total order over the set of elements), then we know that, if the value exists
in the sequence, it must exist somewhere \emph{after} the middle element.  Therefore we
can eliminate the first half of the sequence from our search and simply repeat the search
in the exact same manner on the remaining half of the sequence.  If, however, the value we
are searching for comes before the middle element, then we repeat the search on the first half
of the sequence.

\subsection*{Pseudocode}

%\begin{lstlisting}{}
\begin{Lalgorithm}{Binary\_Search}{L, n, key}{A list $L$ of $n$ elements, and $key$ (the search key)}{$Position$ (such that $X[Position] = key$)}
\Lgroup{
    $Position \gets Find(L, 1, n, key);$
}\\
\textbf{function} $Find(L, bottom, top, key)$ \\
\Lgroup{
    \Lif{$bottom = top$}{
        \Lif{$L[bottom] = key$}{$Find \gets bottom$}
        \Lelse{$Find \gets 0$}}
    \Lelse{
    \Lgroup{
        $middle \gets (bottom + top) / 2;$ \\
        \Lif{$key < L[middle]$}{
            $Find \gets Find(L, bottom, middle - 1, key)$}
        \Lelse{
            $Find \gets Find(L, middle + 1, top, key)$}
    }}
}
\end{Lalgorithm}
%\end{lstlisting}

\subsection*{Analysis}

We can specify the runtime complexity of this binary search algorithm by counting the number
of comparisons required to locate some element $y$ in $L$.  Since half of the list is eliminated
with each comparison, there can be no more than $\log_2{n}$ comparisons before either the positions
of all the $y$ elements are found or the entire list is eliminated and $y$ is determined to not exist
in $L$.
Thus the worst-case runtime complexity of the binary search is $\mathcal{O}(\log{n})$.
It can also be shown that the average-case runtime complexity of the binary search is approximately
$\log_2{n}-1$ comparisons.
This means that any single entry in a phone book containing one million entries can be located
with at most 20 comparisons (and on average 19).
%%%%%
%%%%%
%%%%%
%%%%%
%%%%%
%%%%%
\end{document}
