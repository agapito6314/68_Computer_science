\documentclass[12pt]{article}
\usepackage{pmmeta}
\pmcanonicalname{FineAndWilfsTheoremOnWords}
\pmcreated{2013-03-22 18:00:39}
\pmmodified{2013-03-22 18:00:39}
\pmowner{Ziosilvio}{18733}
\pmmodifier{Ziosilvio}{18733}
\pmtitle{Fine and Wilf's theorem on words}
\pmrecord{18}{40525}
\pmprivacy{1}
\pmauthor{Ziosilvio}{18733}
\pmtype{Theorem}
\pmcomment{trigger rebuild}
\pmclassification{msc}{68Q70}
\pmclassification{msc}{68Q45}
\pmclassification{msc}{68R15}

\endmetadata

% this is the default PlanetMath preamble.  as your knowledge
% of TeX increases, you will probably want to edit this, but
% it should be fine as is for beginners.

% almost certainly you want these
\usepackage{amssymb}
\usepackage{amsmath}
\usepackage{amsfonts}

% used for TeXing text within eps files
%\usepackage{psfrag}
% need this for including graphics (\includegraphics)
%\usepackage{graphicx}
% for neatly defining theorems and propositions
%\usepackage{amsthm}
% making logically defined graphics
%%%\usepackage{xypic}

% there are many more packages, add them here as you need them

% define commands here
\newcommand{\naturals}{\ensuremath{\mathbb{N}}}

\newtheorem{theorem}{Theorem}
\begin{document}
%% Last modified: April 19, 2008
Let $w$ be a word on an alphabet $A$ and let $|w|$ be its length.
A period of $w$ is a value $p>0$
such that $w_{i+p}=w_{i}$ for every $i\in\{1,2,\ldots,|w|-p\}$.
This is the same as saying that $w=u^{k}r$
for some $u,r\in A^{\ast}$ and $k\in\naturals$
such that $|u|=p$ and $r$ is a prefix of $u$.

Fine and Wilf's theorem
gives a condition on the length of the periods a word can have.
\begin{theorem} \label{thm:fw}
Let $w$ be a word on an alphabet $A$ having periods $p$ and $q$.
If $|w|\geq p+q-\gcd(p,q)$,
then $w$ has period $\gcd(p,q)$.
The value $p+q-\gcd(p,q)$
is the smallest one that makes the theorem true.
\end{theorem}

As a counterexample showing that the condition on $|w|$ is necessary,
the word $aaabaaa$ has periods $4$ and $6$, but not $2=\gcd(4,6)$.
This can happen because its length is $7<4+6-2=8$.

Observe that Theorem~\ref{thm:fw} can be restated as follows
(cf.~\cite{l97}).
\begin{theorem} \label{thm:fw-alt}
Let $u$ and $w$ be words over an alphabet $A$.
Suppose $u^h$ and $w^k$, for some $h$ and $k$,
have a common prefix of length
\begin{math}
|u|+|w|-\gcd(|u|,|w|).
\end{math}
Then there exists $z\in A^\ast$ of length $\gcd(|u|,|w|)$
such that $u,w\in z^\ast$.
The value
\begin{math}
|u|+|w|-\gcd(|u|,|w|)
\end{math}
is also the smallest one that makes the theorem true.
\end{theorem}

In fact, Theorem~\ref{thm:fw} clearly implies Theorem~\ref{thm:fw-alt}.
Now, suppose Theorem~\ref{thm:fw-alt} is true.
Suppose $w$ has periods $p$ and $q$
and length at least $p+q-\gcd(p,q)$:
write $w=u^{k}r=v^{h}s$ with
$|u|=p$, $|v|=q$, $r$ prefix of $u$, $s$ prefix of $v$.
Let $M$ be a common multiple of $p$ and $q$ greater than $p$, $q$, and $|w|$:
then $u^{M/p}$ and $v^{M/q}$ have the common prefix $w$,
so they also have a common prefix of length $p+q-\gcd(u,v)$.
Then $u$ and $v$ are powers of a word $z$ of length $\gcd(p,q)$,
and it is easy to see that $w=z^{m}t$
for some $m>0$ and some prefix $t$ of $z$.

\begin{thebibliography}{99}

\bibitem{l97}
M. Lothaire.
\textit{Combinatorics on words.}
Cambridge University Press 1997.

\end{thebibliography}

%%%%%
%%%%%
\end{document}
