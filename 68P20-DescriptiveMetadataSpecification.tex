\documentclass[12pt]{article}
\usepackage{pmmeta}
\pmcanonicalname{DescriptiveMetadataSpecification}
\pmcreated{2013-03-22 15:31:29}
\pmmodified{2013-03-22 15:31:29}
\pmowner{yi6604}{10640}
\pmmodifier{yi6604}{10640}
\pmtitle{descriptive metadata specification}
\pmrecord{5}{37399}
\pmprivacy{1}
\pmauthor{yi6604}{10640}
\pmtype{Definition}
\pmcomment{trigger rebuild}
\pmclassification{msc}{68P20}

% this is the default PlanetMath preamble.  as your knowledge
% of TeX increases, you will probably want to edit this, but
% it should be fine as is for beginners.

% almost certainly you want these
\usepackage{amssymb}
\usepackage{amsmath}
\usepackage{amsfonts}

% used for TeXing text within eps files
%\usepackage{psfrag}
% need this for including graphics (\includegraphics)
%\usepackage{graphicx}
% for neatly defining theorems and propositions
%\usepackage{amsthm}
% making logically defined graphics
%%%\usepackage{xypic}

% there are many more packages, add them here as you need them

% define commands here
\begin{document}
Let $L$ = $\cup$ $Dk$ be a set of literals defined as the union of
domains $Dk$ of simple datatypes (e.g., strings, numbers, dates, etc.). Let also $R$ and $P$ represent sets of labels for resources and properties respectively. A descriptive metadata specification is a structure ($G$, $R$ $\cup$ $L$ $\cup$ $P$, $F$), where:
(1) $F$ : $(V$  $\cup$ $ E)$ $→$ $(R$ $\cup$ $L$ $\cup$ $P$) can assign general labels $R$ $\cup$ $P$ and literals from $L$ to nodes of the graph structure;
(2) for each directed edge $e$ = ($vi$ , $vj$) of $G$, $F$($vi$) $\in$ $R$ $\cup$ $L$, $F$($vj$ ) $\in$ $R$ $\cup$ $L$ and $F(e)$ $\in$ $P$;
(3) $F(vk)$ $\in$ $L$ if and only if node vk has outdegree 0.
%%%%%
%%%%%
\end{document}
