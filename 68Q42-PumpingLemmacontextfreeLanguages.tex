\documentclass[12pt]{article}
\usepackage{pmmeta}
\pmcanonicalname{PumpingLemmacontextfreeLanguages}
\pmcreated{2013-03-22 16:21:10}
\pmmodified{2013-03-22 16:21:10}
\pmowner{rspuzio}{6075}
\pmmodifier{rspuzio}{6075}
\pmtitle{pumping lemma (context-free languages)}
\pmrecord{8}{38487}
\pmprivacy{1}
\pmauthor{rspuzio}{6075}
\pmtype{Theorem}
\pmcomment{trigger rebuild}
\pmclassification{msc}{68Q42}
\pmsynonym{pumping lemma}{PumpingLemmacontextfreeLanguages}
\pmrelated{PumpingLemmaRegularLanguages}

% this is the default PlanetMath preamble.  as your knowledge
% of TeX increases, you will probably want to edit this, but
% it should be fine as is for beginners.

% almost certainly you want these
\usepackage{amssymb}
\usepackage{amsmath}
\usepackage{amsfonts}

% used for TeXing text within eps files
%\usepackage{psfrag}
% need this for including graphics (\includegraphics)
%\usepackage{graphicx}
% for neatly defining theorems and propositions
%\usepackage{amsthm}
% making logically defined graphics
%%%\usepackage{xypic}

% there are many more packages, add them here as you need them

% define commands here

\begin{document}
Let $L$ be a context-free language (a.k.a. type 2 language).  Then there exist
two integers $m$ and $n$ such that, if the length of a word $W$ is greater
than $m$, then $W = ABCDE$ where $A,B,C,D,E$ are subwords such that
\begin{enumerate}
\item The length of the subword $BCD$ is less than $n$.
\item $BD$ is not be empty.
\item For all integers $k > 0$, it is the case that $AB^kCD^kE$ belongs to $L$,
where exponentiation denotes repetition of a subword $k$ times.
\end{enumerate}

An important use of this lemma is that it allows one to show that a language
is not context-free. (Remember, just because a language happens to be described 
in terms of a context-sensitive grammar does not automatically preclude the
possibility of describing the same language also  by a 
context-free language.)  The idea is to assume that the language is
context-free, then arrive at a contradiction via this lemma.

As an illustrative example, consider the following language, which consists of
but one terminal symbol, `x' and which consists of all strings of `x' 's whose
length is a perfect square.  Were this a context-free language, there would 
exist integers $m$ and $n$ as above.  Choose an integer $h$ such that $h^2 > m$.
Then the word $x^{h^2}$ belongs to our language and the lemma tells us that
it can be written as $ABCDE$ so as to satisfy the conditions enumerated above.
Write $A = x^a$, $B = x^b$, $C = x^c$, $D = x^d$, $E = x^e$ for suitable nonnegative integers $a,b,c,d,e$.  Then we have $a + b + c + d + e = k^2$;
by condition , $b + d > 0$ and, by condition , $a + kb + c + kd + e$ would 
have to be a perfect square because $AB^kCD^kE$ would be a word of the
language.  This, however, leads to a contradiction: $h^2 + k (b + d)$ 
could not possibly be a perfect square for all $k$ unless $b + d = 0$.

As an exercise, the reader may consider constructing a context-sensitive
grammar for this language and posting it as an attachment to this entry 
(at which time this sentence will come down).
%%%%%
%%%%%
\end{document}
