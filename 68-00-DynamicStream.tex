\documentclass[12pt]{article}
\usepackage{pmmeta}
\pmcanonicalname{DynamicStream}
\pmcreated{2013-03-22 15:31:34}
\pmmodified{2013-03-22 15:31:34}
\pmowner{zshao}{11001}
\pmmodifier{zshao}{11001}
\pmtitle{Dynamic Stream}
\pmrecord{4}{37405}
\pmprivacy{1}
\pmauthor{zshao}{11001}
\pmtype{Definition}
\pmcomment{trigger rebuild}
\pmclassification{msc}{68-00}
\pmrelated{Information}

% this is the default PlanetMath preamble.  as your knowledge
% of TeX increases, you will probably want to edit this, but
% it should be fine as is for beginners.

% almost certainly you want these
\usepackage{amssymb}
\usepackage{amsmath}
\usepackage{amsfonts}

% used for TeXing text within eps files
%\usepackage{psfrag}
% need this for including graphics (\includegraphics)
%\usepackage{graphicx}
% for neatly defining theorems and propositions
%\usepackage{amsthm}
% making logically defined graphics
%%%\usepackage{xypic}

% there are many more packages, add them here as you need them

% define commands here
\begin{document}
A dynamic stream can represent an information flow --- a sequence of messages
encoded by the sender and communicated using a transmission channel
possibly distorted with noise, to a receiver whose goal is to reconstruct the
sender's messages and interpret message semantics.
---
From:
"SHANNON, C. E. 1948. A mathematical theory of communication. Bell Syst. Tech. J. 27, 379--423, 623--656."
%%%%%
%%%%%
\end{document}
