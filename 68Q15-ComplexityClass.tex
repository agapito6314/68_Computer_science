\documentclass[12pt]{article}
\usepackage{pmmeta}
\pmcanonicalname{ComplexityClass}
\pmcreated{2013-03-22 13:01:59}
\pmmodified{2013-03-22 13:01:59}
\pmowner{Henry}{455}
\pmmodifier{Henry}{455}
\pmtitle{complexity class}
\pmrecord{4}{33433}
\pmprivacy{1}
\pmauthor{Henry}{455}
\pmtype{Definition}
\pmcomment{trigger rebuild}
\pmclassification{msc}{68Q15}

% this is the default PlanetMath preamble.  as your knowledge
% of TeX increases, you will probably want to edit this, but
% it should be fine as is for beginners.

% almost certainly you want these
\usepackage{amssymb}
\usepackage{amsmath}
\usepackage{amsfonts}

% used for TeXing text within eps files
%\usepackage{psfrag}
% need this for including graphics (\includegraphics)
%\usepackage{graphicx}
% for neatly defining theorems and propositions
%\usepackage{amsthm}
% making logically defined graphics
%%%\usepackage{xypic}

% there are many more packages, add them here as you need them

% define commands here
%\PMlinkescapeword{theory}
\begin{document}
If $f(n)$ is any function and $T$ is a Turing machine (of any kind) which halts on all inputs, we say that $T$ is time bounded by $f(n)$ if for any input $x$ with length $|x|$, $T$ halts after at most $f(|x|)$ steps.

For a decision problem $L$ and a class $K$ of Turing machines, we say that $L\in KTIME(f(n))$ if there is a Turing machine $T\in K$ bounded by $f(n)$ which decides $L$.  If $R$ is a search problem then $R\in KTIME(f(n))$ if $L(R)\in KTIME(f(n))$.  The most common classes are all restricted to one read-only input tape and one output/work tape (and in some cases a one-way, read-only guess tape) and are defined as follows:

\begin{itemize}
\item $D$ is the class of deterministic Turing machines. (In this case $KTIME$ is written $DTIME$.)

\item $N$ is the class of non-deterministic Turing machines (and so $KTIME$ is written $NTIME$).

\item $R$ is the class of positive one-sided error Turing machines and $coR$ the class of negative one-sided error machines

\item $BP$ is the class of two-sided error machines

\item $P$ is the class of minimal error machines

\item $ZP$ is the class of zero error machines
\end{itemize}

Although $KTIME(f(n))$ is a time complexity class for any $f(n)$, in actual use time complexity classes are usually the union of $KTIME(f(n))$ for many $f$.  If $\Phi$ is a class of functions then $KTIME(\Phi)=\bigcup_{f\in\Phi} KTIME(f(n))$.  Most commonly this is used when $\Phi=\mathcal{O}(f(n))$.

The most important time complexity classes are the polynomial classes:

$$K\mathcal{P}=\bigcup_{i\in\mathbb{N}} KTIME(n^i)$$

When $K=D$ this is called just $\mathcal{P}$, the class of problems decidable in polynomial time.  One of the major outstanding problems in mathematics is the question of whether $\mathcal{P}=\mathcal{NP}$.

We say a problem $\pi\in KSPACE(f(n))$ if there is a Turing machine $T\in K$ which solves $\pi$, always halts, and never uses more than $f(n)$ cells of its output/work tape.  As above, if $\Phi$ is a class of function then $KSPACE(\Phi)=\bigcup_{f\in\Phi} KSPACE(f(n))$

The most common space complexity classes are $K\mathcal{L}=KSPACE(\mathcal{O}(\log n))$.  When $K=D$ this is just called $\mathcal{L}$.

If $\mathcal{C}$ is any complexity class then $\pi\in co\mathcal{C}$ if $\pi$ is a decision problem and $\overline{\pi}\in\mathcal{C}$ or $\pi$ is a search problem and $L(\pi)\in co\mathcal{C}$.  Of course, this coincides with the definition of $coR$ above.  Clearly $co(co\mathcal{C})=\mathcal{C}$.

Since a machine with a time complexity $f(n)$ cannot possibly use more than $f(n)$ cells, $KTIME(f(n))\subseteq KSPACE(f(n))$.  If $K\subseteq K^\prime$ then $KTIME(f(n))\subseteq K^\prime TIME(f(n))$ and similarly for space.

The following are all trivial, following from the fact that some classes of machines accept and reject under stricter circumstances than others:

$$D\subseteq ZP=R\cap coR$$
$$R\cup coR\subseteq BP\cap N$$
$$BP \subseteq P$$
%%%%%
%%%%%
\end{document}
