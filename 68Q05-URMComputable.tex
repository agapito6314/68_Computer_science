\documentclass[12pt]{article}
\usepackage{pmmeta}
\pmcanonicalname{URMComputable}
\pmcreated{2013-03-22 19:03:42}
\pmmodified{2013-03-22 19:03:42}
\pmowner{CWoo}{3771}
\pmmodifier{CWoo}{3771}
\pmtitle{URM computable}
\pmrecord{14}{41945}
\pmprivacy{1}
\pmauthor{CWoo}{3771}
\pmtype{Definition}
\pmcomment{trigger rebuild}
\pmclassification{msc}{68Q05}
\pmclassification{msc}{03D10}
\pmsynonym{URM-computable}{URMComputable}

\usepackage{amssymb,amscd}
\usepackage{amsmath}
\usepackage{amsfonts}
\usepackage{mathrsfs}

% used for TeXing text within eps files
%\usepackage{psfrag}
% need this for including graphics (\includegraphics)
%\usepackage{graphicx}
% for neatly defining theorems and propositions
\usepackage{amsthm}
% making logically defined graphics
%%\usepackage{xypic}
\usepackage{pst-plot}

% define commands here
\newcommand*{\abs}[1]{\left\lvert #1\right\rvert}
\newtheorem{prop}{Proposition}
\newtheorem{thm}{Theorem}
\newtheorem{ex}{Example}
\newcommand{\real}{\mathbb{R}}
\newcommand{\pdiff}[2]{\frac{\partial #1}{\partial #2}}
\newcommand{\mpdiff}[3]{\frac{\partial^#1 #2}{\partial #3^#1}}
\begin{document}
Let $M$ be an unlimited register machine (URM), and $r$ a finite sequence of non-negative integers.  Recall the following notations:
\begin{itemize}
\item $M(r)$ denotes the computation of $r$ by the program of $M$,
\item $M(r)\!\downarrow$ denotes that the computation halts ($M$ converges on $r$),
\item $M(r)\!\downarrow a$ denotes $M(r)\!\downarrow$, and $a$ is the content of register 1 in the output,
\item $M(r)\!\uparrow$ denotes that the computation does not halt ($M$ diverges $r$).
\end{itemize}
In the case where all but finitely many values of $r$ are $0$, say $r=r_1, r_2, \ldots, r_n, 0, 0, \ldots$, we also write $M(r_1,\ldots,r_n)$ to emphasize the fact that $r_i=0$ for all $i>n$.

\textbf{Definition}.  Let $f: \mathbb{N}^n \to \mathbb{N}$ be an $n$-ary partial function on natural numbers (including $0$ in this discussion).  $f$ is said to be \emph{URM-computable} if there is a URM $M$ such that $M(r_1,\ldots,r_n)\!\downarrow f(r_1,\ldots, r_n)$ precisely when $(r_1,\ldots,r_n)\in \operatorname{dom}(f)$.  When $f$ is URM-computable by $M$, we also say that $M$ \emph{computes} $f$.

In other words, if $(r_1,\ldots, r_n)$ is in the domain of $f$, then we have a halting computation 
\begin{center}
\begin{tabular}{ p{0.7cm} }
start
\end{tabular}
\begin{tabular}{|c|c|c|c|c|c}
\hline
$r_1$ & $\cdots$ & $r_n$ & $0$ & $0$ & $\cdots$ \\
\hline
\end{tabular}
$$\vdots$$
\begin{tabular}{ p{0.5cm} }
halt
\end{tabular}
\begin{tabular}{|c|c|c|c}
\hline
$f(r_1,\ldots,r_n)$ & $\cdot$ & $\cdot$ & $\cdots$ \\
\hline
\end{tabular}
\end{center}
If on the other hand $(r_1,\ldots, r_n)$ is not in the domain of $f$, then the computation of the above input never terminates.

For example, $f(r_1,r_2)=r_1+r_2$, addition of two non-negative integers, is URM-computable, as is shown in \PMlinkname{this entry}{ExamplesOfUnlimitedRegisterMachines}.

Here are two more basic examples:

\begin{itemize}
\item (subtraction by $1$): $f(r_1)=r_1-1$.  Note that $f$ is a partial function that is not total, because $f(0)$ is not defined.  A URM that computes $f$ is the following:
$$M = J(1,4,1),S(2),J(1,2,6),S(3),J(1,1,2),T(3,1) $$
First, $M$ compares the $r_1$ with $r_4:=0$.  If they are the same, it loops indefinitely.  Otherwise, $M$ increments $r_2$ by $1$, and then compares $r_1$ with $r_2$.  If they are the same, then $M$ transfers $r_3:=0$ in register $3$ to $r_1$ in register $1$.  Otherwise, it increments $r_3$ by $1$ and loops back to the second instruction.  The computation continues until $r_1=r_2$, and when this happens, $r_1$ is set to be $r_3$.

\item (monus operation): $f(r_1)=r_1 \dot{-} 1$.  This is like the last example, except $f(0):=0$.  All we have to do is to modify the URM above:
$$M = J(1,4,6),S(2),J(1,2,6),S(3),J(1,1,2),T(3,1) $$
so the first instruction jumps to the last instruction when $r_1=r_4$, instead of looping.
\item (parity checking): $f(r_1)=1$ if $r_1$ is odd, and $f(r_1)=0$ otherwise.  In other words, $f(r_1)$ is the remainder of the division of $r_1$ by $2$.  A URM that computes $f$ is the following:
\begin{eqnarray*}
M &=& J(1,2,14),T(1,2),S(2),S(3),S(3),J(1,3,9),J(2,3,11),J(1,1,4),\\
&& Z(1),J(1,1,14),Z(1),S(1),J(1,1,14)
\end{eqnarray*}
Basically, with input $r_1:=m$, $M$ first sets $r_2:=m+1$.  Then by incrementing $r_3$ by $2$, $M$ tests whether $r_1=r_3$ or $r_2=r_3$.  If the former, then $M$ sets $r_1:=0$, otherwise $r_1$ is set to $1$.  The computation stops when the program jumps to the non-existent instruction $14$.
\end{itemize}

\textbf{Remarks}.  
\begin{itemize}
\item For any URM $M$ and any positive integer $n$, $M$ computes a unique $n$-ary (partial) function $f$.  This can be simply done as follows: take the contents $r$ of the first $n$ registers of the tape as input, and run $M$.  Define a partial function $f: \mathbb{N}^n \to \mathbb{N}$ so that $r\in \operatorname{dom}(f)$ iff $M(r)\downarrow$, and when this is the case, set $f(r)$ to be the integer such that $M(r)\downarrow \! f(r)$.  

\textbf{Examples}.
\begin{itemize}
\item $T(5,2)$ computes, for any $n>0$, the $n$-ary function $f(x_1,\ldots, x_n)=x_1$.
\item $T(5,1)$ computes $f(x_1,\ldots,x_n)=0$ for any $0<n<5$, and $g(x_1,\ldots,x_n)=x_5$ for any $n\ge 5$.
\item $J(1,1,1)$ computes the empty function $\varnothing$ for all $n\ge 0$.
\end{itemize}

\item More generally, a partial function $f:\mathbb{N}^n \to \mathbb{N}^m$ is said to be URM-computable iff there is a URM $M$ such that $M(r_1,\ldots,r_n)\!\downarrow$, and the $i$-th coordinate of $f(r_1,\ldots, r_n)$ is the content of the $i$-th register, $i\in \lbrace 1,\ldots, m\rbrace$, precisely when $(r_1,\ldots,r_n)\in \operatorname{dom}(f)$.

The function $f$ above can be expressed as $(g_1,\ldots, g_m)$, where each $g_i: \mathbb{N}^n \to \mathbb{N}$.  Then it is not hard to show that $f$ is URM-computable iff each $g_i$ is URM-computable.
\item
One of the fundamental facts about URM computability is the following: a function is URM computable iff it is Turing computable.  By Church's thesis, this means that URM computability is equivalent to effective computability.
\end{itemize}

\begin{thebibliography}{9}
\bibitem{SS} J. C. Shepherdson, H. E. Sturgis, {\em Computability of Recursive Functions}. Journal Assoc. Comput. Mach. 10, 217-255, (1963).
\bibitem{nc} N. Cutland, {\em Computability: An Introduction to Recursive Function Theory}. Cambridge University Press, (1980).
\end{thebibliography}
%%%%%
%%%%%
\end{document}
