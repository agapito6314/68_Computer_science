\documentclass[12pt]{article}
\usepackage{pmmeta}
\pmcanonicalname{GardenofEdenTheorem}
\pmcreated{2013-03-22 19:22:07}
\pmmodified{2013-03-22 19:22:07}
\pmowner{Ziosilvio}{18733}
\pmmodifier{Ziosilvio}{18733}
\pmtitle{Garden-of-Eden theorem}
\pmrecord{9}{42322}
\pmprivacy{1}
\pmauthor{Ziosilvio}{18733}
\pmtype{Theorem}
\pmcomment{trigger rebuild}
\pmclassification{msc}{68Q80}
\pmclassification{msc}{37B15}

\endmetadata

% this is the default PlanetMath preamble.  as your knowledge
% of TeX increases, you will probably want to edit this, but
% it should be fine as is for beginners.

% almost certainly you want these
\usepackage{amssymb}
\usepackage{amsmath}
\usepackage{amsfonts}

% used for TeXing text within eps files
%\usepackage{psfrag}
% need this for including graphics (\includegraphics)
%\usepackage{graphicx}
% for neatly defining theorems and propositions
%\usepackage{amsthm}
% making logically defined graphics
%%%\usepackage{xypic}

% there are many more packages, add them here as you need them

% define commands here

\begin{document}
\newcommand{\Acal}{\ensuremath{\mathcal{A}}}
\newcommand{\Neigh}{\ensuremath{\mathcal{N}}}
\newcommand{\restrict}[2]{\ensuremath{\left.{#1}\right|_{#2}}}
\newcommand{\Zset}{\ensuremath{\mathbb{Z}}}

\newtheorem{corollary}{Corollary}
\newtheorem{lemma}{Lemma}
\newtheorem{theorem}{Theorem}

\usepackage{hyperref}

Edward F. Moore's Garden-of-Eden theorem
was the first rigorous statement of cellular automata theory.
Though originally stated for cellular automata on the plane,
it works in arbitrary dimension.
\begin{theorem}[Moore, 1962] \label{thm:mo}
Let
\begin{math}
\Acal = \langle Q, \Neigh, f \rangle
\end{math}
be a cellular automaton on $\Zset^d$ with finitely many states.
If $\Acal$ has two mutually erasable patterns,
then it has an orphan pattern.
\end{theorem}
In other words,
surjective $d$-dimensional cellular automata are pre-injective.

The same year, John Myhill proved the converse implication.
\begin{theorem}[Myhill, 1962] \label{thm:my}
Let
\begin{math}
\Acal = \langle Q, \Neigh, f \rangle
\end{math}
be a cellular automaton on $\Zset^d$ with finitely many states.
If $\Acal$ has an orphan pattern,
then it has two mutually erasable patterns.
\end{theorem}
In other words,
pre-injective $d$-dimensional cellular automata are surjective.

\begin{corollary}
An injective $d$-dimensional cellular automaton is surjective.
\end{corollary}

Theorems~\ref{thm:mo} and~\ref{thm:my} hold because of the following statement.
\begin{lemma} \label{lem:mo}
Let $a > 0$, $d \geq 1$, $r \geq 1$, $k \geq 1$.
For every sufficiently large $n$,
\begin{equation} \label{eq:mo}
(a^{k^d}-1)^{n^d} < a^{(kn-2r)^d}
\end{equation}
\end{lemma}
Observe that,
if $a = |Q|$ and
\begin{equation} \label{eq:Dr}
\Neigh
= \{ x \in \Zset^d \mid |x_i| \leq r \, \forall i \in \{1,\ldots,d\} \}
\;,
\end{equation}
then $(a^{kn})^d$
is the number of possible hypercubic patterns of side $kn$,
while $(a^{kn-2r})^d$
is the maximum number of their images via $f$.

\textit{Proof.}
The inequality (\ref{eq:mo}) is in fact satisfied
precisely by those $n$ that satisfy
\begin{displaymath}
\log_a (a^{k^d}-1) < \left( k - \frac{2r}{n} \right)^d \;,
\end{displaymath}
which is true for $n$ large enough since
\begin{math}
\log_a (a^{k^d}-1) < k^d
\end{math}
while
\begin{math}
\lim_{n \to \infty} (k - 2r/n)^d = k^d.
\end{math}
\hfill $\Box$

For the rest of this entry, we will suppose
that the neighborhood index $\Neigh$
has the form (\ref{eq:Dr}).

\textit{Proof of Moore's theorem.}
Suppose $\Acal$ has two mutually erasable patterns $p_1$, $p_2$:
it is not restrictive to suppose that their common support
is a $d$-hypercube of side $k$.
Define an equivalence relation $\rho$ on patterns of side $n$
by stating that $p \rho q$ if and only if $f(p) = f(q)$:
since $p_1$ and $p_2$ are mutually erasable,
$\rho$ has at most $|Q|^{k^d}-1$ equivalence classes.

By the same criterion,
we define a family $\{\rho_n\}_{n \geq 1}$\
of equivalence relations between hypercubic patterns of side $kn$.
Each such pattern can be subdivided
into $n^d$ subpatterns of side $k$:
and it is clear that,
if two patterns are in relation $\rho_n$,
then each of those subpatterns is in relation $\rho$.

But then, the relation $\rho_n$
cannot have more than $(|Q|^{k^d}-1)^{n^d}$ equivalence classes:
consequently, at most $(|Q|^{k^d}-1)^{n^d}$ patterns of side $kn-2r$
can have a preimage.
By Lemma~\ref{lem:mo} with $a=|Q|$,
for $n$ large enough, some patterns of side $kn-2r$ must be orphan.
\hfill $\Box$

\textit{Proof of Myhill's theorem.}
Let $p$ be an orphan pattern for $\Acal$.
Again, it is not restrictive to suppose
that the support of $p$ is a hypercube of side $k$.
For $n \geq 1$
let $\nu_n$ be the number of patterns of side $kn$
that are \emph{not} orphan.
Split each pattern of side $kn$
into $n^d$ patterns of side $k$, as we did before.
Then $\nu_n$ cannot exceed the number of those
that do not contain a copy of $p$,
which in turn is at most $(|Q|^{k^d}-1)^{n^d}$.

Take $n$ so large that (\ref{eq:mo}) is satisfied:
for a fixed state $q_0 \in Q$, and for $q_1 = f(q_0,\ldots,q_0)$,
there are more configurations that take value $q_0$
outside the hypercube $\{0,\ldots,kn-2r-1\}^d$
than there are configurations that take value $q_1$
outside the hypercube $\{-r,\ldots,kn-1\}^d$
and are not Gardens of Eden.
As the configurations of the second type
are the images of those the first type,
there must be two with the same image:
those configurations correspond to two mutually erasable patterns.
\hfill $\Box$

Theorems~\ref{thm:mo} and ~\ref{thm:my}
have been shown~\cite{csms99}
to hold for cellular automata on amenable groups.
Moore's theorem, in fact, \emph{characterizes} amenable groups
(cf.~\cite{b10}):
whether Myhill's theorem also does,
is an open problem.

\begin{thebibliography}{99}

\bibitem{b10}
Bartholdi, L. (2010)
Gardens of Eden and amenability on cellular automata.
\textit{J. Eur. Math. Soc.} \textbf{12(1)}, 141--148.
Preprint: \href{http://arxiv.org/abs/0709.4280}{\texttt{arXiv:0709.4280v1}}

\bibitem{csms99}
Ceccherini-Silberstein, T., Mach\`{\i}, A. and Scarabotti, F. (1999)
Amenable groups and cellular automata.
\textit{Annales de l'Institut Fourier, Grenoble} \textbf{49(2)}, 673--685.

\bibitem{mo62}
Moore, E.F. (1962)
Machine models of self-reproduction.
\textit{Proc. Symp. Appl. Math.} \textbf{14}, 17--33.

\bibitem{my62}
Myhill, J. (1962)
The converse of Moore's Garden-of-Eden theorem.
\textit{Proc. Amer. Mat. Soc.} \textbf{14}, 685--686.

\end{thebibliography}

%%%%%
%%%%%
\end{document}
