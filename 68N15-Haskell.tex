\documentclass[12pt]{article}
\usepackage{pmmeta}
\pmcanonicalname{Haskell}
\pmcreated{2013-03-22 16:47:13}
\pmmodified{2013-03-22 16:47:13}
\pmowner{PrimeFan}{13766}
\pmmodifier{PrimeFan}{13766}
\pmtitle{Haskell}
\pmrecord{6}{39019}
\pmprivacy{1}
\pmauthor{PrimeFan}{13766}
\pmtype{Definition}
\pmcomment{trigger rebuild}
\pmclassification{msc}{68N15}

\endmetadata

% this is the default PlanetMath preamble.  as your knowledge
% of TeX increases, you will probably want to edit this, but
% it should be fine as is for beginners.

% almost certainly you want these
\usepackage{amssymb}
\usepackage{amsmath}
\usepackage{amsfonts}

% used for TeXing text within eps files
%\usepackage{psfrag}
% need this for including graphics (\includegraphics)
%\usepackage{graphicx}
% for neatly defining theorems and propositions
%\usepackage{amsthm}
% making logically defined graphics
%%%\usepackage{xypic}

% there are many more packages, add them here as you need them

% define commands here

\begin{document}
{\em Haskell} is a computer programming language designed by a committee in 1990 to consolidate the best features of the many purely functional programming languages that were created in the late 1980s.  Haskell is thus neither a procedural programming language nor an object-oriented one, although it offers monads such as {\tt do} to support procedural programming and classes with inheritance to support object-oriented programming (there is also a variant of Haskell called O'Haskell which includes more support for object-oriented programming).  In general, Haskell programs are most naturally written declaratively.

The standard version of the language is Haskell 98; Haskell 2007 hasn't been released yet but is expected to be only a minor revision of Haskell 98.

The standard Haskell prelude includes the function {\tt gcd}, which computes the greatest common divisor of two integers.  The following Haskell code is a reimplementation of the {\tt gcd} function.

% gcd.hs -- compute the gcd of two integers
% Copyright (C) 2007 by Michael Slone.
%
% This program is free software.  You may read, study, 
% copy, modify, and distribute modified copies of this
% program according to the terms of the GNU General 
% Public License, version 2.
%
% The first line,
%   mygcd :: Int -> Int -> Int ,
% is a type declaration.  It indicates that mygcd expects to
% receive two integers and expects to return a single integer.
% The function is defined using equations.  If no equation 
% applies, an error occurs.  For example, trying ``mygcd 5 5.2''
% yields the following error message in ghci.
%
% *Main> mygcd 5 5.2
% 
% <interactive>:1:8:
%     No instance for (Fractional Int)
%       arising from the literal `5.2' at <interactive>:1:8-10
%     Probable fix: add an instance declaration for (Fractional Int)
%     In the second argument of `mygcd', namely `5.2'
%     In the definition of `it': it = mygcd 5 5.2
%
% Whitespace can matter in Haskell, so if you wish to use this 
% code, please copy directly from the TeX source of this page 
% instead of the HTML version.

\begin{verbatim}
-- gcd.hs -- compute the gcd of two integers
-- View this page in TeX mode for documentation and license.

mygcd :: Int -> Int -> Int
mygcd m n
  | (n < 0)   = mygcd m (abs n)
  | (n == 0)  = m
  | (m < n)   = mygcd n m
  | otherwise = mygcd n (mymod m n)

mydiv :: Int -> Int -> Int
mydiv m n
  | (m < 0)   = negate (mydiv (negate m) n)
  | (n < 0)   = negate (mydiv m (negate n))
  | (m < n)   = 0
  | otherwise = 1 + mydiv (m-n) n

mymod :: Int -> Int -> Int
mymod m n = m - n * (mydiv m n)
\end{verbatim}

\PMlinkescapeword{classes}
\PMlinkescapeword{code}
\PMlinkescapeword{currying}
\PMlinkescapeword{functional}
\PMlinkescapeword{gcd}
\PMlinkescapeword{information}
\PMlinkescapeword{language}
\PMlinkescapeword{languages}
\PMlinkescapeword{line}
\PMlinkescapeword{minor}
\PMlinkescapeword{mode}
\PMlinkescapeword{support}
\PMlinkescapeword{type}

%%%%%
%%%%%
\end{document}
