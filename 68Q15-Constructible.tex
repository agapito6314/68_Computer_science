\documentclass[12pt]{article}
\usepackage{pmmeta}
\pmcanonicalname{Constructible}
\pmcreated{2013-03-22 13:03:20}
\pmmodified{2013-03-22 13:03:20}
\pmowner{Henry}{455}
\pmmodifier{Henry}{455}
\pmtitle{constructible}
\pmrecord{7}{33461}
\pmprivacy{1}
\pmauthor{Henry}{455}
\pmtype{Definition}
\pmcomment{trigger rebuild}
\pmclassification{msc}{68Q15}
\pmdefines{time constructible}
\pmdefines{space constructible}

% this is the default PlanetMath preamble.  as your knowledge
% of TeX increases, you will probably want to edit this, but
% it should be fine as is for beginners.

% almost certainly you want these
\usepackage{amssymb}
\usepackage{amsmath}
\usepackage{amsfonts}

% used for TeXing text within eps files
%\usepackage{psfrag}
% need this for including graphics (\includegraphics)
%\usepackage{graphicx}
% for neatly defining theorems and propositions
%\usepackage{amsthm}
% making logically defined graphics
%%%\usepackage{xypic}

% there are many more packages, add them here as you need them

% define commands here
%\PMlinkescapeword{theory}
\begin{document}
A function $f:\mathbb{N}\rightarrow\mathbb{N}$ is \emph{time constructible} if there is a deterministic Turing machine $T$ (with alphabet $\{0,1,B\}$) such that when $T$ receives as input the a series of $n$ ones, it halts after exactly $f(n)$ steps.  Similarly $f$ is \emph{space constructible} if there is a similar Turing machine which halts after using exactly $f(n)$ cells.

Most 'natural' functions are both time and space constructible, including constant functions, polynomials, and exponentials, for example.
%%%%%
%%%%%
\end{document}
