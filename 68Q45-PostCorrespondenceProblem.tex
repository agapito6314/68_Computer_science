\documentclass[12pt]{article}
\usepackage{pmmeta}
\pmcanonicalname{PostCorrespondenceProblem}
\pmcreated{2013-03-22 19:09:59}
\pmmodified{2013-03-22 19:09:59}
\pmowner{CWoo}{3771}
\pmmodifier{CWoo}{3771}
\pmtitle{Post correspondence problem}
\pmrecord{6}{42072}
\pmprivacy{1}
\pmauthor{CWoo}{3771}
\pmtype{Definition}
\pmcomment{trigger rebuild}
\pmclassification{msc}{68Q45}

\usepackage{amssymb,amscd}
\usepackage{amsmath}
\usepackage{amsfonts}
\usepackage{mathrsfs}

% used for TeXing text within eps files
%\usepackage{psfrag}
% need this for including graphics (\includegraphics)
%\usepackage{graphicx}
% for neatly defining theorems and propositions
\usepackage{amsthm}
% making logically defined graphics
%%\usepackage{xypic}
\usepackage{pst-plot}

% define commands here
\newcommand*{\abs}[1]{\left\lvert #1\right\rvert}
\newtheorem{prop}{Proposition}
\newtheorem{thm}{Theorem}
\newtheorem{ex}{Example}
\newcommand{\real}{\mathbb{R}}
\newcommand{\pdiff}[2]{\frac{\partial #1}{\partial #2}}
\newcommand{\mpdiff}[3]{\frac{\partial^#1 #2}{\partial #3^#1}}
\begin{document}
Let $\Sigma$ be an alphabet.  As usual, $\Sigma^+$ denotes the set of all non-empty words over $\Sigma$.  Let $P\subset \Sigma^+ \times \Sigma^+$ be finite.  Call a finite sequence 
$$(u_1,v_1),\ldots, (u_n,v_n)$$
of pairs in $P$ a \emph{correspondence} in $P$ if $$u_1\cdots u_n = v_1\cdots v_n.$$
The word $u_1\cdots u_n$ is called a \emph{match} in $P$.

For example, if $\Sigma=\lbrace a,b\rbrace$, and $P=\lbrace (b,b^2), (a^2,a), (b^2a,b^3), (ab^2,a^2b) \rbrace$.  Then $$(a^2,a), (ab^2,a^2b), (b^2a,b^3), (ab^2,a^2b), (b,b^2)$$ is a correspondence in $P$.

On the other hand, there are no correspondences in $\lbrace (ab, a), (a, ba^2)\rbrace$.

The Post correspondence problem asks the following:
\begin{quote}
Is there an algorithm (Turing machine or any other equivalent computing models) such that when an arbitrary $P$ is given as an input, returns $1$ if there exists a correspondence in $P$ and $0$ otherwise.
\end{quote}

The problem is named after E. Post because he proved
\begin{thm} The Post correspondence problem is unsolvable (no such algorithms exist). \end{thm}
%%%%%
%%%%%
\end{document}
