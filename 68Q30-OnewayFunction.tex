\documentclass[12pt]{article}
\usepackage{pmmeta}
\pmcanonicalname{OnewayFunction}
\pmcreated{2013-03-22 13:03:14}
\pmmodified{2013-03-22 13:03:14}
\pmowner{Henry}{455}
\pmmodifier{Henry}{455}
\pmtitle{one-way function}
\pmrecord{7}{33458}
\pmprivacy{1}
\pmauthor{Henry}{455}
\pmtype{Definition}
\pmcomment{trigger rebuild}
\pmclassification{msc}{68Q30}
\pmsynonym{one way function}{OnewayFunction}
\pmsynonym{one-way}{OnewayFunction}
\pmsynonym{one way}{OnewayFunction}

% this is the default PlanetMath preamble.  as your knowledge
% of TeX increases, you will probably want to edit this, but
% it should be fine as is for beginners.

% almost certainly you want these
\usepackage{amssymb}
\usepackage{amsmath}
\usepackage{amsfonts}

% used for TeXing text within eps files
%\usepackage{psfrag}
% need this for including graphics (\includegraphics)
%\usepackage{graphicx}
% for neatly defining theorems and propositions
%\usepackage{amsthm}
% making logically defined graphics
%%%\usepackage{xypic}

% there are many more packages, add them here as you need them

% define commands here
%\PMlinkescapeword{theory}
\begin{document}
A function $f$ is a \emph{one-way function} if for any probabilistic, polynomial time computable function $g$ and any polynomial function $p$ there is $m$ such that for all $n>m$:

$$\operatorname{Pr}[f(g(f(x)))=f(x)]<\frac{1}{p(n)}$$

where $x$ has length $n$ and all numbers of length $n$ are equally likely.

That is, no probabilistic, polynomial time function can effectively compute $f^{-1}$.

Note that, since $f$ need not be injective, this is a stricter requirement than
$$\operatorname{Pr}[g(f(x)))=x]<\frac{1}{p(n)}$$

since not only is $g(f(x))$ (almost always) not $x$, it is (almost always) no value such that $f(g(f(x)))=f(x)$.
%%%%%
%%%%%
\end{document}
