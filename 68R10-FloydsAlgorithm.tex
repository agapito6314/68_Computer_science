\documentclass[12pt]{article}
\usepackage{pmmeta}
\pmcanonicalname{FloydsAlgorithm}
\pmcreated{2013-03-22 12:16:29}
\pmmodified{2013-03-22 12:16:29}
\pmowner{vampyr}{22}
\pmmodifier{vampyr}{22}
\pmtitle{Floyd's algorithm}
\pmrecord{6}{31722}
\pmprivacy{1}
\pmauthor{vampyr}{22}
\pmtype{Algorithm}
\pmcomment{trigger rebuild}
\pmclassification{msc}{68R10}
\pmclassification{msc}{05C38}
\pmclassification{msc}{05C85}
\pmsynonym{all pairs shortest path algorithm}{FloydsAlgorithm}

\endmetadata

% this is the default PlanetMath preamble.  as your knowledge
% of TeX increases, you will probably want to edit this, but
% it should be fine as is for beginners.

% almost certainly you want these
\usepackage{amssymb}
\usepackage{amsmath}
\usepackage{amsfonts}

% used for TeXing text within eps files
%\usepackage{psfrag}
% need this for including graphics (\includegraphics)
%\usepackage{graphicx}
% for neatly defining theorems and propositions
%\usepackage{amsthm}
% making logically defined graphics
%%%\usepackage{xypic} 

% there are many more packages, add them here as you need them

% define commands here


\newcommand{\Lindent}{0.4in}
\newenvironment{Lalgorithm}[4]{
\textbf{Algorithm} \textsc{#1}\texttt{(#2)}\newline
\textit{Input}: #3\newline
\textit{Output}: #4\newline

}{}
\newenvironment{Lfloatalgorithm}[6][h]{
\begin{figure}[#1]
\caption{#2}
\begin{Lalgorithm}{#3}{#4}{#5}{#6}
}{
\end{Lalgorithm}
\end{figure}
}
\newcommand{\Lgets}{\ensuremath{\gets}}
\newcommand{\Lgroup}[1]{\textbf{begin}\\\hspace*{\Lindent}\parbox{\textwidth}{#1}\\\textbf{end}}
\newcommand{\Lif}[2]{\textbf{if} #1 \textbf{then}\\\hspace*{\Lindent}\parbox{\textwidth}{#2}}
\newcommand{\Lelse}[1]{\textbf{else}\\\hspace*{\Lindent}\parbox{\textwidth}{#1}}
\newcommand{\Lelseif}[2]{\textbf{else if} #1 \textbf{then}\\\hspace*{\Lindent}\parbox{\textwidth}{#2}}
\newcommand{\Lfor}[2]{\textbf{for} #1 \textbf{do}\\\hspace*{\Lindent}\parbox{\textwidth}{#2}}
\newcommand{\Lwhile}[2]{\textbf{while} #1 \textbf{do}\\\hspace*{\Lindent}\parbox{\textwidth}{#2}}
\begin{document}
Floyd's algorithm is also known as the all pairs shortest path algorithm.  It will compute the shortest path between all possible pairs of vertices in a (possibly weighted) graph or digraph simultaneously in $O(n^3)$ time (where $n$ is the number of vertices in the graph).

\begin{Lalgorithm}{Floyd}{V}{A weighted graph or digraph with vertices $V$}{A matrix $cost$ of shortest paths and a matrix $pred$ of predecessors in the shortest path}
\Lfor{$(a,b) \in V^2$}{
\Lif{$adjacent(a,b)$}{
$cost(a,b) \gets weight(a,b)$\\
$pred(a,b) \gets a$
}
\Lelse{
$cost(a,b) \gets \infty$\\
$pred(a,b) \gets null$
}
}

\Lfor{$c \in V$}{
\Lfor{$(a,b) \in V^2$}{
\Lif{$cost(a,c) < \infty$ and $cost(c,b) < \infty$}{
\Lif{$cost(a,b) = \infty$ or $cost(a,c) + cost(c,b) < cost(a,b)$}{
$cost(a,b) \gets cost(a,c)+cost(c,b)$ \\
$pred(a,b) \gets pred(c,b)$
}
}
}
}

\end{Lalgorithm}
%%%%%
%%%%%
\end{document}
