\documentclass[12pt]{article}
\usepackage{pmmeta}
\pmcanonicalname{Stack}
\pmcreated{2013-03-22 16:15:18}
\pmmodified{2013-03-22 16:15:18}
\pmowner{Mravinci}{12996}
\pmmodifier{Mravinci}{12996}
\pmtitle{stack}
\pmrecord{4}{38361}
\pmprivacy{1}
\pmauthor{Mravinci}{12996}
\pmtype{Definition}
\pmcomment{trigger rebuild}
\pmclassification{msc}{68P05}

% this is the default PlanetMath preamble.  as your knowledge
% of TeX increases, you will probably want to edit this, but
% it should be fine as is for beginners.

% almost certainly you want these
\usepackage{amssymb}
\usepackage{amsmath}
\usepackage{amsfonts}

% used for TeXing text within eps files
%\usepackage{psfrag}
% need this for including graphics (\includegraphics)
%\usepackage{graphicx}
% for neatly defining theorems and propositions
%\usepackage{amsthm}
% making logically defined graphics
%%%\usepackage{xypic}

% there are many more packages, add them here as you need them

% define commands here

\begin{document}
A \emph{stack} is a one-dimensional data structure to which new members or elements are generally added (pushed) at the end and removed (popped) from the start. (Compare this to a queue).

Obviously a new pushed element gets an index number one higher than the most recently pushed element. Whether the actual implementation puts the first element at the highest memory address available for the stack (and decreases the stack pointer with each push) or the lowest (and increases the stack with each push) there is no need to re-index elements as might be the case with a queue.
%%%%%
%%%%%
\end{document}
