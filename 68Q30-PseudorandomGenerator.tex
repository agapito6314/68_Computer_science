\documentclass[12pt]{article}
\usepackage{pmmeta}
\pmcanonicalname{PseudorandomGenerator}
\pmcreated{2013-03-22 13:03:16}
\pmmodified{2013-03-22 13:03:16}
\pmowner{Henry}{455}
\pmmodifier{Henry}{455}
\pmtitle{pseudorandom generator}
\pmrecord{6}{33460}
\pmprivacy{1}
\pmauthor{Henry}{455}
\pmtype{Definition}
\pmcomment{trigger rebuild}
\pmclassification{msc}{68Q30}
\pmdefines{stretch function}

% this is the default PlanetMath preamble.  as your knowledge
% of TeX increases, you will probably want to edit this, but
% it should be fine as is for beginners.

% almost certainly you want these
\usepackage{amssymb}
\usepackage{amsmath}
\usepackage{amsfonts}

% used for TeXing text within eps files
%\usepackage{psfrag}
% need this for including graphics (\includegraphics)
%\usepackage{graphicx}
% for neatly defining theorems and propositions
%\usepackage{amsthm}
% making logically defined graphics
%%%\usepackage{xypic}

% there are many more packages, add them here as you need them

% define commands here
%\PMlinkescapeword{theory}
\begin{document}
Let $G$ be a deterministic polynomial-time function from $\mathbb{N}^{<\omega}$ to $\mathbb{N}^{<\omega}$ with \emph{stretch function} $l:\mathbb{N}\rightarrow\mathbb{N}$, so that if $x$ has length $n$ then $G(x)$ has length $l(n)$.  Then let $G_n$ be the distribution on strings of length $l(n)$ defined by the output of $G$ on a randomly selected string of length $n$ selected by the uniform distribution.

Then we say $G$ is \emph{pseudorandom generator} if $\{G_n\}_{n\in\mathbb{N}}$ is pseudorandom.

In effect, $G$ translates a random input of length $n$ to a pseudorandom output of length $l(n)$.  Assuming $l(n)>n$, this expands a random sequence (and can be applied multiple times, since $G_n$ can be replaced by the distribution of $G(G(x))$).
%%%%%
%%%%%
\end{document}
