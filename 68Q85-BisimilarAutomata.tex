\documentclass[12pt]{article}
\usepackage{pmmeta}
\pmcanonicalname{BisimilarAutomata}
\pmcreated{2013-03-22 19:23:17}
\pmmodified{2013-03-22 19:23:17}
\pmowner{CWoo}{3771}
\pmmodifier{CWoo}{3771}
\pmtitle{bisimilar automata}
\pmrecord{14}{42343}
\pmprivacy{1}
\pmauthor{CWoo}{3771}
\pmtype{Definition}
\pmcomment{trigger rebuild}
\pmclassification{msc}{68Q85}
\pmclassification{msc}{68Q10}
\pmsynonym{strongly bisimilar automata}{BisimilarAutomata}

\usepackage{amssymb,amscd}
\usepackage{amsmath}
\usepackage{amsfonts}
\usepackage{mathrsfs}

% used for TeXing text within eps files
%\usepackage{psfrag}
% need this for including graphics (\includegraphics)
%\usepackage{graphicx}
% for neatly defining theorems and propositions
\usepackage{amsthm}
% making logically defined graphics
%%\usepackage{xypic}
\usepackage{pst-plot}

% define commands here
\newcommand*{\abs}[1]{\left\lvert #1\right\rvert}
\newtheorem{prop}{Proposition}
\newtheorem{thm}{Theorem}
\newtheorem{ex}{Example}
\newcommand{\real}{\mathbb{R}}
\newcommand{\pdiff}[2]{\frac{\partial #1}{\partial #2}}
\newcommand{\mpdiff}[3]{\frac{\partial^#1 #2}{\partial #3^#1}}

\begin{document}
Let $M=(S_M,\Sigma,\delta_M,I_M,F_M)$ and $N=(S_N,\Sigma,\delta_N,I_N,F_N)$ be two NDFA's (non-deterministic finite automata).  Let $\approx \subseteq S_M \times S_N$ be a binary relation between the states of the automata $M$ and $N$.  We may extend $\approx$ to a binary relation between subsets of the states of the automata, as follows: for any $P\subseteq S_M$ and $Q \subseteq S_N$, set
$$C(P):=\lbrace q\in S_N\mid p\approx q\mbox{ for some }p\in P\rbrace\mbox{ and }C(Q):=\lbrace p\in S_M\mid p\approx q\mbox{ for some }q\in Q\rbrace.$$
Then, using the same notation, define $\approx \subseteq P(S_M)\times P(S_N)$ by
$$P \approx Q \mbox{ iff }P \subseteq C(Q)\mbox{ and }Q \subseteq C(P).$$

\textbf{Definition}.  We say that $M$ is \emph{bisimilar} to $N$ if there is a binary relation $\approx \subseteq S_M \times S_N$ such that
\begin{enumerate}
\item $I_M \approx I_N$,
\item if $p \approx q$, then $\delta_M(p,a) \approx \delta_N(q,a)$ for any $a\in \Sigma$,
\item if $p \approx q$, then $p\in F_M$ iff $q\in F_N$.
\end{enumerate}
In other words, $M$ is bisimilar to $N$ as automata precisely when $M$ is bisimilar to $N$ as LTS, and satisfy conditions $1$ and $3$ above.

Any NDFA $M=(S,\Sigma,\delta,I,F)$ is bisimilar to itself, for the identity relation is clearly a bisimulation.  Next, if $M$ is bisimilar to $N$ with bisimulation $\approx$, $N$ is bisimilar to $M$ with the converse relation $\approx^{-1}$.  Finally, if $M$ is bisimilar to $N$ with bisimulation $\approx_1$ and $N$ is bisimilar to $P$ with bisimulation $\approx_2$, $M$ is bisimilar to $N$ with bisimulation $\approx_1\circ \approx_2$.  Therefore, bisimilarity is an equivalence relation on the class of NDFA's.

Another property of bisimulations on NDFA's is that the they are preserved by taking unions: an arbitrary non-empty union of bisimulations is again a bisimulation.  From this property, it is not hard to show that if $A\approx B$, then $\delta(A,x)\approx \delta(B,x)$ for any word over $\Sigma$.  As a result, bismilar NDFA's accept the same set of words.

By taking the union of all bisimulations on a given NDFA $M=(S,\Sigma,\delta,I,F)$, we get a bisimulation that is also an equivalence relation on the set of states of $M$.  For each $p\in S$, let $[p]$ be the equivalence class containing $p$, and for any subset $A\subseteq S$, let $[A]:=\lbrace [p]\mid p\in A\rbrace$.  Then we get an NDFA $[M]:=([S],\Sigma,[\Delta],[I],[F])$, with $$[\Delta]([p],a):=[\delta(p,a)]$$ for any $a\in \Sigma$.  It can be shown that $[M]$ is minimal in the sense that $[[M]]$ is isomorphic to $[M]$, and that $M$ is bisimilar to $[M]$.  In addition, if $M$ has no inaccessible states, then $M$ is bisimilar to a unique minimal automaton, in the sense that, if $N$ is any minimal automaton bisimlar to $M$, then $N$ is isomorphic to $[M]$.

%%%%%
%%%%%
\end{document}
