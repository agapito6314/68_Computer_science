\documentclass[12pt]{article}
\usepackage{pmmeta}
\pmcanonicalname{Monad}
\pmcreated{2013-03-22 12:27:43}
\pmmodified{2013-03-22 12:27:43}
\pmowner{mathcam}{2727}
\pmmodifier{mathcam}{2727}
\pmtitle{monad}
\pmrecord{11}{32614}
\pmprivacy{1}
\pmauthor{mathcam}{2727}
\pmtype{Definition}
\pmcomment{trigger rebuild}
\pmclassification{msc}{68Q70}
\pmclassification{msc}{18C15}
\pmsynonym{associative law of a monad}{Monad}
\pmsynonym{left and right identity laws of a monad}{Monad}
\pmrelated{NaturalTransformation}
\pmrelated{Functor}
\pmrelated{Category}
\pmrelated{FundamentalGroupoid2}

\endmetadata

\usepackage{amssymb}
\usepackage{amsmath}
\usepackage{amsfonts}
\usepackage[all]{xy}
\begin{document}
A \emph{monad} over a category $\mathcal{C}$ is a triple $(T,\eta,\mu)$, where
$T$ is an endofunctor of $\mathcal{C}$, $\eta$ is a natural transformation from the identity functor on $\mathcal{C}$, and $\mu$ is a natural transformations from $T\circ T$ to $T$, such that the following two properties hold:

\begin{itemize}
\item $\mu\circ(\mu\circ T)\equiv\mu\circ(T\circ\mu)$
\item $\mu\circ(T\circ\eta)\equiv\mathrm{id}_\mathcal{C}\equiv\mu\circ(\eta\circ T)$
\end{itemize}

These laws are illustrated in the following diagrams.

\begin{tabular}{cc}
\begin{minipage}{.48\textwidth}
$$
\xymatrix{
T^3(\mathcal{C}) \ar[rr]^{T\mu_{\mathcal{C}}} \ar[dd]_{\mu_{T(\mathcal{C})}} &&
T^2(\mathcal{C}) \ar[dd]^{\mu_{\mathcal{C}}} \\
\\
T^2(\mathcal{C}) \ar[rr]_{\mu_{\mathcal{C}}} && T(\mathcal{C})
}
$$
\end{minipage}
&
\begin{minipage}{.48\textwidth}
$$
\xymatrix{
T(\mathrm{id}_{\mathcal{C}}(A)) \ar[rr]^{T(\eta_{\mathcal{C}})}
\ar[ddrr]_{\mathrm{id}_{\mathcal{C}}} &&
T^2(\mathcal{C}) \ar[dd]^{\mu_{\mathcal{C}}} &&
\mathrm{id}_{\mathcal{C}}(T(\mathcal{C}))
\ar[ll]^{\eta^{T(\mathcal{C})}}
\ar[ddll]_{\mathrm{id}^{\mathcal{C}}} \\
\\
&& T(\mathcal{C})
}
$$
\end{minipage} \\
$\mu\circ(\mu\circ T)\equiv\mu\circ(T\circ\mu)$ &
$\mu\circ(T\circ\eta)\equiv\mathrm{id}_\mathcal{C}\equiv\mu\circ(\eta\circ T)$
\end{tabular}

As an application, monads have been successfully applied in the field of functional programming.  A pure functional program can have no side effects, but some computations are frequently much simpler with such behavior.  Thus a mathematical model of computation such as a monad is needed.  In this case, monads serve to represent state transformations, mutable variables, and interactions between a program and its environment.  For further information in this regard, see http://www.nomaware.com/monads/html/.
%%%%%
%%%%%
\end{document}
