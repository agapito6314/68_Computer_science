\documentclass[12pt]{article}
\usepackage{pmmeta}
\pmcanonicalname{HeapRemovalAlgorithm}
\pmcreated{2013-03-22 12:29:30}
\pmmodified{2013-03-22 12:29:30}
\pmowner{mps}{409}
\pmmodifier{mps}{409}
\pmtitle{heap removal algorithm}
\pmrecord{11}{32716}
\pmprivacy{1}
\pmauthor{mps}{409}
\pmtype{Algorithm}
\pmcomment{trigger rebuild}
\pmclassification{msc}{68P20}
\pmclassification{msc}{68P10}
\pmclassification{msc}{68P05}
\pmrelated{Heap}
\pmrelated{HeapInsertionAlgorithm}
\pmrelated{Heapsort}

\usepackage{amssymb}
\usepackage{amsmath}
\usepackage{amsfonts}
%\usepackage{lbh-pseudocode}
\usepackage{program} % ok
%\usepackage{pseudocode}
\begin{document}
\PMlinkescapeword{depth}

\section*{Definition}

Let $H$ be a heap storing $n$ elements, and let $\preceq$ be an operator that defines a total order over all of the values in $H$.
If $H$ is non-empty, then there is a value $x$ in $H$ such that $x\preceq y$ holds for all values in $H$.
The \emph{heap removal algorithm} removes $x$ from $H$ and returns it,
while maintaining the heap property of $H$.

The process of this algorithm is similar in nature to that of the heap insertion algorithm.  First, if $H$ only holds one value, then that value is $x$, which can simply be removed and returned.
Otherwise, let $z$ be the value stored in the right-most leaf of $H$.
Since $x$ is defined by the heap property to be the root of the tree,
the value of $x$ is saved, $z$ is removed, and the value at the root of the tree is set to $z$.  Then $z$ is \emph{sifted} downwards into the tree until the heap property is regained.

The sifting process is similar to that of the heap insertion algorithm, only in reverse.  First, if $z$ is a leaf, the process ends.  Otherwise, let $a, b$ be the two children of $z$, chosen such that $a \preceq b$ holds.  If $z \preceq a$ holds, the process ends.  Otherwise, $a$ and $z$ are swapped and the process repeats for $z$.

\section*{Analysis}

Since $H$ is a balanced binary tree, it has a maximum depth of $\lceil\log_2n\rceil+1$.  Since the maximum number of times the sift operation can occur is constrained by the depth of the tree, the worst-case time complexity for heap insertion is $\mathcal{O}(\log n)$.

\clearpage
\section*{Pseudocode}

What follows is the pseudocode for implementing heap removal.  For the given
pseudocode, we presume that the heap is actually represented implicitly in
an array (see the binary tree entry for details), and that the heap contains at least one value.

\begin{program}
\mathrm{HeapRemove}(H, n, \preceq)
\text{{\bf Input}.  A heap $(H,\preceq)$ (represented as an array) containing $n>0$ values.}
\text{{\bf Output}. Removes and returns a value $x$ from $H$ such that $x\preceq y$ holds for all $y$ in $H$.}
\text{{\bf Procedure}:}
top \gets H[0]
n \gets n - 1
H[0] \gets H[n]
parent \gets 0
child \gets 1
\WHILE child < n \DO
  \IF child+1 < n 
  \THEN
  \IF H[child+1]\preceq H[child] 
  \THEN
    child \gets child+1
  \FI 
  \FI
  \IF H[parent]\not\preceq H[child]
  \THEN \mathrm{swap}(H[parent],H[child])
        parent \gets child
        child \gets 2\cdot child + 1
  \ELSE child \gets n
  \FI
\OD
\end{program}

%%%%%
%%%%%
\end{document}
