\documentclass[12pt]{article}
\usepackage{pmmeta}
\pmcanonicalname{AmdahlsLaw}
\pmcreated{2013-03-22 12:04:15}
\pmmodified{2013-03-22 12:04:15}
\pmowner{alozano}{2414}
\pmmodifier{alozano}{2414}
\pmtitle{Amdahl's Law}
\pmrecord{17}{31133}
\pmprivacy{1}
\pmauthor{alozano}{2414}
\pmtype{Theorem}
\pmcomment{trigger rebuild}
\pmclassification{msc}{68M20}
%\pmkeywords{parallel computation}

\usepackage{amssymb}
\usepackage{amsmath}
\usepackage{amsfonts}
\usepackage{graphicx}
%%%\usepackage{xypic}
\begin{document}
Amdahl's Law reveals the maximum speedup that can be expected from parallel algorithms given the proportion of parts that must be computed sequentially.  It gives the speedup $S$ as

$$ S \le \frac{1}{f+(1-f)/N}  $$

Where $f$ is the fraction of the problem that must be computed sequentially and $N$ is the number of processors.

Note that as $f$ approaches zero, $S$ nears $N$, which we'd expect from a perfectly parallelizable algorithm.

% dummy change
%%%%%
%%%%%
%%%%%
\end{document}
