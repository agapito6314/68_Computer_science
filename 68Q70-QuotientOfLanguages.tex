\documentclass[12pt]{article}
\usepackage{pmmeta}
\pmcanonicalname{QuotientOfLanguages}
\pmcreated{2013-03-22 18:56:06}
\pmmodified{2013-03-22 18:56:06}
\pmowner{CWoo}{3771}
\pmmodifier{CWoo}{3771}
\pmtitle{quotient of languages}
\pmrecord{7}{41789}
\pmprivacy{1}
\pmauthor{CWoo}{3771}
\pmtype{Definition}
\pmcomment{trigger rebuild}
\pmclassification{msc}{68Q70}
\pmclassification{msc}{68Q45}
\pmdefines{quotient}
\pmdefines{left quotient}
\pmdefines{right quotient}

\usepackage{amssymb,amscd}
\usepackage{amsmath}
\usepackage{amsfonts}
\usepackage{mathrsfs}

% used for TeXing text within eps files
%\usepackage{psfrag}
% need this for including graphics (\includegraphics)
%\usepackage{graphicx}
% for neatly defining theorems and propositions
\usepackage{amsthm}
% making logically defined graphics
%%\usepackage{xypic}
\usepackage{pst-plot}

% define commands here
\newcommand*{\abs}[1]{\left\lvert #1\right\rvert}
\newtheorem{prop}{Proposition}
\newtheorem{thm}{Theorem}
\newtheorem{ex}{Example}
\newcommand{\real}{\mathbb{R}}
\newcommand{\pdiff}[2]{\frac{\partial #1}{\partial #2}}
\newcommand{\mpdiff}[3]{\frac{\partial^#1 #2}{\partial #3^#1}}
\begin{document}
Let $L_1,L_2$ be two languages over some alphabet $\Sigma$.  The \emph{quotient} of $L_1$ by $L_2$ is defined to be the set $$L_1/L_2:=\lbrace u \in \Sigma^* \mid uv \in L_1 \mbox{ for some } v\in L_2\rbrace.$$ $L_1/L_2$ is sometimes written $L_1 L_2^{-1}$.  The quotient so defined is also called the \emph{right quotient}, for one can similarly define the \emph{left quotient} of $L_1$ by $L_2$: $$L_1\backslash L_2:= \lbrace u\in \Sigma^* \mid vu \in L_1 \mbox{ for some } v\in L_2 \rbrace.$$  $L_1\backslash L_2$ is sometimes written $L_2^{-1} L_1$.

Below are some examples of quotients:
\begin{itemize}
\item If $L_1 = \lbrace a^n b^n c^n \mid n\ge 0\rbrace$ and $L_2 = \lbrace b,c\rbrace^*$, then 
\begin{itemize}
\item $L_1 / L_2 = \lbrace a^m b^n \mid m\ge n \ge 0 \rbrace$
\item $L_2 / L_1 = L_2$
\item $L_1 \backslash L_2 = \lbrace \lambda \rbrace$, the singleton consisting the empty word
\item $L_2 \backslash L_1 = L_2$
\end{itemize}
\item for any language $L$ over $\Sigma$:
\begin{itemize}
\item $L / \Sigma^*$ is the language of all prefixes of words of $L$
\item $\Sigma^* / L = \Sigma^*$
\item $L \backslash \Sigma^*$ is the language of all suffixes of words of $L$
\item $\Sigma^* \backslash L = \Sigma^*$
\end{itemize}
\item $\lambda \in L/L \cap L\backslash L$, and if $\lambda \in L$, then $L \subseteq L/L \cap L\backslash L$.
\end{itemize}

Here are some basic properties of quotients:
\begin{enumerate}
\item $L_1 \subseteq (L_1 / L_2)L_2 \cap L_2 (L_1 \backslash L_2)$.
\item $(L_1/L_2)L_2 \subseteq (L_1L_2)/L_2$, and $L_2 (L_1 \backslash L_2) \subseteq (L_2L_1) \backslash L_2$.
\item right and left quotients are related via reversal:
\begin{eqnarray*}
(L_1\backslash L_2)^R &=& \lbrace u^R \mid vu \in L_1 \mbox{ for some } v\in L_2 \rbrace \\ 
&=& \lbrace u^R \mid (vu)^R \in L_1^R \mbox{ for some } v^R \in L_2^R \rbrace \\ 
&=& \lbrace u^R \mid u^Rv^R \in L_1^R \mbox{ for some } v^R \in L_2^R \rbrace \\ 
&=& L_1^R / L_2^R.
\end{eqnarray*}
\end{enumerate}

A family $\mathscr{F}$ of languages is said to be \emph{closed under quotient by a language $L$} if for every language $M\in \mathscr{F}$, $M / L\in \mathscr{F}$.  Furthermore, $\mathscr{F}$ is said to be \emph{closed under quotient} if $M/L \in \mathscr{F}$ for any $M,L\in\mathscr{F}$.  Closure under quotient is also termed closure under right quotient.  Closure under left quotient is similarly defined.

It can be shown that the families of regular, context-free, and type-0 languages are closed under quotient (both left and right) by a regular language.  The family of context-sensitive languages does not have this closure property.  

Since all of the families mentioned above are closed under reversal, each of the families, except the context-sensitive family, is closed under left quotient by a regular language, according to the second property above.

\begin{thebibliography}{9}
\bibitem{hu} J.E. Hopcroft, J.D. Ullman, {\em Formal Languages and Their Relation to Automata}, Addison-Wesley, (1969).
\end{thebibliography}
%%%%%
%%%%%
\end{document}
