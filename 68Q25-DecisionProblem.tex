\documentclass[12pt]{article}
\usepackage{pmmeta}
\pmcanonicalname{DecisionProblem}
\pmcreated{2013-03-22 13:01:33}
\pmmodified{2013-03-22 13:01:33}
\pmowner{Henry}{455}
\pmmodifier{Henry}{455}
\pmtitle{decision problem}
\pmrecord{12}{33423}
\pmprivacy{1}
\pmauthor{Henry}{455}
\pmtype{Definition}
\pmcomment{trigger rebuild}
\pmclassification{msc}{68Q25}
\pmdefines{enumerates}
\pmdefines{decide}

\usepackage{amsmath}
\usepackage{amssymb}
\begin{document}
Let $T$ be a Turing machine and let $L\subseteq\Gamma^+$ be a language.  We say $T$ \emph{decides} $L$ if for any $x\in L$, $T$ accepts $x$, and for any $x\notin L$, $T$ rejects $x$.

We say $T$ \emph{enumerates} $L$ if:
$$x\in L \text{ iff } T \text{ accepts } x$$

For some Turing machines (for instance non-deterministic machines) these definitions are equivalent, but for others they are not.  For example, in order for a deterministic Turing machine $T$ to decide $L$, it must be that $T$ halts on every input.  On the other hand $T$ could enumerate $L$ if it does not halt on some strings which are not in $L$.

$L$ is sometimes said to be a \emph{decision problem}, and a Turing machine which decides it is said to solve the decision problem.

The set of strings which $T$ accepts is denoted $L(T)$.
%%%%%
%%%%%
\end{document}
