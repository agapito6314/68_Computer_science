\documentclass[12pt]{article}
\usepackage{pmmeta}
\pmcanonicalname{RangeProblem}
\pmcreated{2013-03-22 13:02:25}
\pmmodified{2013-03-22 13:02:25}
\pmowner{Henry}{455}
\pmmodifier{Henry}{455}
\pmtitle{range problem}
\pmrecord{5}{33442}
\pmprivacy{1}
\pmauthor{Henry}{455}
\pmtype{Definition}
\pmcomment{trigger rebuild}
\pmclassification{msc}{68Q25}
\pmdefines{strong range problem}
\pmdefines{strong range}

% this is the default PlanetMath preamble.  as your knowledge
% of TeX increases, you will probably want to edit this, but
% it should be fine as is for beginners.

% almost certainly you want these
\usepackage{amssymb}
\usepackage{amsmath}
\usepackage{amsfonts}

% used for TeXing text within eps files
%\usepackage{psfrag}
% need this for including graphics (\includegraphics)
%\usepackage{graphicx}
% for neatly defining theorems and propositions
%\usepackage{amsthm}
% making logically defined graphics
%%%\usepackage{xypic}

% there are many more packages, add them here as you need them

% define commands here
%\PMlinkescapeword{theory}
\begin{document}
A \emph{range problem} is a weakened form of a search problem.  It consists of two functions $f_l$ and $f_u$ (the lower and upper bounds) and a linear ordering $<$ on the ranges of $f_1$ and $f_2$.  A Turing machine solves a range problem if, for any $x$, the machine eventually halts with an output $y$ such that $f_1(x)<y<f_2(x)$.

For example, given any function $f$ with range in $\mathbb{R}$ and any $g:\mathbb{N}\rightarrow\mathbb{R}$, the \emph{strong range problem} $\operatorname{StrongRange}_g(f)$ is given by lower bound $f(x)\cdot(1-\frac{1}{1-g(|x|)})$ and upper bound $f(x)\cdot(1-\frac{1}{1+g(|x|)})$ (note that $g$ is passed the length of $x$, not the value, which need not even be a number).
%%%%%
%%%%%
\end{document}
