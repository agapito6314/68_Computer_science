\documentclass[12pt]{article}
\usepackage{pmmeta}
\pmcanonicalname{SearchProblem}
\pmcreated{2013-03-22 13:01:39}
\pmmodified{2013-03-22 13:01:39}
\pmowner{Henry}{455}
\pmmodifier{Henry}{455}
\pmtitle{search problem}
\pmrecord{7}{33425}
\pmprivacy{1}
\pmauthor{Henry}{455}
\pmtype{Definition}
\pmcomment{trigger rebuild}
\pmclassification{msc}{68Q25}
\pmdefines{calculate}

% this is the default PlanetMath preamble.  as your knowledge
% of TeX increases, you will probably want to edit this, but
% it should be fine as is for beginners.

% almost certainly you want these
\usepackage{amssymb}
\usepackage{amsmath}
\usepackage{amsfonts}

% used for TeXing text within eps files
%\usepackage{psfrag}
% need this for including graphics (\includegraphics)
%\usepackage{graphicx}
% for neatly defining theorems and propositions
%\usepackage{amsthm}
% making logically defined graphics
%%%\usepackage{xypic}

% there are many more packages, add them here as you need them

% define commands here
%\PMlinkescapeword{theory}
\begin{document}
If $R$ is a binary relation such that $\operatorname{field}(R)\subseteq\Gamma^+$ and $T$ is a Turing machine, then $T$ \emph{calculates} $f$ if:

\begin{itemize}
\item If $x$ is such that there is some $y$ such that $R(x,y)$ then $T$ accepts $x$ with output $z$ such that $R(x,z)$ (there may be multiple $y$, and $T$ need only find one of them)

\item If $x$ is such that there is no $y$ such that $R(x,y)$ then $T$ rejects $x$
\end{itemize}

Note that the \PMlinkescapetext{graph} of a partial function is a binary relation, and if $T$ calculates a partial function then there is at most one possible output.

A \PMlinkescapetext{relation} $R$ can be viewed as a \emph{search problem}, and a Turing machine which calculates $R$ is also said to solve it.  Every search problem has a corresponding decision problem, namely $L(R)=\{x\mid \exists y R(x,y)\}$.

This definition may be generalized to $n$-ary relations using any suitable encoding which allows multiple strings to be compressed into one string (for instance by listing them consecutively with a delimiter).
%%%%%
%%%%%
\end{document}
