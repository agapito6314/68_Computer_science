\documentclass[12pt]{article}
\usepackage{pmmeta}
\pmcanonicalname{LeftmostDerivation}
\pmcreated{2013-03-22 18:54:50}
\pmmodified{2013-03-22 18:54:50}
\pmowner{CWoo}{3771}
\pmmodifier{CWoo}{3771}
\pmtitle{leftmost derivation}
\pmrecord{7}{41763}
\pmprivacy{1}
\pmauthor{CWoo}{3771}
\pmtype{Definition}
\pmcomment{trigger rebuild}
\pmclassification{msc}{68Q45}
\pmclassification{msc}{68Q42}
\pmdefines{rightmost derivation}

\usepackage{amssymb,amscd}
\usepackage{amsmath}
\usepackage{amsfonts}
\usepackage{mathrsfs}

% used for TeXing text within eps files
%\usepackage{psfrag}
% need this for including graphics (\includegraphics)
%\usepackage{graphicx}
% for neatly defining theorems and propositions
\usepackage{amsthm}
% making logically defined graphics
%%\usepackage{xypic}
\usepackage{pst-plot}

% define commands here
\newcommand*{\abs}[1]{\left\lvert #1\right\rvert}
\newtheorem{prop}{Proposition}
\newtheorem{thm}{Theorem}
\newtheorem{ex}{Example}
\newcommand{\real}{\mathbb{R}}
\newcommand{\pdiff}[2]{\frac{\partial #1}{\partial #2}}
\newcommand{\mpdiff}[3]{\frac{\partial^#1 #2}{\partial #3^#1}}
\begin{document}
Let $G=(\Sigma, N, P, \sigma)$ be a context-free grammar.  Recall that a word $v$ is generated by $G$ if it is
\begin{itemize}
\item a word over the set $T:=\Sigma - N$ of terminals, and
\item derivable from the starting symbol $\sigma$.
\end{itemize}
The second condition simply says that there is a finite sequence of derivation steps, starting from $\sigma$, and ending at $v$:
$$\sigma = v_0 \to v_1 \to \cdots \to v_n = v$$
For each derivation step $v_i\to v_{i+1}$, there is a production $X\to w$ in $P$ such that
\begin{eqnarray*}
v_i &=& aXb, \\
v_{i+1} &=& awb.
\end{eqnarray*}
Note that $X$ is a non-terminal (an element of $N$), and $a,b,w$ are words over $\Sigma$.

\textbf{Definition}.  Using the notations above, if $X$ is the leftmost non-terminal occurring in $v_i$, then we say the derivation step $v_i\to v_{i+1}$ is \emph{leftmost}.  Dually, $v_i\to v_{i+1}$ is \emph{rightmost} if $X$ is the rightmost non-terminal occurring in $v_i$.

Equivalently, $v_i\to v_{i+1}$ is leftmost (or rightmost) if $a$ (or $b$) is a word over $T$.

A derivation is said to be \emph{leftmost} (or \emph{rightmost}) if each of its derivation steps is leftmost (or rightmost).

\textbf{Example}.  Let $G$ be the grammar consisting of $a,b$ as terminals, $\sigma,X,Y,Z$ as non-terminals (with $\sigma$ as the starting symbol), and $\sigma \to XY$, $X\to a$, $Y\to b$, $\sigma \to ZY$, and $Z\to X\sigma$ as productions.  $G$ is clearly context-free.

The word $a^2b^2=aabb$ can be generated by $G$ by the following three derivations:
\begin{enumerate}
\item $\sigma \to ZY \to X\sigma Y \to XXYY \to XaYY \to XaYb \to Xabb \to aabb$,
\item $\sigma \to ZY \to X\sigma Y \to a \sigma Y \to a XYY \to aaYY \to aabY \to aabb$,
\item $\sigma \to ZY \to Zb \to X\sigma b \to XXY b\to XXbb \to Xabb \to aabb$.
\end{enumerate}
The second is leftmost, the third is rightmost, and the first is neither.

Note that in every derivation, the first and last derivation steps are always leftmost and rightmost.

\textbf{Remarks}.  
\begin{itemize}
\item
One of the main properties of a context-free grammar $G$ is that every word generated by $G$ is derivable by a leftmost (correspondingly rightmost) derivation, which can be used to show that for every context-free language $L$, there is a pushdown automaton accepting every word in $L$, and conversely that the set of words accepted by a pushdown automaton is context-free.
\item
Leftmost derivations may be defined for any arbitrary formal grammar $G$ satisfying the condition
\begin{quote}
(*) no terminal symbols occur on the left side of any production.  
\end{quote}
Given two words $u,v$, we define $u\Rightarrow_L v$ if there is a production $A\to B$ such that $u=xAy$ and $v=xBy$ such that $x$ is a terminal word.  By taking the reflexive transitive closure of $\Rightarrow_L$, we have the leftmost derivation $\Rightarrow_L^*$.  It can be shown that for any grammar $G$ satisfying condition (*), the language $L_L(G)$ consisting of terminal words generated by $G$ via leftmost derivations is always context-free.
\end{itemize}

\begin{thebibliography}{9}
\bibitem{sg} S. Ginsburg {\em The Mathematical Theory of Context-Free Languages}, McGraw-Hill, New York (1966).
\bibitem{dk} D. C. Kozen {\em Automata and Computability}, Springer, New York (1997).
\end{thebibliography}
%%%%%
%%%%%
\end{document}
