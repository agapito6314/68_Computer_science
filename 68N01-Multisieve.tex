\documentclass[12pt]{article}
\usepackage{pmmeta}
\pmcanonicalname{Multisieve}
\pmcreated{2013-03-22 16:13:33}
\pmmodified{2013-03-22 16:13:33}
\pmowner{Mravinci}{12996}
\pmmodifier{Mravinci}{12996}
\pmtitle{multisieve}
\pmrecord{5}{38324}
\pmprivacy{1}
\pmauthor{Mravinci}{12996}
\pmtype{Definition}
\pmcomment{trigger rebuild}
\pmclassification{msc}{68N01}

% this is the default PlanetMath preamble.  as your knowledge
% of TeX increases, you will probably want to edit this, but
% it should be fine as is for beginners.

% almost certainly you want these
\usepackage{amssymb}
\usepackage{amsmath}
\usepackage{amsfonts}

% used for TeXing text within eps files
%\usepackage{psfrag}
% need this for including graphics (\includegraphics)
%\usepackage{graphicx}
% for neatly defining theorems and propositions
%\usepackage{amsthm}
% making logically defined graphics
%%%\usepackage{xypic}

% there are many more packages, add them here as you need them

% define commands here

\begin{document}
\emph{MultiSieve} is a computer program written by Mark Rodenkirch and  designed for primality testing. It uses "both trial division and discrete logs as a sieve to eliminate numbers faster," according to Rodenkirch. It is available for Windows on the Intel x86 platform, and Rodenkirch has worked on a port for Mac OS X. According to the prime pages of the University of Tennessee at Martin, the largest prime MultiSieve has found so far is $338707 \times 2^{1354830} + 1$, approximately $9.97354620220965 \times 10^{407849}$.
%%%%%
%%%%%
\end{document}
