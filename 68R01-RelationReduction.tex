\documentclass[12pt]{article}
\usepackage{pmmeta}
\pmcanonicalname{RelationReduction}
\pmcreated{2013-10-24 16:56:16}
\pmmodified{2013-10-24 16:56:16}
\pmowner{Jon Awbrey}{15246}
\pmmodifier{Jon Awbrey}{15246}
\pmtitle{relation reduction}
\pmrecord{36}{40290}
\pmprivacy{1}
\pmauthor{Jon Awbrey}{15246}
\pmtype{Definition}
\pmcomment{trigger rebuild}
\pmclassification{msc}{68R01}
\pmclassification{msc}{68P15}
\pmclassification{msc}{08A02}
\pmclassification{msc}{05C65}
\pmclassification{msc}{05B30}
\pmclassification{msc}{05B20}
\pmclassification{msc}{03B10}
\pmclassification{msc}{03E20}
\pmsynonym{relation identification}{RelationReduction}
\pmsynonym{relation reconstruction}{RelationReduction}
\pmrelated{RelationTheory}
\pmrelated{RelationComposition2}
\pmrelated{RelationConstruction}
\pmrelated{TacitExtension}
\pmdefines{relational reducibility}
\pmdefines{relational irreducibility}
\pmdefines{reductand}
\pmdefines{reduction base}
\pmdefines{reduction method}

\endmetadata

% this is the default PlanetMath preamble.  as your knowledge
% of TeX increases, you will probably want to edit this, but
% it should be fine as is for beginners.

% almost certainly you want these
\usepackage{amssymb}
\usepackage{amsmath}
\usepackage{amsfonts}

% used for TeXing text within eps files
%\usepackage{psfrag}
% need this for including graphics (\includegraphics)
%\usepackage{graphicx}
% for neatly defining theorems and propositions
%\usepackage{amsthm}
% making logically defined graphics
%%%\usepackage{xypic}

% there are many more packages, add them here as you need them

% define commands here

\begin{document}
\textbf{Relation reduction} and \textbf{relational reducibility} have to do with the extent to which a given relation is determined by an indexed family or a sequence of other relations, called the \textit{relation dataset}.  The relation under examination is called the \textit{reductand}.  The relation dataset typically consists of a specified set of other relations, simpler in some measure than the reductand, called the \textit{reduction base}, plus a specified operation on relations, called the \textit{reduction method} or the \textit{reduction step}.

A question of relation reduction or relational reducibility is sometimes posed as a question of \textbf{relation reconstruction} or \textbf{relational reconstructibility}, since a useful way of stating the question is to ask whether the reductand can be reconstructed from the relational dataset.

A relation that is not uniquely determined by a particular relation dataset is said to be irreducible in just that respect.  A relation that is not uniquely determined by any relation dataset in a particular class of relation datasets is said to be irreducible in respect of that class.

\tableofcontents

\section{Discussion}

The main thing that keeps the general problem of relational reducibility from being fully well-defined is that one would have to survey all of the conceivable ways of ``getting new relations from old" in order to say precisely what is meant by the claim that the relation $L$ is reducible to the set of relations $\{ L_j : j \in J \}$.  This is tantamount to claiming that if one is given a set of ``simpler" relations $L_j$, for indices $j$ in some set $J$, that this collection of data would somehow or other fix the original relation $L$ that one is seeking to analyze, to determine, to specify, or to synthesize.

In practice, however, apposite discussion of a particular application typically settles on either one of two different notions of reducibility as capturing the pertinent issues, namely:

\begin{itemize}
\item
Reduction under \PMlinkname{composition}{RelationComposition}.
\item
Reduction under projections.
\end{itemize}

As it happens, there is an interesting relationship between these two notions of reducibility, the implications of which may be taken up partly in parallel with the discussion of the basic concepts.

\section{Projective reducibility of relations}

It is convenient to begin with the \textit{projective reduction} of relations, partly because this type of reduction is simpler and more intuitive (in the visual sense), but also because a number of conceptual tools that are needed in any case arise quite naturally in the projective setting.

The work of intuiting how projections operate on multidimensional relations is often facilitated by keeping in mind the following sort of geometric image:

\begin{itemize}
\item
Picture a $k$-adic relation $L$ as a body that resides in a $k$-dimensional space $X$.  If the domains of the relation $L$ are $X_1, \ldots, X_k$, then the \textit{extension} of the relation $L$ is a subset of the cartesian product $X = X_1 \times \ldots \times X_k$.
\end{itemize}

In this setting, the interval $K = [1, k] = \{ 1, \ldots, k \}$ is called the \textit{index set} of the \textit{indexed family} of sets $X_1, \ldots, X_k$.

For any subset $F$ of the index set $K$, there is the corresponding subfamily of sets, $\{ X_j : j \in F \}$, and there is the corresponding cartesian product over this subfamily, notated and defined as $X_F = \prod_{j \in F} X_j$.

For any point $x$ in $X$, the \textit{projection} of $x$ on the subspace $X_F$ is notated as $\operatorname{proj}_{F}(x)$, or still more simply as $\operatorname{proj}_{F}x.$

More generally, for any relation $L \subseteq X$, the projection of $L$ on the subspace $X_F$ is written as $\operatorname{proj}_{F}(L)$, or still more simply as $\operatorname{proj}_{F}L.$

The question of \textit{projective reduction} for $k$-adic relations can be stated with moderate generality in the following way:

\begin{itemize}
\item
Given a set of $k$-place relations in the same space $X$ and a set of projections from $X$ to the associated subspaces, do the projections afford sufficient data to tell the different relations apart?
\end{itemize}

\section{Projective reducibility of triadic relations}

See \PMlinkname{main entry}{TriadicRelation} on triadic relations.

By way of illustrating the different sorts of things that can occur in considering the projective reducibility of relations, it is convenient to reuse the four examples of 3-adic relations that are discussed in the main entry on that subject.

\subsection{Examples of projectively irreducible relations}

The 3-adic relations $L_0$ and $L_1$ are shown in the next two Tables:

\begin{quote}\begin{tabular}{|c|c|c|}
\multicolumn{3}{c}{$L_0 = \{ (x, y, z) \in \mathbb{B}^3 : x + y + z = 0 \}$} \\
\hline\
$X$ & $Y$ & $Z$ \\
\hline\hline
 0  &  0  &  0  \\
\hline
 0  &  1  &  1  \\
\hline
 1  &  0  &  1  \\
\hline
 1  &  1  &  0  \\
\hline
\end{tabular}\end{quote}

\begin{quote}\begin{tabular}{|c|c|c|}
\multicolumn{3}{c}{$L_1 = \{ (x, y, z) \in \mathbb{B}^3 : x + y + z = 1 \}$} \\
\hline
$X$ & $Y$ & $Z$ \\
\hline\hline
 0  &  0  &  1  \\
\hline
 0  &  1  &  0  \\
\hline
 1  &  0  &  0  \\
\hline
 1  &  1  &  1  \\
\hline
\end{tabular}\end{quote}

Viewed in terms of operations on relational data tables, a 2-adic projection of a 3-adic relation $L$ is the 2-adic relation that results from deleting one column of the table for $L$ and then deleting all but one row of any resulting rows that happen to be identical in content.  In other words, the multiplicity of any repeated row is ignored.

In the case of the above two relations, $L_0, L_1 \subseteq X \times Y \times Z \cong \mathbb{B}^3$, the 2-adic projections are indexed by the columns or domains that remain, as shown in the following Tables.

\begin{quote}\begin{tabular}{|c|c|p{2cm}|c|c|p{2cm}|c|c|}
\multicolumn{2}{c}{$\operatorname{proj}_{XY}L_0$} &
\multicolumn{1}{c}{ } &
\multicolumn{2}{c}{$\operatorname{proj}_{XZ}L_0$} &
\multicolumn{1}{c}{ } &
\multicolumn{2}{c}{$\operatorname{proj}_{YZ}L_0$} \\
\cline{1-2}\cline{4-5}\cline{7-8}
$X$ & $Y$ & & $X$ & $Z$ & & $Y$ & $Z$ \\
\cline{1-2}\cline{4-5}\cline{7-8}
 0  &  0  & &  0  &  0  & &  0  &  0  \\
\cline{1-2}\cline{4-5}\cline{7-8}
 0  &  1  & &  0  &  1  & &  1  &  1  \\
\cline{1-2}\cline{4-5}\cline{7-8}
 1  &  0  & &  1  &  1  & &  0  &  1  \\
\cline{1-2}\cline{4-5}\cline{7-8}
 1  &  1  & &  1  &  0  & &  1  &  0  \\
\cline{1-2}\cline{4-5}\cline{7-8}
\end{tabular}\end{quote}

\begin{quote}\begin{tabular}{|c|c|p{2cm}|c|c|p{2cm}|c|c|}
\multicolumn{2}{c}{$\operatorname{proj}_{XY}L_1$} &
\multicolumn{1}{c}{ } &
\multicolumn{2}{c}{$\operatorname{proj}_{XZ}L_1$} &
\multicolumn{1}{c}{ } &
\multicolumn{2}{c}{$\operatorname{proj}_{YZ}L_1$} \\
\cline{1-2}\cline{4-5}\cline{7-8}
$X$ & $Y$ & & $X$ & $Z$ & & $Y$ & $Z$ \\
\cline{1-2}\cline{4-5}\cline{7-8}
 0  &  0  & &  0  &  1  & &  0  &  1  \\
\cline{1-2}\cline{4-5}\cline{7-8}
 0  &  1  & &  0  &  0  & &  1  &  0  \\
\cline{1-2}\cline{4-5}\cline{7-8}
 1  &  0  & &  1  &  0  & &  0  &  0  \\
\cline{1-2}\cline{4-5}\cline{7-8}
 1  &  1  & &  1  &  1  & &  1  &  1  \\
\cline{1-2}\cline{4-5}\cline{7-8}
\end{tabular}\end{quote}

It is clear on inspection that the following three equations hold:

\begin{quote}$\begin{array}{lcr}
\operatorname{proj}_{XY}L_0 = \operatorname{proj}_{XY}L_1, &
\operatorname{proj}_{XZ}L_0 = \operatorname{proj}_{XZ}L_1, &
\operatorname{proj}_{YZ}L_0 = \operatorname{proj}_{YZ}L_1. \\
\end{array}$\end{quote}

These equations say that $L_0$ and $L_1$ cannot be distinguished from each other solely on the basis of their 2-adic projection data.  In such a case, either relation is said to be \textit{irreducible with respect to 2-adic projections}.  Since reducibility with respect to 2-adic projections is the only interesting case where it concerns the reduction of 3-adic relations, it is customary to say more simply of such a relation that it is \textit{projectively irreducible}, the 2-adic basis being understood.  It is immediate from the definition that projectively irreducible relations always arise in non-trivial multiplets of mutually indiscernible relations.

\subsection{Examples of projectively reducible relations}

The 3-adic sign relations $L_{\mathrm{A}}$ and $L_{\mathrm{B}}$ are shown in the next two Tables:

\begin{quote}\begin{tabular}{|c|c|c|}
\multicolumn{3}{c}{$L_{\mathrm{A}}$ = Sign Relation of Interpreter A} \\
\hline\
Object       & Sign            & Interpretant     \\
\hline\hline
$\mathrm{A}$ & $\mathrm{``A"}$ & $\mathrm{``A"}$  \\
\hline
$\mathrm{A}$ & $\mathrm{``A"}$ & $\mathrm{``i"}$  \\
\hline
$\mathrm{A}$ & $\mathrm{``i"}$ & $\mathrm{``A"}$  \\
\hline
$\mathrm{A}$ & $\mathrm{``i"}$ & $\mathrm{``i"}$  \\
\hline
$\mathrm{B}$ & $\mathrm{``B"}$ & $\mathrm{``B"}$  \\
\hline
$\mathrm{B}$ & $\mathrm{``B"}$ & $\mathrm{``u"}$  \\
\hline
$\mathrm{B}$ & $\mathrm{``u"}$ & $\mathrm{``B"}$  \\
\hline
$\mathrm{B}$ & $\mathrm{``u"}$ & $\mathrm{``u"}$  \\
\hline
\end{tabular}\end{quote}

\begin{quote}\begin{tabular}{|c|c|c|}
\multicolumn{3}{c}{$L_{\mathrm{B}}$ = Sign Relation of Interpreter B} \\
\hline\
Object       & Sign            & Interpretant     \\
\hline\hline
$\mathrm{A}$ & $\mathrm{``A"}$ & $\mathrm{``A"}$  \\
\hline
$\mathrm{A}$ & $\mathrm{``A"}$ & $\mathrm{``u"}$  \\
\hline
$\mathrm{A}$ & $\mathrm{``u"}$ & $\mathrm{``A"}$  \\
\hline
$\mathrm{A}$ & $\mathrm{``u"}$ & $\mathrm{``u"}$  \\
\hline
$\mathrm{B}$ & $\mathrm{``B"}$ & $\mathrm{``B"}$  \\
\hline
$\mathrm{B}$ & $\mathrm{``B"}$ & $\mathrm{``i"}$  \\
\hline
$\mathrm{B}$ & $\mathrm{``i"}$ & $\mathrm{``B"}$  \\
\hline
$\mathrm{B}$ & $\mathrm{``i"}$ & $\mathrm{``i"}$  \\
\hline
\end{tabular}\end{quote}

In the case of the two sign relations, $L_{\mathrm{A}}, L_{\mathrm{B}} \subseteq X \times Y \times Z \cong O \times S \times I$, the 2-adic projections are indexed by the columns or domains that remain, as shown in the following Tables.

\begin{quote}\begin{tabular}{|c|c|p{1cm}|c|c|p{1cm}|c|c|}
\multicolumn{2}{c}{$\operatorname{proj}_{OS}L_{\mathrm{A}}$} &
\multicolumn{1}{c}{ } &
\multicolumn{2}{c}{$\operatorname{proj}_{OI}L_{\mathrm{A}}$} &
\multicolumn{1}{c}{ } &
\multicolumn{2}{c}{$\operatorname{proj}_{SI}L_{\mathrm{A}}$} \\
\cline{1-2}\cline{4-5}\cline{7-8}
Object & Sign & & Object & Interpretant & & Sign & Interpretant \\
\cline{1-2}\cline{4-5}\cline{7-8}
$\mathrm{A}$ & $\mathrm{``A"}$ & &
$\mathrm{A}$ & $\mathrm{``A"}$ & &
$\mathrm{``A"}$ & $\mathrm{``A"}$ \\
\cline{1-2}\cline{4-5}\cline{7-8}
$\mathrm{A}$ & $\mathrm{``i"}$ & &
$\mathrm{A}$ & $\mathrm{``i"}$ & &
$\mathrm{``A"}$ & $\mathrm{``i"}$ \\
\cline{1-2}\cline{4-5}\cline{7-8}
$\mathrm{B}$ & $\mathrm{``B"}$ & &
$\mathrm{B}$ & $\mathrm{``B"}$ & &
$\mathrm{``i"}$ & $\mathrm{``A"}$ \\
\cline{1-2}\cline{4-5}\cline{7-8}
$\mathrm{B}$ & $\mathrm{``u"}$ & &
$\mathrm{B}$ & $\mathrm{``u"}$ & &
$\mathrm{``i"}$ & $\mathrm{``i"}$ \\
\cline{1-2}\cline{4-5}\cline{7-8}
\multicolumn{6}{c|}{~} & $\mathrm{``B"}$ & $\mathrm{``B"}$ \\
\cline{7-8}
\multicolumn{6}{c|}{~} & $\mathrm{``B"}$ & $\mathrm{``u"}$ \\
\cline{7-8}
\multicolumn{6}{c|}{~} & $\mathrm{``u"}$ & $\mathrm{``B"}$ \\
\cline{7-8}
\multicolumn{6}{c|}{~} & $\mathrm{``u"}$ & $\mathrm{``u"}$ \\
\cline{7-8}
\end{tabular}\end{quote}

\begin{quote}\begin{tabular}{|c|c|p{1cm}|c|c|p{1cm}|c|c|}
\multicolumn{2}{c}{$\operatorname{proj}_{OS}L_{\mathrm{B}}$} &
\multicolumn{1}{c}{ } &
\multicolumn{2}{c}{$\operatorname{proj}_{OI}L_{\mathrm{B}}$} &
\multicolumn{1}{c}{ } &
\multicolumn{2}{c}{$\operatorname{proj}_{SI}L_{\mathrm{B}}$} \\
\cline{1-2}\cline{4-5}\cline{7-8}
Object & Sign & & Object & Interpretant & & Sign & Interpretant \\
\cline{1-2}\cline{4-5}\cline{7-8}
$\mathrm{A}$ & $\mathrm{``A"}$ & &
$\mathrm{A}$ & $\mathrm{``A"}$ & &
$\mathrm{``A"}$ & $\mathrm{``A"}$ \\
\cline{1-2}\cline{4-5}\cline{7-8}
$\mathrm{A}$ & $\mathrm{``u"}$ & &
$\mathrm{A}$ & $\mathrm{``u"}$ & &
$\mathrm{``A"}$ & $\mathrm{``u"}$ \\
\cline{1-2}\cline{4-5}\cline{7-8}
$\mathrm{B}$ & $\mathrm{``B"}$ & &
$\mathrm{B}$ & $\mathrm{``B"}$ & &
$\mathrm{``u"}$ & $\mathrm{``A"}$ \\
\cline{1-2}\cline{4-5}\cline{7-8}
$\mathrm{B}$ & $\mathrm{``i"}$ & &
$\mathrm{B}$ & $\mathrm{``i"}$ & &
$\mathrm{``u"}$ & $\mathrm{``u"}$ \\
\cline{1-2}\cline{4-5}\cline{7-8}
\multicolumn{6}{c|}{~} & $\mathrm{``B"}$ & $\mathrm{``B"}$ \\
\cline{7-8}
\multicolumn{6}{c|}{~} & $\mathrm{``B"}$ & $\mathrm{``i"}$ \\
\cline{7-8}
\multicolumn{6}{c|}{~} & $\mathrm{``i"}$ & $\mathrm{``B"}$ \\
\cline{7-8}
\multicolumn{6}{c|}{~} & $\mathrm{``i"}$ & $\mathrm{``i"}$ \\
\cline{7-8}
\end{tabular}\end{quote}

It is clear on inspection that the following three inequations hold:

\begin{quote}$\begin{array}{lcr}
\operatorname{proj}_{OS}L_{\mathrm{A}} \ne \operatorname{proj}_{OS}L_{\mathrm{B}}, &
\operatorname{proj}_{OI}L_{\mathrm{A}} \ne \operatorname{proj}_{OI}L_{\mathrm{B}}, &
\operatorname{proj}_{SI}L_{\mathrm{A}} \ne \operatorname{proj}_{SI}L_{\mathrm{B}}. \\
\end{array}$\end{quote}

These inequations say that $L_{\mathrm{A}}$ and $L_{\mathrm{B}}$ can be distinguished from each other solely on the basis of their 2-adic projection data.  But this is not enough to say that either one of them is projectively reducible to their 2-adic projection data.  To say that a 3-adic relation is projectively reducible in that respect, one has to show that it can be distinguished from \textit{every} other 3-adic relation on the basis of the 2-adic projection data alone.

In other words, to show that a 3-adic relation $L$ on $O \times S \times I$ is \textit{reducible} or \textit{reconstructible} in the 2-adic projective sense, it is necessary to show that no distinct $L'$ on $O \times S \times I$ exists such that $L$ and $L'$ have the same set of projections.  Proving this takes a much more comprehensive or exhaustive investigation of the space of possible relations on $O \times S \times I$ than looking merely at one or two relations at a time.

\textbf{Fact.}  As it happens, each of the relations $L_{\mathrm{A}}$ and $L_{\mathrm{B}}$ is uniquely determined by its 2-adic projections.  This can be seen by following the proof that is given below.

Before tackling the proof, however, it will speed things along to recall a few ideas and notations from other articles.

\begin{itemize}
\item
If $L$ is a relation over a set of domains that includes the domains $U$ and $V$, then the abbreviated notation $L_{UV}$ can be used for the projection $\operatorname{proj}_{UV}L$.
\item
The operation of reversing a projection asks what elements of a bigger space project onto given elements of a smaller space.  The set of elements that project onto $x$ under a given projection $f$ is called the \textit{fiber} of $x$ under $f$ and is written $f^{-1}(x)$ or $f^{-1}x$.
\item
If $X$ is a finite set, the \textit{cardinality} of $X$, written $\operatorname{card}(X)$ or $|X|$, means the number of elements in $X$.
\end{itemize}

\textbf{Proof.}  Let $L$ be either one of the relations $L_{\mathrm{A}}$ or $L_{\mathrm{B}}$.  Consider any coordinate position $(s, i)$ in the $SI$-plane $S \times I$.  If $(s, i)$ is not in $L_{SI}$ then there can be no element $(o, s, i)$ in $L$, therefore we may restrict our attention to positions $(s, i)$ in $L_{SI}$, knowing that there exist at least $|L_{SI}| = 8$ elements in $L$, and seeking only to determine what objects $o$ exist such that $(o, s, i)$ is an element in the fiber of $(s, i)$.  In other words, for what $o$ in $O$ is $(o, s, i)$ in the fiber $\operatorname{proj}_{SI}^{-1}(s, i)$?  Now, the circumstance that $L_{OS}$ has exactly one element $(o, s)$ for each coordinate $s$ in $S$ and that $L_{OI}$ has exactly one element $(o, i)$ for each coordinate $i$ in $I$, plus the ``coincidence" of it being the same $o$ at any one choice for $(s, i)$, tells us that $L$ has just the one element $(o, s, i)$ over each point of $S \times I$.  All together, this proves that both $L_{\mathrm{A}}$ and $L_{\mathrm{B}}$ are reducible in an informative sense to 3-tuples of 2-adic relations, that is, they are \textit{projectively 2-adically reducible}.

\subsection{Summary}

The \textit{projective analysis} of 3-adic relations, illustrated by means of concrete examples, has been pursued just far enough at this point to state this clearly demonstrated result:

\begin{itemize}
\item
Some 3-adic relations are, and other 3-adic relations are not, reducible to, or reconstructible from, their 2-adic projection data.  In short, some 3-adic relations are projectively reducible and some 3-adic relations are projectively irreducible.
\end{itemize}

%%%%%
%%%%%
\end{document}
