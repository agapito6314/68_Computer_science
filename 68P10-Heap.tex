\documentclass[12pt]{article}
\usepackage{pmmeta}
\pmcanonicalname{Heap}
\pmcreated{2013-03-22 12:29:14}
\pmmodified{2013-03-22 12:29:14}
\pmowner{mps}{409}
\pmmodifier{mps}{409}
\pmtitle{heap}
\pmrecord{6}{32707}
\pmprivacy{1}
\pmauthor{mps}{409}
\pmtype{Data Structure}
\pmcomment{trigger rebuild}
\pmclassification{msc}{68P10}
\pmclassification{msc}{68P20}
\pmclassification{msc}{68P05}
\pmrelated{BinaryTree}
\pmrelated{BalancedTree}
\pmrelated{HeapInsertionAlgorithm}
\pmrelated{HeapRemovalAlgorithm}
\pmrelated{Heapsort}
\pmdefines{heap property}

\endmetadata

\usepackage{amssymb}
\usepackage{amsmath}
\usepackage{amsfonts}
\usepackage[all]{xy}
\begin{document}
Let $\preceq$ be a total order on some set $A$.  A \emph{heap} is then a data structure for storing elements in $A$.  A heap is a balanced binary tree, with the property that if $y$ is a descendent of $x$ in the heap, then $x\preceq y$ must hold.  This property is often referred to as the \emph{heap property}.

If $\preceq$ is $\leq$, then the root of the heap always gives the smallest element of the heap, and if $\preceq$ is $\geq$, then the root of the heap always gives the largest element of the heap.  More generally, the root of the heap is some $a\in A$ such that $a\preceq x$ holds for all $x$ in the heap.

For example, the following heap represents the multiset $\left\{ 1, 2, 4, 4, 6, 8 \right\}$ for the total order $\geq$ on $\mathbb{Z}$.

$$
\xymatrix{
&&& 8 \ar@{-}[dll] \ar@{-}[drr] \\
& 4 \ar@{-}[dl] \ar@{-}[dr] &&&& 6 \ar@{-}[dl] \\
1 && 4 && 2
}
$$

Due to the heap property, heaps have a very elegant application to the sorting problem.  The heapsort is an in-place sorting algorithm centered entirely around a heap.  Heaps are also used to implement priority queues.
%%%%%
%%%%%
\end{document}
