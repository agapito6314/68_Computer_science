\documentclass[12pt]{article}
\usepackage{pmmeta}
\pmcanonicalname{ChomskySchutzenbergerTheorem}
\pmcreated{2013-03-22 18:55:40}
\pmmodified{2013-03-22 18:55:40}
\pmowner{CWoo}{3771}
\pmmodifier{CWoo}{3771}
\pmtitle{Chomsky-Sch\"utzenberger theorem}
\pmrecord{5}{41780}
\pmprivacy{1}
\pmauthor{CWoo}{3771}
\pmtype{Theorem}
\pmcomment{trigger rebuild}
\pmclassification{msc}{68Q42}
\pmclassification{msc}{68Q45}

\endmetadata

\usepackage{amssymb,amscd}
\usepackage{amsmath}
\usepackage{amsfonts}
\usepackage{mathrsfs}

% used for TeXing text within eps files
%\usepackage{psfrag}
% need this for including graphics (\includegraphics)
%\usepackage{graphicx}
% for neatly defining theorems and propositions
\usepackage{amsthm}
% making logically defined graphics
%%\usepackage{xypic}
\usepackage{pst-plot}

% define commands here
\newcommand*{\abs}[1]{\left\lvert #1\right\rvert}
\newtheorem{prop}{Proposition}
\newtheorem{thm}{Theorem}
\newtheorem{ex}{Example}
\newcommand{\real}{\mathbb{R}}
\newcommand{\pdiff}[2]{\frac{\partial #1}{\partial #2}}
\newcommand{\mpdiff}[3]{\frac{\partial^#1 #2}{\partial #3^#1}}
\begin{document}
An important characterization of context-free languages is captured in what is known as the Chomsky-Sch\"utzenberger theorem.  It shows the intimate connection between context-free and parenthesis languages.

\begin{thm}[Chomsky-Sch\"utzenberger]  A langauge $L$ over an alphabet $\Sigma$ is context-free iff for some $n\ge 0$, there there is a homomorphism $h:\Sigma_n^* \to \Sigma^*$ such that $$L=h(\boldsymbol{\operatorname{Paren}_n}\cap R),$$ where $\boldsymbol{\operatorname{Paren}_n}$ is the parenthesis language over $\Sigma_n$, and $R$ is a regular language (over $\Sigma_n$).
\end{thm}

Note that the ``if'' part is the trivial consequence of the following facts: parenthesis languages are context-free; any homomorphic image of a context-free language is context-free; any intersection of a context-free language with a regular language is context-free.

\begin{thebibliography}{9}
\bibitem{cs} N. Chomsky, M.P. Sch\"utzenberger, {\em The Algebraic Theory of Context-Free Languages, Computer Programming and Formal Systems}, North-Holland, Amsterdam (1963).
\bibitem{dk} D. C. Kozen, {\em Automata and Computability}, Springer, New York (1997).
\end{thebibliography}
%%%%%
%%%%%
\end{document}
