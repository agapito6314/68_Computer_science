\documentclass[12pt]{article}
\usepackage{pmmeta}
\pmcanonicalname{DerivationLanguage}
\pmcreated{2013-03-22 18:58:15}
\pmmodified{2013-03-22 18:58:15}
\pmowner{CWoo}{3771}
\pmmodifier{CWoo}{3771}
\pmtitle{derivation language}
\pmrecord{13}{41831}
\pmprivacy{1}
\pmauthor{CWoo}{3771}
\pmtype{Definition}
\pmcomment{trigger rebuild}
\pmclassification{msc}{68Q42}
\pmclassification{msc}{68Q45}
\pmsynonym{Szilard language}{DerivationLanguage}

\usepackage{amssymb,amscd}
\usepackage{amsmath}
\usepackage{amsfonts}
\usepackage{mathrsfs}

% used for TeXing text within eps files
%\usepackage{psfrag}
% need this for including graphics (\includegraphics)
%\usepackage{graphicx}
% for neatly defining theorems and propositions
\usepackage{amsthm}
% making logically defined graphics
%%\usepackage{xypic}
\usepackage{pst-plot}

% define commands here
\newcommand*{\abs}[1]{\left\lvert #1\right\rvert}
\newtheorem{prop}{Proposition}
\newtheorem{thm}{Theorem}
\newtheorem{ex}{Example}
\newcommand{\real}{\mathbb{R}}
\newcommand{\pdiff}[2]{\frac{\partial #1}{\partial #2}}
\newcommand{\mpdiff}[3]{\frac{\partial^#1 #2}{\partial #3^#1}}
\begin{document}
\subsubsection*{Background}

Let $G=(\Sigma,N,P,\sigma)$ be a formal grammar.  A pair $(W_1,W_2)$ of words over $\Sigma$ is said to correspond to the production $p\to q$ if $$W_1= u_1 p v_1\qquad\mbox{and}\qquad W_2=u_2 q v_2$$ for some words $u_i,v_j$ over $\Sigma$.  We also say that $(W_1,W_2)$ is a derivation step, and write $W_1\to W_2$.  

Recall that a derivation in $G$ is a finite sequence of words $$W_1,W_2, \ldots, W_n$$ over $\Sigma$ such that $W_i \Rightarrow W_{i+1}$, for $i=1,\ldots,n-1$.  The derivation is also written $$W_1\Rightarrow W_2 \Rightarrow \cdots \Rightarrow W_n.$$
The derivation above has $n-1$ steps.  Zero-step derivations are also permitted.  These are just words over $\Sigma$.

The reflexive transitive closure of $\Rightarrow$ is $\Rightarrow^*$.  Thus, $V\Rightarrow^*W$ means that there is a derivation starting with $V$ and ending with $W$.  There may be more than one derivation from $V$ to $W$.

If we consider each production as a ``letter'' in the alphabet $P$, then the above derivation can be represented by a ``word'' over $P$ in the following manner: the ``word'' is formed by taking concatenations of the ``letters'', where concatenations correspond to successive applications of productions in $P$:
$$[p_1\to q_1][p_2\to q_2]\cdots [p_{n-1}\to q_{n-1}].$$
Derivation language is thus a certain collection of derivation words, formally defined below.

\subsubsection*{Definitions}

Treating productions as ``letters'' is really nothing more than labeling each production with some symbol.  Formally, call a \emph{labeling} of an alphabet $P$ a surjection $f: F\to P$, where $F$ is some alphabet.  For any $p\in P$, a label for $p$ is an element $x\in F$ such that $f(x)=p$.  We will only be interested in injective labeling (hence a bijection) from now on.

\textbf{Definition}.  Suppose we are given a labeling $f$ of the set $P$ of productions in the grammar $G$.  Given a derivation $D:W_1 \Rightarrow W_2 \Rightarrow \ldots \Rightarrow W_n$, a \emph{derivation word} $U$ for $D$ is defined inductively as follows:
\begin{enumerate}
\item if $n=1$, then the empty word $U:=\lambda$,
\item if $n=2$, then $U:=x\in F$ is a label for a production that $W_1 \Rightarrow W_2$ corresponds to,
\item if $n>2$, then $U:=U_1U_2$, where 
\begin{itemize}
\item $U_1$ is a derivation word for the derivation $W_1 \Rightarrow \cdots \Rightarrow W_i$, 
\item $U_2$ is a derivation word for the derivation $W_i \Rightarrow \cdots \Rightarrow W_n$.
\end{itemize}
\end{enumerate}
If $U$ is a derivation word for derivation $D$, let us abbreviate this by writing $f[U]=D$.  Note that we are not applying the labeling $f$ to $U$, it is merely a notational convenience.

A derivation word is sometimes called a \emph{control word}, for it defines or controls whether and how a word may be derived from another word.  Note that any $W_1\Rightarrow^* W_2$ may correspond to several distinct derivations, and hence several distinct derivation words.  Also, the same derivation word may correspond to distinct derivations.

For example, let $G$ be a grammar over two symbols $($ and $)$ with productions $\sigma \to \lambda$, $\sigma \to (\sigma)$, and $\sigma \to \sigma \sigma$ ($G$ generates the parenthesis language $\boldsymbol{\operatorname{Paren}_1}$) Label the productions as $a,b,c$ respectively.  Then the derivation $\sigma \Rightarrow^* (()())$ correspond to derivation words $bcbbaa$ and $bcbaba$.  Notice that $\sigma \sigma \Rightarrow (\sigma) \sigma$ and $\sigma\sigma \Rightarrow \sigma (\sigma)$ both correspond to the derivation word $b$.

\textbf{Definition}.  The \emph{derivation language} of a grammar $G=(\Sigma,N,P,\sigma)$ is the set of all derivation words for derivations on words generated by $G$.  In other words, consider the labeling $f:F\to P$.  The derivation language of $G$ is the set
$$\lbrace U\in F^* \mid f[U] \mbox{ is a derivation of the form }\sigma \Rightarrow^* u \mbox{ for some }u \in N^*\rbrace.$$
The derivation language of $G$ is also called the \emph{Szilard language} of $G$, named after its inventor, and is denoted by $\operatorname{Sz}(G)$.

For example, let $G$ be the grammar over a the letter $a$, with productions given by $\sigma\to \sigma$, $\sigma \to a$.  If the productions are labeled $b,c$, then $\operatorname{Sz}(G)=\lbrace b^n c \mid n\ge 0\rbrace$.  Note that $L(G)=\lbrace a\rbrace$.  Likewise, if $G'$ is the grammar over $a$, with productions $\sigma\to A \sigma$, $A\to \lambda$, and $\sigma\to a$, labeled $p,q,r$ respectively, then $L(G')=\lbrace a\rbrace$.  However, $\operatorname{Sz}(G')$ is quite different from $\operatorname{Sz}(G)$:
\begin{eqnarray*}
\operatorname{Sz}(G')&=\lbrace u\in F^*  \mid& u=vrw\mbox{, where }v\in \lbrace p,q\rbrace^*, w\in \lbrace q\rbrace^*, \\
&& \#_u(p)=\#_u(q)\mbox{ and }\#_x(p)\ge \#_x(q)\mbox{ for all }x\le u\rbrace
\end{eqnarray*}
where 
\begin{itemize}
\item $F=\lbrace p,q,r\rbrace$,
\item $\#_u(r)$ means the number of occurrences of $r$ in word $u$,
\item $v\le u$ means that $v$ is a prefix of $u$.
\end{itemize}

\textbf{Remarks}.  Let $G$ be a formal grammar.
\begin{itemize}
\item
Some properties of $\operatorname{Sz}(G)$:
\begin{enumerate}
\item $\operatorname{Sz}(G)$ is always context-sensitive.
\item If $G$ is regular, so is $\operatorname{Sz}(G)$.
\item if $G$ is context-free, $\operatorname{Sz}(G)$ may not be; in fact, for any context-free language $L$, there is a context-free grammar $G$ such that $L=L(G)$ but $\operatorname{Sz}(G)$ is not context-free.
\item There exists a context-free language $L$ such that $\operatorname{Sz}(G)$ is not context-free for any grammar $G$ generating $L$.
\end{enumerate}
\item
However, if one considers the subset $\operatorname{Sz}_L(G)$ of $\operatorname{Sz}(G)$ consisting of all derivation words corresponding to leftmost derivations, then $\operatorname{Sz}_L(G)$ is context-free.
\item
It is an open question that, given any (context-sensitive) language $L$, whether there is a grammar $G$ such that $L=\operatorname{Sz}(G)$.
\end{itemize}

\begin{thebibliography}{9}
\bibitem{AS} A. Salomaa, {\em Formal Languages}, Academic Press, New York (1973).
\bibitem{AS} G. E. R\'{e}v\'{e}sz, {\em Introduction to Formal Languages}, Dover Publications (1991).
\end{thebibliography}
%%%%%
%%%%%
\end{document}
