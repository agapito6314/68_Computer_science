\documentclass[12pt]{article}
\usepackage{pmmeta}
\pmcanonicalname{Bubblesort}
\pmcreated{2013-03-22 12:31:07}
\pmmodified{2013-03-22 12:31:07}
\pmowner{CWoo}{3771}
\pmmodifier{CWoo}{3771}
\pmtitle{bubblesort}
\pmrecord{9}{32757}
\pmprivacy{1}
\pmauthor{CWoo}{3771}
\pmtype{Algorithm}
\pmcomment{trigger rebuild}
\pmclassification{msc}{68P10}
\pmsynonym{bubble sort}{Bubblesort}
\pmrelated{SortingProblem}
\pmrelated{StableSortingAlgorithm}
\pmrelated{InPlaceSortingAlgorithm}

\endmetadata

\usepackage{amssymb}
\usepackage{amsmath}
\usepackage{amsfonts}
%\usepackage{lbh-pseudocode}
\usepackage{program}
\begin{document}
\PMlinkescapeword{even}
\PMlinkescapeword{order}
\PMlinkescapeword{scenario}
The \emph{bubblesort} algorithm is a simple and na\"ive approach to the sorting problem.  Let $\le$ define a total ordering over a list $A$ of $n$ values.
The bubblesort consists of advancing through $A$, swapping adjacent values $A[i]$ and $A[i+1]$ if $A[i+1]\le A[i]$ holds.  By going through all of $A$ in this manner $n$ times, one is guaranteed to achieve the proper ordering.

\subsubsection*{Pseudocode}

The following is pseudocode for the bubblesort algorithm.  Note that it keeps track of whether or not any swaps occur during a traversal, so that it may terminate as soon as $A$ is sorted.

\begin{program}
\mathrm{BubbleSort}(A, n, \le)
\text{{\bf Input}: List $A$ of $n$ values.}
\text{{\bf Output}: $A$ sorted with respect to relation $\le$.}
\text{{\bf Procedure}:}
done \gets \mathbf{false}
\WHILE (\text{not\ } done) \DO
  done \gets \mathbf{true}
  \FOR i\gets 0 \TO n-1 \DO
    \IF A[i+1]\le A[i]
    \THEN \mathrm{swap}(A[i], A[i+1])
          done \gets \mathbf{false}
    \FI
  \OD
\OD    
\end{program}

\subsubsection*{Analysis}

The worst-case scenario is when $A$ is given in reverse order.  In this case,
exactly one element can be put in order during each traversal, and thus all $n$ traversals are required.  Since each traversal consists of $n-1$ comparisons, the worst-case complexity of bubblesort is $\mathcal{O}(n^2)$.
 
Bubblesort is perhaps the simplest sorting algorithm to implement.  Unfortunately, it is also the least efficient, even among $\mathcal{O}(n^2)$ algorithms.  Bubblesort can be shown to be a stable sorting algorithm (since two items of equal keys are \emph{never} swapped, initial relative ordering of items of equal keys is preserved), and it is clearly an in-place sorting algorithm.
%%%%%
%%%%%
\end{document}
