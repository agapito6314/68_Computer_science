\documentclass[12pt]{article}
\usepackage{pmmeta}
\pmcanonicalname{ClosurePropertiesOnLanguages}
\pmcreated{2013-03-22 18:59:36}
\pmmodified{2013-03-22 18:59:36}
\pmowner{CWoo}{3771}
\pmmodifier{CWoo}{3771}
\pmtitle{closure properties on languages}
\pmrecord{6}{41860}
\pmprivacy{1}
\pmauthor{CWoo}{3771}
\pmtype{Feature}
\pmcomment{trigger rebuild}
\pmclassification{msc}{68Q15}
\pmclassification{msc}{03D55}
\pmrelated{AbstractFamilyOfLanguages}

\usepackage{amssymb,amscd}
\usepackage{amsmath}
\usepackage{amsfonts}
\usepackage{mathrsfs}
\usepackage{tabls}

% used for TeXing text within eps files
%\usepackage{psfrag}
% need this for including graphics (\includegraphics)
%\usepackage{graphicx}
% for neatly defining theorems and propositions
\usepackage{amsthm}
% making logically defined graphics
%%\usepackage{xypic}
\usepackage{pst-plot}

% define commands here
\newcommand*{\abs}[1]{\left\lvert #1\right\rvert}
\newtheorem{prop}{Proposition}
\newtheorem{thm}{Theorem}
\newtheorem{ex}{Example}
\newcommand{\real}{\mathbb{R}}
\newcommand{\pdiff}[2]{\frac{\partial #1}{\partial #2}}
\newcommand{\mpdiff}[3]{\frac{\partial^#1 #2}{\partial #3^#1}}
\begin{document}
This entry lists some common closure properties on the families of languages corresponding to the Chomsky hierarchy, as well as other related families.

\begin{center}
\begin{tabular}{|c|c|c|c|c|c|c|}
\hline\hline
operation & REG & DCFL & CFL & CSL & RC & RE \\
\hline\hline
union & Y & N & Y & Y & Y & Y \\
\hline
intersection & Y & N & N & Y & Y & Y \\
\hline
set difference & Y & N & N & Y & Y & N \\
\hline
complementation & Y & Y & N & Y & Y & N \\
\hline
intersection with a regular language & Y & Y & Y & Y & Y & Y \\
\hline
concatenation & Y & N & Y & Y & Y & Y \\
\hline
Kleene star & Y & N & Y & Y & Y & Y \\
\hline
Kleene plus & Y & N & Y & Y & Y & Y \\
\hline
reversal & Y & Y & Y & Y & Y & Y \\
\hline
$\lambda$-free homomorphism & Y & N & Y & Y & Y & Y \\
\hline
homomorphism & Y & N & Y & N & N & Y \\
\hline
inverse homomorphism & Y & Y & Y & Y & Y & Y \\
\hline
$\lambda$-free substitution & Y & N & Y & Y & Y & Y \\
\hline
substitution & Y & N & Y & N & N & Y \\
\hline
$\lambda$-free GSM mapping & Y & N & Y & Y & Y & Y \\
\hline
GSM mapping & Y & N & Y & N & N & Y \\
\hline
inverse GSM mapping & Y & Y & Y & Y & Y & Y \\
\hline
$k$ limited erasing & Y & & Y & Y & & Y \\
\hline
rational transduction & Y & N & Y & N & N & Y \\
\hline
right quotient with a regular language & Y & Y & Y & N & & Y \\
\hline
left quotient with a regular language & Y & Y & Y & N & & Y \\
\hline
\end{tabular}
\end{center}
where the definitions of the cells in the top row are the following language families:


\begin{center}
\begin{tabular}{|c|c|}
\hline\hline
abbreviation & name \\
\hline\hline
REG & regular \\
\hline
DCFL & deterministic context-free \\
\hline
CFL & context-free \\
\hline
CSL & context-sensitive \\
\hline
RC & recursive \\
\hline
RE & recursively enumerable \\
\hline
\end{tabular}
\end{center}

\textbf{Remarks}.  Based on the table above, studies in closure properties have been expanded in other directions, such as
\begin{itemize}
\item closure properties of the above operations on more families of languages, such as linear languages, indexed languages, and mildly context-sensitive languages
\item given that an \emph{arbitrary} family of languages is closed under a subset of operations above, what more can one say about the family?  what other closure properties can be deduced? This is known as AFL (abstract family of languages) theory.
\item closure properties of yet more operations on languages, such as commutative closures, shuffle closures, insertion closures, deletion closures, subword extensions, prefix extensions, and suffix extensions.
\item knowing that a closure property of an operation is not satisfied for a given language family, study the closure of the property on the family.  For example, what is the Boolean closure of context-free languages?
\item knowning the a closure property of an operation is not satisfied for a given language family, study whether it is decidable that taking the operation on language(s) leads to a language in the family.  For example, is it decidable that the intersection of two context-free languages is context-free?  Furthermore, if the problem is decidable, what is its complexity?
\end{itemize}

\begin{thebibliography}{9}
\bibitem{AP} A. P. Parkes, {\em Introduction to Languages, Machines, and Logic}, Springer (2002).
\bibitem{AS} A. Salomaa, {\em Formal Languages}, Academic Press, New York (1973).
\bibitem{hu} J.E. Hopcroft, J.D. Ullman, {\em Formal Languages and Their Relation to Automata}, Addison-Wesley, (1969).
\end{thebibliography}
%%%%%
%%%%%
\end{document}
