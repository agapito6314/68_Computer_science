\documentclass[12pt]{article}
\usepackage{pmmeta}
\pmcanonicalname{CombiningURMs}
\pmcreated{2013-03-22 19:04:07}
\pmmodified{2013-03-22 19:04:07}
\pmowner{CWoo}{3771}
\pmmodifier{CWoo}{3771}
\pmtitle{combining URMs}
\pmrecord{16}{41953}
\pmprivacy{1}
\pmauthor{CWoo}{3771}
\pmtype{Application}
\pmcomment{trigger rebuild}
\pmclassification{msc}{68Q05}
\pmclassification{msc}{03D10}

\endmetadata

\usepackage{amssymb,amscd}
\usepackage{amsmath}
\usepackage{amsfonts}
\usepackage{mathrsfs}

% used for TeXing text within eps files
%\usepackage{psfrag}
% need this for including graphics (\includegraphics)
%\usepackage{graphicx}
% for neatly defining theorems and propositions
\usepackage{amsthm}
% making logically defined graphics
%%\usepackage{xypic}
\usepackage{pst-plot}

% define commands here
\newcommand*{\abs}[1]{\left\lvert #1\right\rvert}
\newtheorem{prop}{Proposition}
\newtheorem{thm}{Theorem}
\newtheorem{cor}{Corollary}
\newtheorem{ex}{Example}
\newcommand{\real}{\mathbb{R}}
\newcommand{\pdiff}[2]{\frac{\partial #1}{\partial #2}}
\newcommand{\mpdiff}[3]{\frac{\partial^#1 #2}{\partial #3^#1}}
\begin{document}
\subsubsection*{Combining URMs}

The basic idea of combining existing URMs is to produce URMs that can perform more complicated computations.  Suppose we are given two URMs $M=I_1,\ldots, I_m$ and $N=I_1',\ldots, I_n'$, define $M, N$ to be the URM $$M,N:=I_1,I_2,\ldots,I_{m+n}$$ where $I_{m+i}=I_i'$, where $i\in \lbrace 1,\ldots, n\rbrace$.  In other words, the instructions of $N$ are re-indexed and appended to the last instruction of $M$.

In theory, we would like $M, N$ to first go through the instructions of $M$, and if the computations of $M$ terminate, go through the computations of $N$.  In order to achieve this goal, modifications to $M$ and $N$ need to be made so computations of $M$ are kept apart from the computations of $N$:

\begin{enumerate}
\item If $I_k=J(i,j,q)$ is in $N$, then change $I_k$ to $J(i,j,q+m)$.  This is clearly necessary as the indices of the instructions of $N$ have been shifted by $m$.  Let the modified $N$ be denoted by $N(m)$.
\item If $I_k=J(i,j,p)$ is in $M$, and $p>m$, then change $I_k$ to $J(i,j,m+1)$.  The purpose of this change is to ensure that computations of $M$ do not accidentally jump to an instruction of $N$.  Furthermore, when $M$ halts, $N$ starts.  Call this modified machine $M'$.  It is easy to see that every $M$ can be modified to $M'$.
\end{enumerate}

When $M$ computes a function $f:\mathbb{N}^n \to \mathbb{N}$ and $N$ computes a function $g:\mathbb{N}\to \mathbb{N}$, we would further want the combined machine to compute $g\circ f$.  In essence, if $r$ is an input, and if $f(r)$ is computed by $M$, we want $f(r)$ to be the input of the machine $N$, so that $g(f(r))$ may be computed.  To get the desired result, one more modification should be made:

\begin{enumerate}
\item[3.] Suppose $M$ computes $f:\mathbb{N}^n \to \mathbb{N}$.  Define $k=\max(n,\rho(M))$.  Then $M^*$ is defined as the URM $M', Z[k]$, where $M'$ is defined in 2. above, and $Z[k]$ is the machine with instructions $Z(2),Z(3),\cdots, Z(k)$.
\end{enumerate}
In other words, whenever $f(r)$ is computed by $M$, $M^*$ resets the contents of all registers potentially used by $M$ to $0$, except the first one, which contains the result $f(r)$.  As a result, the input for $N$ has $0$ in all but possibly the first register.

With the three modifications above, let us define $M\circ N$ to be the machine $M^* , N(|M|)$.  It is easy to see that $\circ$ is an associative operation.  We can summarize our discussion above into the following:

\begin{prop}  Let $M,N$ be URMs that compute $f:\mathbb{N}^n \to \mathbb{N}$ and $g:\mathbb{N}\to \mathbb{N}$ respectively, then $M\circ N$ computes $g\circ f$.  In other words, if $f$ and $g$ are URM computable, so is $g\circ f$. \end{prop}

For example, if $M$ computes $f(r,s):=r+s$ for any $r,s\in \mathbb{N}$, and $N$ computes $g(r):=r\dot{-} 1$ for any $r\in \mathbb{N}$, then $M\circ N$ computes $(g\circ f)(r,s)=(r+s)\dot{-} 1$.  In addition, with input $r$, the URM 
$$\underbrace{N\circ N \circ \cdots \circ N}_s$$ computes $g^s(r):=r\dot{-} s$.  Note that $g^s$ is unary, and therefore not the same as the binary operation $h(r,s):=r\dot{-} s$.  To construct a URM computing $h$ using the URM $N$, the method described above does not work, and a more general construction is needed.

\subsubsection*{URM as a Subroutine}

Another way of generating URMs from existing ones is by inserting the existing URMs, so called subroutines, as part of a larger URM.  Given a URM $M=I_1,\ldots, I_m$, a \emph{subroutine} of $M$ is defined as a block of consecutive instructions in $M$.  Based on this definition, we see that combining machines can be seen as a special case: the URMs $M$ and $N$ are subroutines of the URM $M, N$.  In fact, each instruction in a URM can be considered as a subroutine.

As in the case with combining two URMs, when inserting a URM as a subroutine in a larger URM, one would like to keep the computations done within the subroutine isolated from the remaining computations.  Since only one tape is allowed, one can allocate a portion of the tape reserved only for subroutine computations, another portion for storage, while the rest for computations by the main program.  This can be done as follows:
\begin{enumerate}
\item[4.] For any URM $N$, define $\rho^*(N):=\max(\rho(N),n(N))$, where $\rho(N)$ is the number of registers used for computations by $N$, and $n(N)$ is the arity of the function computed by $N$.
\item[5.] Suppose $N_1,\ldots, N_k$ are to be inserted as subroutines in a program $M$, define $$n:=\max \lbrace \rho^*(N_i) \mid i=1,\ldots, k\rbrace.$$ Allocate the first $n$ registers for computations by the subroutines $N_i$.
\item[6.] Allocate the next $n^*$ registers for storage, where $n^*=n_1 + \cdots + n_k$.
\item[7.] For any URM $N$, and any positive integers $R_1,\ldots,R_s,R_t$, define the following URM:
$$N[R_1,\ldots, R_s; R_t]:=T(R_1,1),T(R_2,2),\ldots, T(R_s,s), N, T(1,R_t),Z(1),\ldots, Z(k),$$
where $k=\rho^*(N)$.
\item[8.] Computations of by the remaining instructions of $M$ will take place in registers $n+n^*+1$ or beyond.
\item[9.] For each $N_i$ to be inserted, insert $N[R_1,\ldots, R_s; R_t]$ instead, where each $R_j$ is determined by the computations done by instructions just prior to the point of insertion.
\end{enumerate}
The above only serves as a guideline, not a definite rule to follow.  The following example serves as an illustration: let $N_1,\ldots,N_k$ be URMs that compute $m$-ary functions $f_1,\ldots,f_k$, and let $M$ be a URM that computes a $k$-ary function $g$.  Define an $m$-ary function $h:\mathbb{N}^m\to \mathbb{N}$ as the composition of the $f$'s and $g$: $$h(r_1,\ldots, r_m):=g(f_1(r_1,\ldots, r_m),\ldots,f_k(r_1,\ldots, r_m)),$$ provided, of course, that the right hand side is defined.  We show that $h$ is URM-computable by finding a suitable URM $P$ that computes $h$:  
\begin{enumerate}
\item Given input $r$ occupying the first $n$ registers, first copy contents of the input to the storage area by the following instructions: $$P_1:=T(1,n+1),T(2,n+2),\ldots, T(m,n+m)$$
\item Insert programs $$P_2:=N_1[n+1,\ldots, n+m; n+m+1], \ldots, N_k[n+1,\ldots, n+m; n+m+k]$$
This has the effect of computing $f_1(r_1,\ldots, r_m),\ldots,f_k(r_1,\ldots, r_m)$, and putting the results in registers $n+m+1,n+m+2,\ldots, n+m+k$.
\item Insert the program $$P_3:=M[n+m+1,\ldots, n+m+k;1]$$ so the results are copied to the first $k$ registers, and $g(f_1(r_1,\ldots, r_m),\ldots,f_k(r_1,\ldots, r_m))$ is computed.
\item Then $P:=P_1,P_2,P_3$ computes $h$.
\end{enumerate}
Note that $r\in \operatorname{dom}(h)$ iff $N_1(r)\downarrow, \ldots, N_k(r)\downarrow$ and $M(f_1(r_1,\ldots, r_m),\ldots,f_k(r_1,\ldots, r_m))\downarrow$.

What we have shown above is summarized as follows:
\begin{prop} If $m$-ary functions $f_1,\ldots,f_k$ and a $k$-ary function $g$ are URM-computable, then so is $g(f_1,\ldots, f_k)$. \end{prop}

As a corollary, the following function is URM-computable:
\begin{cor} If $f(x_1,\ldots,x_n)$ is URM-computable, so is $f(x_{\sigma(1)},\ldots, x_{\sigma(n)})$, where $\sigma$ is a permutation on $\lbrace 1,\ldots, n\rbrace$. \end{cor}
\begin{proof} For each $i\in \lbrace 1,\ldots, n\rbrace$, the function $g_i(x_1,\ldots, x_n)=x_{\sigma(i)}$ is computed by $T(\sigma(i),1)$, and $f(x_{\sigma(1)},\ldots, x_{\sigma(n)})=f(g_1,\ldots,g_n)$.  \end{proof}

\begin{thebibliography}{9}
\bibitem{nc} N. Cutland, {\em Computability: An Introduction to Recursive Function Theory}. Cambridge University Press, (1980).
\end{thebibliography}
%%%%%
%%%%%
\end{document}
