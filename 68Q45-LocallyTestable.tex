\documentclass[12pt]{article}
\usepackage{pmmeta}
\pmcanonicalname{LocallyTestable}
\pmcreated{2013-03-22 18:59:03}
\pmmodified{2013-03-22 18:59:03}
\pmowner{CWoo}{3771}
\pmmodifier{CWoo}{3771}
\pmtitle{locally testable}
\pmrecord{7}{41849}
\pmprivacy{1}
\pmauthor{CWoo}{3771}
\pmtype{Definition}
\pmcomment{trigger rebuild}
\pmclassification{msc}{68Q45}
\pmclassification{msc}{68Q42}
\pmsynonym{k-testable}{LocallyTestable}
\pmdefines{$k$-testable}

\endmetadata

\usepackage{amssymb,amscd}
\usepackage{amsmath}
\usepackage{amsfonts}
\usepackage{mathrsfs}

% used for TeXing text within eps files
%\usepackage{psfrag}
% need this for including graphics (\includegraphics)
%\usepackage{graphicx}
% for neatly defining theorems and propositions
\usepackage{amsthm}
% making logically defined graphics
%%\usepackage{xypic}
\usepackage{pst-plot}

% define commands here
\newcommand*{\abs}[1]{\left\lvert #1\right\rvert}
\newtheorem{prop}{Proposition}
\newtheorem{thm}{Theorem}
\newtheorem{ex}{Example}
\newcommand{\real}{\mathbb{R}}
\newcommand{\pdiff}[2]{\frac{\partial #1}{\partial #2}}
\newcommand{\mpdiff}[3]{\frac{\partial^#1 #2}{\partial #3^#1}}
\begin{document}
A regular language $L$ over an alphabet $\Sigma$ is \emph{locally testable} if, loosely speaking, testing whether or not an arbitrary word $u$ (over $\Sigma$) is in $L$ is completely determined by its subwords of some fixed length.  The name \emph{locally testable} comes from the fact that properties of $u$, and not $L$, determine the membership of $u$ in $L$.

To formalize this notion, we first define, for any word $u$ over $\Sigma$, the set $\operatorname{sw}_k(u)$ of all subwords of $$\#u\#$$ of length $k$, where $\#$ is a symbol not in $\Sigma$.

\textbf{Definition}.  We say that a regular language $L$ is \emph{$k$-testable} if for any $u,v\in \Sigma^*$ such that $$\operatorname{sw}_k(u)=\operatorname{sw}_k(v),$$ we have $u\in L$ iff $v\in L$.  The equation above says three things at once: 
\begin{itemize}
\item the set of subwords of $u$ of length $k$ is equal to the set of subwords of $v$ of length $k$,
\item the prefix of $u$ of length $k$ is equal to the prefix of $v$ of length $k$, and
\item the suffix of $u$ of length $k$ is equal to the suffix of $v$ of length $k$.
\end{itemize}
We say that $L$ is \emph{locally testable} if it is $k$-testable for some $k\ge 0$.

If we denote $\mathscr{T}(k)$ the family of all $k$-testable languages, and $\mathscr{T}(\infty)$ the family of all locally testable languages, then $$\mathscr{T}(\infty)=\bigcup_{k=0}^{\infty} \mathscr{T}(k).$$

Note that there are only two $0$-testable languages: $\Sigma^*$ and $\varnothing$.

\begin{prop}  Let $\mathscr{D}$ be the family of definite languages.  Then
\begin{enumerate}
\item $\mathscr{T}(k)\subset \mathscr{T}(k+1)$ for any $k\ge 0$, and the inclusion is strict.
\item $\mathscr{D}$ and $\mathscr{T}(k)$ are not comparable for any $k> 0$.  In other words, for every $k$, there is a $k$-testable language that is not definite, and a definite language that is not $k$-testable.
\item $\mathscr{D}\subset \mathscr{T}(\infty)$, and the inclusion is strict.
\end{enumerate}
\end{prop}

We only sketch the proof here.  For the first assertion, note that for every $k\ge 0$, $$\operatorname{sw}_{k+1}(u)=\operatorname{sw}_{k+1}(v)\quad\mbox{implies}\quad \operatorname{sw}_k(u)=\operatorname{sw}_k(v).$$  In addition, the language $\lbrace ab^k\rbrace^*$ is $(k+1)$-testable but not $k$-testable.  For the second statement, note that $\lbrace ab^k\rbrace^*$ is not definite for any $k\ge 0$.  On the other hand, a finite language containing a single word of length $k+1$ is not $k$-testable.  The last assertion is a direct consequence of the second one.

Thus, the families $\mathscr{T}(k)$ provide us with an example of an infinite chain of subfamilies of the family of regular languages.

With regard to the closure properties in $\mathscr{T}(k)$, it is easily see that $\mathscr{T}(k)$ for all $k\ge 0$ including $k=\infty$, is closed under complementation and intersection, and hence all Boolean operations.  The star-closure of $\mathscr{T}(\infty)$ is $\mathscr{R}$, the family of all regular languages.

\textbf{Remark}.  Every locally testable language is star-free, but not conversely.

\begin{thebibliography}{9}
\bibitem{AS} A. Salomaa, {\em Formal Languages}, Academic Press, New York (1973).
\end{thebibliography}
%%%%%
%%%%%
\end{document}
