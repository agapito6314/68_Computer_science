\documentclass[12pt]{article}
\usepackage{pmmeta}
\pmcanonicalname{InsertionSort}
\pmcreated{2013-03-22 11:44:38}
\pmmodified{2013-03-22 11:44:38}
\pmowner{mathcam}{2727}
\pmmodifier{mathcam}{2727}
\pmtitle{insertion sort}
\pmrecord{16}{30181}
\pmprivacy{1}
\pmauthor{mathcam}{2727}
\pmtype{Algorithm}
\pmcomment{trigger rebuild}
\pmclassification{msc}{68P10}
\pmclassification{msc}{55U40}
\pmclassification{msc}{55U99}
\pmclassification{msc}{55U15}
\pmclassification{msc}{55U10}
\pmclassification{msc}{55P20}
\pmclassification{msc}{55-00}
\pmclassification{msc}{85-00}
\pmclassification{msc}{83-02}
\pmrelated{SortingProblem}
\pmrelated{BinarySearch}
\pmrelated{SelectionSort}

% this is the default PlanetMath preamble.  as your knowledge
% of TeX increases, you will probably want to edit this, but
% it should be fine as is for beginners.

% almost certainly you want these
\usepackage{amssymb}
\usepackage{amsmath}
\usepackage{amsfonts}
\usepackage{amsthm}

% used for TeXing text within eps files
%\usepackage{psfrag}
% need this for including graphics (\includegraphics)
%\usepackage{graphicx}
% for neatly defining theorems and propositions
%\usepackage{amsthm}
% making logically defined graphics
%%%%%%\usepackage{xypic}

% there are many more packages, add them here as you need them

% define commands here

\newcommand{\mc}{\mathcal}
\newcommand{\mb}{\mathbb}
\newcommand{\mf}{\mathfrak}
\newcommand{\ol}{\overline}
\newcommand{\ra}{\rightarrow}
\newcommand{\la}{\leftarrow}
\newcommand{\La}{\Leftarrow}
\newcommand{\Ra}{\Rightarrow}
\newcommand{\nor}{\vartriangleleft}
\newcommand{\Gal}{\text{Gal}}
\newcommand{\GL}{\text{GL}}
\newcommand{\Z}{\mb{Z}}
\newcommand{\R}{\mb{R}}
\newcommand{\Q}{\mb{Q}}
\newcommand{\C}{\mb{C}}
\newcommand{\<}{\langle}
\renewcommand{\>}{\rangle}
\begin{document}
\newcommand{\Lindent}{0.4in}
\newenvironment{Lalgorithm}[4]{
    \textbf{Algorithm} \textsc{#1}\texttt{(#2)}\newline
    \textit{Input}: #3\newline
    \textit{Output}: #4\newline

}{}
\newenvironment{Lfloatalgorithm}[6][h]{
    \begin{figure}[#1]
    \caption{#2}
    \begin{Lalgorithm}{#3}{#4}{#5}{#6}
}{
    \end{Lalgorithm}
    \end{figure}
}
\newcommand{\Lgets}{\ensuremath{\gets}}
\newcommand{\Lgroup}[1]{\textbf{begin}\\\hspace*{\Lindent}\parbox{\textwidth}{#1}\\\textbf{end}}
\newcommand{\Lif}[2]{\textbf{if} #1 \textbf{then}\\\hspace*{\Lindent}\parbox{\textwidth}{#2}} 
\newcommand{\Lelse}[1]{\textbf{else}\\\hspace*{\Lindent}\parbox{\textwidth}{#1}}
\newcommand{\Lelseif}[2]{\textbf{else if} #1 \textbf{then}\\\hspace*{\Lindent}\parbox{\textwidth}{#2}}
\newcommand{\Lfor}[2]{\textbf{for} #1 \textbf{do}\\\hspace*{\Lindent}\parbox{\textwidth}{#2}}



\subsection*{The Problem}

See the sorting problem.

\subsection*{The Algorithm}

Suppose $L = \left\{ x_1, x_2, \dots, x_n\right\}$ is the initial list of unsorted elements.
The \emph{insertion sort} algorithm will construct a new list, containing the
elements of $L$ in order, which we will call $L'$.  The algorithm constructs this list one element at a time.

Initially $L'$ is empty.  We then take the first element of $L$ and put it in $L'$.
We then take the second element of $L$ and also add it to $L'$, placing it before any elements in $L'$ that should
come after it.  This is done one element at a time until all $n$ elements of $L$ are in $L'$, in sorted order.
Thus, each step $i$ consists of looking up the position in $L'$ where the element $x_i$ should be placed
and inserting it there (hence the name of the algorithm).
This requires a search, and then the shifting of all the elements in $L'$
that come after $x_i$ (if $L'$ is stored in an array).  If storage is in an array, then the binary search algorithm
can be used to quickly find $x_i$'s new position in $L'$.

Since at step $i$, the length of list $L'$ is $i$ and the length of list $L$ is $n - i$, we can implement this
algorithm as an in-place sorting algorithm.  Each step $i$ results in $L[1..i]$ becoming fully sorted.

\subsection*{Pseudocode}

This algorithm uses a modified binary search algorithm to find the position in $L$ where an element $key$
should be placed to maintain ordering.
                                                                                                           
\begin{Lalgorithm}{Insertion\_Sort}{L, n}{A list $L$ of $n$ elements}{The list $L$ in sorted order}        
\Lgroup{
    \Lfor{$i \gets 1\mbox{ to }n$}{\Lgroup{
        $value \gets L[i]$ \\
        $position \gets Binary\_Search(L, 1, i , value)$ \\
        \Lfor{$j \gets i\mbox{ downto }position$}{$L[j] \gets L[j - 1]$} \\
        $L[position] \gets value$
    }}
} \\
\\
\textbf{function} Binary\_Search(L, bottom, top, key) \\
\Lgroup{
    \Lif{$bottom >= top$}{$Binary\_Search \gets bottom$}
    \Lelse{\Lgroup{
        $middle \gets (bottom + top) / 2$ \\
        \Lif{$key < L[middle]$}{$Binary\_Search \gets Binary\_Search(L, bottom, middle - 1, key)$}
        \Lelse{$Binary\_Search \gets Binary\_Search(L, middle + 1, top, key)$}
    }}
}
\end{Lalgorithm}

\subsection*{Analysis}

In the worst case, each step $i$ requires a shift of $i - 1$ elements for the insertion (consider an input
list that is sorted in reverse order).  Thus the runtime complexity is $\mathcal{O}(n^2)$.  Even the optimization
of using a binary search does not help us here, because the deciding factor in this case is the insertion.
It is possible to use a data type with $\mathcal{O}(\log{n})$ insertion time, giving $\mathcal{O}(n\log{n})$ runtime,
but then the algorithm can no longer be done as an in-place sorting algorithm.  Such data structures are also quite
complicated.

A similar algorithm to the insertion sort is the selection sort, which requires fewer data movements than the insertion
sort, but requires more comparisons.

\PMlinkescapeword{type}
\PMlinkescapeword{order}
\PMlinkescapeword{structure}
\PMlinkescapeword{factor}
%%%%%
%%%%%
%%%%%
%%%%%
%%%%%
%%%%%
\end{document}
